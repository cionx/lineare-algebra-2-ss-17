\section{}





\subsection{}

\begin{lemma}
  Es sei $R$ ein Ring.
  \begin{enumerate}
    \item
      Für alle $x \in R$ gilt $0 \cdot x = 0 = x \cdot 0$.
    \item
      Für alle $x, y \in R$ gilt $(-x)y = -(xy) = x(-y)$.
  \end{enumerate}
\end{lemma}

\begin{proof}
  \begin{enumerate}
    \item
      Es gilt $0 \cdot x= (0 + 0) \cdot x  = 0 \cdot x + 0 \cdot x$, und durch Subtraktion von $0 \cdot x$ ergibt sich, dass $0 = 0 \cdot x$.
      Analog ergibt sich, dass auch $x \cdot 0 = 0$ gilt.
    \item
      Es gilt $xy + (-x)y = (x + (-x))y = 0 \cdot y = 0$, weshalb $(-x)y = -(xy)$ gilt.
      Analog ergibt sich, dass auch $x(-y) = -(xy)$ gilt.
    \qedhere
  \end{enumerate}
\end{proof}

Es gilt $0 \in \ringcenter{R}$ da $0 \cdot y = 0 = y \cdot 0$ für alle $y \in R$ gilt.
Für $x_1, x_2 \in \ringcenter{R}$ gilt
\[
    (x_1 + x_2) y
  = x_1 y + x_2 y
  = y x_1 + y x_2
  = y (x_1 + x_2) 
  \qquad
  \text{für alle $y \in R$},
\]
und somit auch $x_1 + x_2 \in \ringcenter{R}$.
Für jedes $x \in R$ gilt
\[
    (-x) y
  = -(xy)
  = -(yx)
  = y (-x)
  \qquad
  \text{für alle $y \in R$},
\]
und somit auch $-x \in \ringcenter{R}$.
Insgesamt zeigt dies, dass $\ringcenter{R}$ eine Untergruppe der additiven Gruppe von $R$ ist.
Es gilt $1 \in \ringcenter{R}$ da $1 \cdot y = y = y \cdot 1$ für alle $y \in R$ gilt.
Für alle $x_1, x_2 \in \ringcenter{R}$ gilt
\[
    (x_1 x_2) y
  = x_1 x_2 y
  = x_1 y x_2
  = y x_1 x_2
  = y (x_1 x_2)
  \qquad
  \text{für alle $y \in R$},
\]
und somit auch $x_1, x_2 \in \ringcenter{R}$.
Insgesamt zeigt dies, dass $\ringcenter{R}$ ein Unterring von $R$ ist.

Zusätzlich bemerken wir noch, dass für jedes $x \in \ringcenter{R}$ mit $x \in \unitgroup{R}$ auch $x^{-1} \in \ringcenter{R}$ gilt, denn dann gilt
\[
    x^{-1} y
  = x^{-1} y x x^{-1}
  = x^{-1} x y x^{-1}
  = y x^{-1}
  \qquad
  \text{für alle $y \in R$}.
\]





\subsection{}
Es sei $K$ ein Körper.
Wir zeigen, dass
\[
    \ringcenter{\matrices{n}{K}}
  = K \cdot I
  = \{\lambda \cdot I \suchthat \lambda \in K\}
\]
gilt, wobei $I \in \matrices{n}{K}$ die Einheitsmatrix bezeichnet.
Dass $K \cdot I \subseteq \ringcenter{\matrices{n}{K}}$ gilt, ergibt sich direkt daraus, dass
\[
    (\lambda I) A
  = \lambda A
  = A \cdot (\lambda I)
  \qquad
  \text{für alle $\lambda \in K$, $A \in \matrices{n}{K}$}
\]
gilt.
Andererseits sei $C \in \ringcenter{\matrices{n}{K}}$.
Für alle $i,j = 1, \dotsc, n$ sei $E_{ij} \in \matrices{n}{K}$ die Matrix deren $(i,j)$-ter Eintrag $1$ ist, und deren andere Einträge alle $0$ sind, d.h.\ es gilt
\[
    (E_{ij})_{kl}
  = \delta_{(i,j), (k,l)}
  = \delta_{ik} \delta_{jl}
  = \begin{cases}
      1 & \text{falls $(k,l) = (i,j)$}, \\
      0 & \text{sonst},
    \end{cases}
    \qquad
    \text{für alle $k,l = 1, \dotsc, n$}.
\]
Wir zeigen nun in zwei Schritten, dass bereits $C = \lambda \cdot I$ für ein passendes $\lambda \in K$ gilt:
Im ersten Schritt zeigen wir, dass $C$ eine Diagonalmatrix ist, und im zweiten Schritt zeigen wir dann, dass alle Diagonaleinträge von $C$ bereits gleich sind.

Wir geben die Rechnungen zunächst in einer kompakte, formellastigen Form an.
Anschließend geben wir die gleiche Argumentation noch einmal in anschaulicheren, dafür aber etwas längeren Form an.



\subsubsection{Kompakte Version}

\begin{itemize}
  \item
    Wir zeigen, dass $C$ eine Diagonalmatrix ist:
    Für jede Matrix $A \in \matrices{n}{K}$ gilt $CA = AC$.
    In Koeffizienten bedeutet dies, dass
    \begin{equation}
      \label{equation: equality in coefficients}
        \sum_{k=0}^n C_{ik} A_{kj}
      = (CA)_{ij}
      = (AC)_{ij}
      = \sum_{k=0}^n A_{ik} C_{kj}
      \quad
      \text{für alle $A \in \matrices{n}{K}$ und $i,j = 1, \dotsc n$}
    \end{equation}
    gilt.
    Indem wir die Matrix $A  = E_{ii}$ betrachten, erhalten wir dabei zum einen, dass
    \[
        \sum_{k=0}^n C_{ik} (E_{ii})_{kj}
      = \sum_{k=0}^n C_{ik} \delta_{ik} \delta_{ij}
      = \delta_{ij} C_{ii}
      \qquad
      \text{für alle $i, j = 1, \dotsc, n$}
    \]
    gilt, und zum anderen, dass
    \[
          \sum_{k=0}^n (E_{ii})_{ik} C_{kj}
        = \sum_{k=0}^n \delta_{ik} C_{kj}
        = C_{ij}
        \quad
        \text{für alle $i, j = 0, \dotsc, n$}
    \]
    gilt.
    Für alle $1 \leq i \neq j \leq n$ erhalten wir somit aus \eqref{equation: equality in coefficients}, dass $0 = \delta_{ij} C_{ii} = C_{ij}$ gilt.
    (Für $i = j$ erhalten wir nur die triviale Aussage, dass $C_{ii} = \delta_{ij} C_{ii} = C_{ij} = C_{ii}$ gilt.)
    Das zeigt, dass $C$ eine Diagonalmatrix ist.
    
  \item
    Es seien nun $\lambda_1, \dotsc, \lambda_n \in K$ die Diagonaleinträge von $C$, d.h.\ es gelte $C_{ii} = \lambda_i$ für alle $i = 1, \dotsc, n$.
    Dann gilt $C_{ik} = \delta_{ik} \lambda_i$ für alle $i, k = 1, \dotsc, n$, bzw.\ äquivalent $C_{kj} = \delta_{jk} \lambda_j$ für alle $j, k = 1, \dotsc, n$.
    Die beiden Seiten von Gleichung \eqref{equation: equality in coefficients} vereinfacht sich somit zu
    \[
        \sum_{k=0}^n C_{ik} A_{kj}
      = \sum_{k=0}^n \delta_{ik} \lambda_i A_{kj}
      = \lambda_i A_{ij}
      \quad\text{und}\quad
        \sum_{k=0}^n A_{ik} C_{kj}
      = \sum_{k=0}^n A_{ik} \delta_{jk} \lambda_j
      = \lambda_j A_{ij},
    \]
    und Gleichung \eqref{equation: equality in coefficients} selbst vereinfacht sich somit zu
    \[
        \lambda_i A_{ij}
      = \lambda_j A_{ij}
      \qquad
      \text{für alle $A \in \matrices{n}{K}$ und $i, j = 1, \dotsc, n$}.
    \]
    Indem wir die Matrix $A = E_{ij}$ mit $A_{ij} = 1$ betrachten, erhalten wir somit, dass $\lambda_i = \lambda_j$ für alle $i,j = 1, \dotsc, n$ gilt.
    Also gilt $\lambda_1 = \dotsb = \lambda_j \eqqcolon \lambda$ und somit $C = \lambda I$.
\end{itemize}



\subsubsection{Anschauliche Version}

Wir wollen zunächst eine Anschauung dafür entwickeln, wie Multiplikation mit Diagonalmatrizen funktioniert:

\begin{observation}
  \label{observation: multiplication with diagonal matrices}
  Sind $D_1 \in \matrices{m}{K}$ und $D_2 \in \matrices{n}{K}$ zwei Diagonalmatrizen
  \[
      D_1
    = \begin{pmatrix}
        \lambda_1 &         &           \\
                  & \ddots  &           \\
                  &         & \lambda_m 
      \end{pmatrix}
    \quad\text{und}\quad
      D_2
    = \begin{pmatrix}
        \mu_1 &         &       \\
              & \ddots  &       \\
              &         & \mu_n 
      \end{pmatrix},
  \]
  so lassen sich für eine beliebige Matrix $A \in \mnatrices{m}{n}{K}$ die Produkte $D_1 A$ und $A D_2$ als
  \[
      D_1 A
    = \begin{pmatrix}
        \lambda_1 A_{11}  & \cdots  & \lambda_1 A_{1n}  \\
        \vdots            & \ddots  & \vdots            \\
        \lambda_m A_{m1}  & \cdots  & \lambda_m A_{mn}
      \end{pmatrix}
    \quad\text{und}\quad
      A D_2
    = \begin{pmatrix}
        \mu_1 A_{11}  & \cdots  & \mu_n A_{1n}  \\
        \vdots        & \ddots  & \vdots        \\
        \mu_1 A_{m1}  & \cdots  & \mu_n A_{mn}
      \end{pmatrix}
  \]
  berechnen.
  Durch Multiplikation mit $D_1$ von links wird also die $i$-te Zeile von $A$ mit $\lambda_i$ multipliziert, und durch Multiplikation mit $D_2$ von rechts wird die $j$-te Spalte von $A$ mit $\mu_j$ multipliziert.
  Dies lässt sich schematisch als
  \begin{align*}
        \begin{pmatrix}
          \lambda_1 &         &           \\
                    & \ddots  &           \\
                    &         & \lambda_m
        \end{pmatrix}
        \cdot
        \renewcommand\arraystretch{0.7}
        \begin{pmatrix}  
          & & z_1     & & \\
        \cmidrule(lr){1-5}
          & & \vdots  & & \\
        \cmidrule(lr){1-5}
          & & z_m     & & 
        \end{pmatrix}
    &=  \renewcommand\arraystretch{0.6}
        \begin{pmatrix}  
          & & \lambda_1 z_1 & & \\
        \cmidrule(lr){1-5}
          & & \vdots        & & \\
        \cmidrule(lr){1-5}
          & & \lambda_m z_m & &
        \end{pmatrix},
  \shortintertext{und}
        \left(
        \begin{array}{@{} c|c|c @{}}
              &         &     \\
          s_1 & \cdots  & s_n \\
              &         &
        \end{array}
        \right)
        \cdot
        \begin{pmatrix}
          \mu_1 &         &       \\
                & \ddots  &       \\
                &         & \mu_n
        \end{pmatrix}
    &=  \left(
        \begin{array}{@{} c|c|c @{}}
                    &         &           \\
          \mu_1 s_1 & \cdots  & \mu_n s_n \\
                    &         &           \\
        \end{array}
        \right)
  \end{align*}
  darstellen.
\end{observation}

Wir zeigen nun noch einmal in zwei Schritten, dass $C = \lambda \cdot I$ für ein passendes $\lambda \in K$ gilt:
\begin{itemize}
  \item
    Wir zeigen zunächst, dass $C$ eine Diagonalmatrix ist:
    Hierfür sei $1 \leq i \leq n$.
    Dann ist $E_{ii}$ eine Diagonalmatrix, deren $i$-tere Diagonaleintrag $1$ ist, und deren Diagonaleinträge sonst alle verschwinden.
    Da $C \in \ringcenter{\matrices{n}{K}}$ gilt, erhalten wir, dass $C E_{ii} = E_{ii} C$.
    Nach Beobachtunng~\ref{observation: multiplication with diagonal matrices} entsteht dabei die Matrix $C E_{ii}$ aus $C$, indem die $i$-te Spalte unverändert bleibt, aber alle anderen Spalten durch die Nullspalte ersetzt werden.
    Analog entsteht $E_{ii} C$ aus $C$, indem die $i$-te Zeilen unverändert bleibt, aber alle anderen Zeilen durch die Nullzeile ersetzt werden.
    Anschaulich gesehen gilt also, dass
    \[
      C E_{ii}
      =
      \begin{pmatrix}
        0       & \cdots  & 0       & C_{1i}    & 0       & \cdots  & 0       \\
        \vdots  & \ddots  & \vdots  & \vdots    & \vdots  & \ddots  & \vdots  \\
        0       & \cdots  & 0       & C_{ii}    & 0       & \cdots  & 0       \\
        \vdots  & \ddots  & \vdots  & \vdots    & \vdots  & \ddots  & \vdots  \\
        0       & \cdots  & 0       & C_{ni}    & 0       & \cdots  & 0
      \end{pmatrix}
      \quad\text{und}\quad
      E_{ii} C
      = \begin{pmatrix}
        0       & \cdots  & 0       & \cdots  & 0       \\
        \vdots  & \ddots  & \vdots  & \ddots  & \vdots  \\
        0       & \cdots  & 0       & \cdots  & 0       \\
        C_{i1}  & \cdots  & C_{ii}  & \cdots  & C_{in}  \\
        0       & \cdots  & 0       & \cdots  & 0       \\
        \vdots  & \ddots  & \vdots  & \ddots  & \vdots  \\
        0       & \cdots  & 0       & \cdots  & 0
      \end{pmatrix}.
    \]
    Da nach Annahme $C E_{ii} = E_{ii} C$ gilt, erhalte wir, dass in der $i$-ten Zeile und $i$-ten Spalte von $C$ bis auf den gemeinsamen Eintrag $C_{ii}$ alle anderen Einträge verschwinden müssen, d.h.\ für alle $j = 0, \dotsc, \hat{i}, \dotsc, n$ gilt $C_{ij} = 0$ und $C_{ji} = 0$.
    Da dies für alle $i = 1, \dotsc, n$ gilt, erhalten wir, dass in jeder Spalte (und in jeder Zeile) von $C$ alle nicht-Diagonaleinträge verschwinden.
    Also ist $C$ eine Diagonalmatrix.
    
  \item
    Für alle $i = 1, \dotsc, n$ sei $\lambda_i \in K$ der $i$-te Diagonaleintrag von $C$, d.h.\ es gelte
    \[
        C
      = \begin{pmatrix}
          \lambda_1 &         &           \\
                    & \ddots  &           \\
                    &         & \lambda_n
        \end{pmatrix}.
    \]
    Wir zeigen, dass alle Diagonaleinträge von $C$ bereits gleich sind:
    Es seien $1 \leq i \neq j \leq n$.
    Der einzige nicht-verschwindende Eintrag von $E_{ij}$ befindet sich in der $i$-ten Zeile und $j$-ten Spalte von $E_{ij}$.
    Aus Beobachtung~\ref{observation: multiplication with diagonal matrices} folgt nun, dass $C E_{ij} = \lambda_i E_{ij}$ und $E_{ij} C = \lambda_j E_{ij}$.
    Anschaulich lässt sich die Anwendung von Beobachtung~\ref{observation: multiplication with diagonal matrices} als
    \begin{gather*}
        C E_{ij}
      = \begin{pmatrix}
          \lambda_1 &         &           &         &         &           \\
                    & \ddots  &           &         &         &           \\
                    &         & \lambda_i &         &         &           \\
                    &         &           & \ddots  &         &           \\
                    &         &           &         & \ddots  &           \\
                    &         &           &         &         & \lambda_n
        \end{pmatrix}
        \begin{pmatrix}
          0 &         &         &           &         &   \\
            & \ddots  &         &           &         &   \\
            &         & \ddots  & 1         &         &   \\
            &         &         & \ddots    &         &   \\
            &         &         &           & \ddots  &   \\
            &         &         &           &         & 0
        \end{pmatrix}
      = \begin{pmatrix}
          0 &         &         &           &         &   \\
            & \ddots  &         &           &         &   \\
            &         & \ddots  & \lambda_i &         &   \\
            &         &         & \ddots    &         &   \\
            &         &         &           & \ddots  &   \\
            &         &         &           &         & 0
        \end{pmatrix}
    \shortintertext{und}
        E_{ij} C
      = \begin{pmatrix}
          0 &         &         &           &         &   \\
            & \ddots  &         &           &         &   \\
            &         & \ddots  & 1         &         &   \\
            &         &         & \ddots    &         &   \\
            &         &         &           & \ddots  &   \\
            &         &         &           &         & 0
        \end{pmatrix}
        \begin{pmatrix}
          \lambda_1 &         &         &           &         &           \\
                    & \ddots  &         &           &         &           \\
                    &         & \ddots  &           &         &           \\
                    &         &         & \lambda_j &         &           \\
                    &         &         &           & \ddots  &           \\
                    &         &         &           &         & \lambda_n
        \end{pmatrix}
      = \begin{pmatrix}
          0 &         &         &           &         &   \\
            & \ddots  &         &           &         &   \\
            &         & \ddots  & \lambda_j &         &   \\
            &         &         & \ddots    &         &   \\
            &         &         &           & \ddots  &   \\
            &         &         &           &         & 0
        \end{pmatrix}
    \end{gather*}
    notieren.
    Durch Vergleich der Einträge $(C E_{ij}) = \lambda_i$ und $(E_{ij} C) = \lambda_j$ ergibt sich hieraus, dass $\lambda_i = \lambda_j$ gilt.
    Das dies für alle $1 \leq i \neq j \leq n$ gilt, muss bereits $\lambda_1 = \dotsb = \lambda_n \eqqcolon \lambda$, und somit $C = \lambda I$.
\end{itemize}



\subsubsection{Bemerkung: Verallgemeinerung auf beliebige Ringe}

% \begin{remark}
%   Die Idee der obigen Rechnung lässt sich grob wie folgt beschreiben:
%   
%   Es sei $C \in \matrices{n}{K}$.
%   Dass $C$ mit einer Matrix $A \in \matrices{n}{K}$ kommutiert, dass also $CA = AC$ gilt, ist eine zusätzliche Bedingung an $C$.
%   Dabei hängt es von der Matrix $A$ ab, wie restriktiv diese Bedingung ist:
%   \begin{itemize}
%     \item
%       Betrachtet man etwa $A = I$, so erhält man die Bedingung $C = C$.
%       Diese Bedingung liefert uns keine näheren Informationen über $C$.
%     \item
%       Betrachtet man $A = E_{ii}$ für ein $1 \leq i \leq n$, so liefert uns die Bedingungen $CA = AC$, dass in der $i$-ten Zeile und $i$-ten Spalte von $A$ bis auf den gemeinsamen Diagonaleintrag $A_{ii}$ jeder andere Diagonaleintrag verschwindet.
%     \item
%       Betrachtet man eine Diagonalmatrix $A$ mit paarweise verschiedenen Diagonaleinträgen $\lambda_1, \dotsc, \lambda_n \in K$, also
%       \[
%           A
%         = \begin{pmatrix}
%             \lambda_1 &         &           \\
%                       & \ddots  &           \\
%                       &         & \lambda_n
%           \end{pmatrix}
%           \quad\text{mit}\quad
%           \text{$\lambda_i \neq \lambda_j$ für $i \neq j$},
%       \]
%       so erhalten wir durch Beobachtung~\ref{observation: multiplication with diagonal matrices} aus der Bedingung $CA = AC$, dass
%       \[
%           \lambda_j C_{ij}
%         = (CA)_{ij}
%         = (AC)_{ij}
%         = \lambda_i C_{ij}
%         \qquad
%         \text{für alle $i, j = 1, \dotsc, n$}.
%       \]
%       Da $\lambda_i \neq \lambda_j$ für alle $1 \leq i \neq j \leq n$ gilt, folgt daraus, dass $C_{ij} = 0$ für alle $1 \leq i,j \leq n$ gilt.
%       Also muss $C$ bereits eine Diagonalmatrix sein.
%   \end{itemize}
%   Je mehr Matrizen mit $C$ kommutieren, desto mehr Bedingungen werden an $C$ gestellt, und desto mehr Restriktionen gibt es bezüglich des Aussehens von $C$.
%   
%   Ist nun $C \in \ringcenter{ \matrices{n}{K} }$ so gilt für jede Matrix $A \in \matrices{n}{K}$ die entsprechende Bedingung $A C = C A$.
%   Indem wir für $A$ „ausreichend viele“ passende Matrizen betrachten, erhalten wir unterschiedliche Restriktionen an die Form von $C$.
%   Die Hoffnung ist, hierdurch bereits die gewünschte Form $C = \lambda \cdot I$ mit $\lambda \in K$ zu erzwingen.
%   
%   Wegen der Endlichdimensionalität von $\matrices{n}{K}$ lassen sich dabei die Famlie von $n^2 \cdot |K|$ vielen Bedingungen
%   \[
%     CA = AC
%     \qquad
%     \text{für alle $A \in \matrices{n}{K}$}
%   \]
%   auf bereits $n^2$ viele Bedingungen reduzieren:
%   Ist $\{ A_1, \dotsc, A_{n^2} \}$ eine $K$-Basis von $\matrices{n}{K}$ und gilt $C A_i = A_i C$ für alle $i = 1, \dotsc, n^2$, so gilt bereits $C A = A C$ für alle $A \in \matrices{n}{K}$, denn mit $A = \sum_{i=1}^{n^2} \lambda_i A_i$ erhält man die Gleichungskette
%   \[
%       C A
%     = C \left( \sum_{i=1}^{n^2} \lambda_i A_i \right)
%     = \sum_{i=1}^{n^2} \lambda_i C A_i
%     = \sum_{i=1}^{n^2} \lambda_i A_i C
%     = \left( \sum_{i=1}^{n^2} \lambda_i A_i \right) C
%     = A C.
%   \]
%   
%   Betrachtet man nun die (naheliegende) Basis $\basis{B} = \{ E_{ij} \suchthat i, j = 1, \dotsc n \}$ von $\matrices{n}{K}$, so gilt für $C \in \matrices{n}{K}$ also, dass genau dann $C \in \ringcenter{ \matrices{n}{K} }$ gilt, wenn $C E_{ij} = E_{ij} C$ für alle $i, j = 1, \dotsc, n$ gilt.
%   Da die Matrizen aus $\basis{B}$ eine möglichst einfache Form haben, sind die Bedingungen $C E_{ij} = E_{ij} C$ dabei ebenfalls möglich einfach.
% \end{remark}

\begin{remark}
  Wir haben in unseren Beweis nur genutzt, dass $K$ ein kommutativer Ring ist.
  Mit unveränderten Beweis ergibt sich deshalb, dass
  \begin{alignat}{2}
    \label{equation: center of matrix ring for commutative ground rings}
      \ringcenter{ \matrices{n}{R} }
    &= R \cdot I
    &
    & \text{für jeden kommutativen Ring $R$}
  \intertext{
  gilt.
  Möchte man auch nicht-kommutative Ringe betrachten, so lässt sich \eqref{equation: center of matrix ring for commutative ground rings} zu
  }
    \label{equation: center of matrix ring for arbitrary ground rings}
      \ringcenter{ \matrices{n}{R} }
    &= \ringcenter{R} \cdot I
    &\qquad
    &\text{für jeden Ring $R$}
  \end{alignat}
  erweitern;
  ist dabei $R$ kommutativ, so gilt $\ringcenter{R} = R$ (dies ist nur eine Umformulierung der Kommutativität von $R$), und \eqref{equation: center of matrix ring for arbitrary ground rings} vereinfacht sich wieder zu \eqref{equation: center of matrix ring for commutative ground rings}.
  
  Ein Beweis für \eqref{equation: center of matrix ring for arbitrary ground rings} ergibt sich durch leichte Abänderung, bzw.\ Erweiterung unseres Beweises für \eqref{equation: center of matrix ring for commutative ground rings}:
  Es ergibt sich wie zuvor, dass $\ringcenter{R} \cdot I \subseteq \ringcenter{ \matrices{n}{R} }$ gilt.
  
  Zum Beweis der Inklusion $\ringcenter{ \matrices{n}{R} } \subseteq \ringcenter{R} \cdot I$ zeigt man für $C \in \ringcenter{ \matrices{n}{R} }$ ebenfalls zunächst, dass
  \begin{equation}
    \label{equation: central matrices are scalar matrices}
    C = r \cdot I
    \qquad
    \text{für ein $r \in R$}
  \end{equation}
  gilt.
  Im Fall $R = K$ haben wir für unseren Beweis von \eqref{equation: central matrices are scalar matrices} die Kommutativität von $K$ nicht genutzt;
  für den allgemeinen Fall können wir deshalb unseren zweischritten Beweis unverändert übernehmen.
  
  In einem dritten Schritt muss nun noch gezeigt werden, dass bereits $r \in \ringcenter{R}$ gilt.
  Dies ergibt sich daraus, dass für alle $s \in R$ die Gleichheitskette
  \[
      (rs) \cdot I
    = (r \cdot I)(s \cdot I)
    = C (s \cdot I)
    = (s \cdot I) C
    = (s \cdot I) (r \cdot I)
    = (sr) \cdot I
  \]
  gilt, und somit durch Vergleich der Diagonaleinträge bereits $rs = sr$ für alle $s \in R$.
\end{remark}





\subsection{}

Wir zeigen, dass
\[
    \ringcenter{ \polynomialring{R}{t} }
  = \left\{
      \sum_{i=0}^\infty a_i \in \polynomialring{R}{t}
    \suchthatscale
      \text{$a_i \in \ringcenter{R}$ für alle $i$}
    \right\}
  = \polynomialring{ \ringcenter{R} }{t}
\]
gilt.
Die zweite Gleichheit gilt, da es sich hierbei um die Definition von $\polynomialring{ \ringcenter{R} }{t}$ handelt.

Ist $p = \sum_{i=0}^\infty a_i t^i \in \polynomialring{ \ringcenter{R} }{t}$, so gilt $a_i b = b a_i$ für alle $b \in R$ und $i \geq 0$.
Für jedes $q = \sum_{j=0}^\infty b_j t^j \in \polynomialring{R}{t}$ gilt deshalb
\[
     p q 
   = \left( \sum_{i=0}^\infty a_i t^i \right) \left( \sum_{j=0}^\infty b_j t^j \right)
   = \sum_{i,j=0}^\infty a_i b_j t^{i+j}
   = \sum_{j,i=0}^\infty b_j a_i t^{j+i}
   = \left( \sum_{j=0}^\infty b_j t^j \right) \left( \sum_{i=0}^\infty a_i t^i \right)
   = q p.
\]
Also gilt $p \in \ringcenter{ \polynomialring{R}{t} }$.

Andererseits sei $p \in \ringcenter{ \polynomialring{R}{t} }$.
Dann gilt $p q = q p$ für jedes $q \in \polynomialring{R}{t}$.
Für alle $b \in R$ gilt dann
\[
    \sum_{i=0}^\infty (a_i b) t^i
  = \left( \sum_{i=0}^\infty a_i t^i \right) \left( b t^0 \right)
  = \left( b t^0 \right) \left( \sum_{i=0}^\infty a_i t^i \right)
  = \sum_{i=0}^\infty (b a_i) t^i.
\]
Für jedes $i \geq 0$ gilt also $b a_i = a_i b$ für alle $b \in R$ (denn zwei Polynome sind genau dann gleich, wenn alle ihre Koeffizienten gleich sind).
Dies bedeutet gerade, dass $a_i \in \ringcenter{R}$ für alle $i \geq 0$ gilt.
