\section{}





\subsection{}
Vollständigkeithalber geben wir die Konstruktion hier noch einmal an, wobei wir im Folgenden $r, s \in \naturals$ fixieren.



\subsubsection{Eindeutigkeit}

\begin{lemma}
  \label{lemma: multilinear maps are uniquely determined by generating sets}
  Es seien $V_1, \dotsc, V_n, W$ Vektorräume, und für jedes $i = 1, \dotsc, n$ sei $E_i \subseteq V_i$ ein Erzeugendensystem von $V$.
  Sind $f, g \colon V_1 \times \dotsb \times V_n \to W$ zwei multilineare Abbildungen mit
  \[
      f( e_1, \dotsc, e_n )
    = g( e_1, \dotsc, e_n)
    \qquad
    \text{für alle $e_1 \in E_1, \dotsc, e_n \in E_n$},
  \]
  so gilt bereits $f = g$.
\end{lemma}

\begin{proof}
  Es seien $v_1 \in V_1, \dotsc, v_n \in V_n$.
  Für jedes $i = 1, \dotsc, n$ lässt sich $v_i$ als $v_i = \sum_{e_i \in E_i} \lambda^{(i)}_{e_i} e_i$ darstellen.
  Wegen der Multilinearität von $f$ und $g$ folgt damit, dass
  \begin{align*}
        f(v_1, \dotsc, v_n)
    &=  f\left( \sum_{e_1 \in E_1} \lambda^{(1)}_{e_1} e_1, \dotsc, \sum_{e_n \in E_n} \lambda^{(n)}_{e_n} e_n \right)
     =  \sum_{\substack{i = 1, \dotsc, n \\ e_i \in E_i}} \lambda^{(1)}_{e_1} \dotsm \lambda^{(n)}_{e_n} f(e_1, \dotsc, e_n)
    \\
    &=  \sum_{\substack{i = 1, \dotsc, n \\ e_i \in E_i}} \lambda^{(1)}_{e_1} \dotsm \lambda^{(n)}_{e_n} g(e_1, \dotsc, e_n)
     =  \dotsb
     =  g(v_1, \dotsc, v_n)
  \end{align*}
  gilt.
  Somit gilt bereits $f = g$.
\end{proof}

Aus Lemma~\ref{lemma: multilinear maps are uniquely determined by generating sets} ergibt sich direkt die Eindeutigkeit der gesuchten bilinearen Abbildung $\wedge$, denn $\exteriorpower{r}{V}$ wird von den Elementen $v_1 \wedge \dotsb \wedge v_r$ mit $v_i \in V$ erzeugt, und $\exteriorpower{s}{V}$ wird von den Elementen $w_1 \wedge \dotsb \wedge w_s$ mit $w_j \in V$ erzeugt, und das Verhalten von $\wedge$ auf diesen Erzeugern ist eindeutig festgelegt.



\subsubsection{Existenz}
Wir konstruieren die gewünschte Abbildung schrittweise.
Dabei betrachten wir im Folgenden nur den Fall $r, s > 0$.
Die Fälle $r = 0$ und $s = 0$ funktionieren analog.

\begin{enumerate}
  \item
    Wir betrachten zunächst die Abbildung
    \[
              m_1
      \colon  V^r \times V^s
      \to     \exteriorpower{r+s}{V},
      \quad   ((v_1, \dotsc, v_r), (w_1, \dotsc, w_s))
      \mapsto v_1 \wedge \dotsb \wedge v_r \wedge w_1 \wedge \dotsb \wedge w_s.
    \]
    (Der Buchstabe $m$ steht hier für „Multiplikation“.)
  \item
    Für jedes $x = (v_1, \dotsc, v_r) \in V^r$ sei
    \begin{align*}
                  \tilde{\lambda}_x
       \coloneqq  m_1(x, -)
       \colon     V^s
      &\to        \exteriorpower{r+s}{V},
      \\
                  (w_1, \dotsc, w_s)
      &\mapsto    m_1(x, (w_1, \dotsc, w_s))
       =          v_1 \wedge \dotsb \wedge v_r \wedge w_1 \wedge \dotsb \wedge w_s.
    \end{align*}
    (Der Buchstabe $\lambda$ steht hier für „Linksmultiplikation“.)
    Der Ausdruck $v_1 \wedge \dotsb \wedge v_r \wedge w_1 \wedge \dotsb \wedge w_s$ ist in jeder seiner Komponenten multilinear, sowie insgesamt alternierend;
    die Abbildung $\tilde{\lambda}_x$ ist deshalb multilinear und alternierend.
    Für jedes $x = (v_1, \dotsc, v_r) \in V^r$ erhalten wir nach der universellen Eigenschaft der äußeren Potenz, dass $\tilde{\lambda}_x$ eine (eindeutige) lineare Abbildung $\lambda_x \colon \exteriorpower{s}{V} \to \exteriorpower{r+s}{V}$ induziert, so dass
    \[
        \lambda_x( w_1 \wedge \dotsb \wedge w_s )
      = \tilde{\lambda}_x(w_1, \dotsc, w_s)
      = v_1 \wedge \dotsb \wedge v_r \wedge w_1 \wedge \dotsb \wedge w_s
    \]
    für alle $w_1, \dotsc, w_s \in V$ gilt.
    Ein allgemeines Element $y \in \exteriorpower{s}{V}$ ist dabei von der Form $y = \sum_j b_j w^{(j)}_1 \wedge \dotsb \wedge w^{(j)}_s$, und wegen der Linearität von $\lambda_x$ gilt dann
    \[
        \lambda_x( y )
      = \lambda_x\left( \sum_j b_j w^{(j)}_1 \wedge \dotsb \wedge w^{(j)}_s \right)
      = \sum_j b_j v_1 \wedge \dotsb \wedge v_r \wedge w^{(j)}_1 \wedge \dotsb \wedge w^{(j)}_s.
    \]
    Durch Zusammenfügen der Funktionen $\lambda_x$, $x \in V^r$ erhalten wir eine wohldefinierte Funktion
    \[
              m_2
      \colon  V^r \times \exteriorpower{s}{V}
      \to     \exteriorpower{r+s}{V},
      \quad   (x, y)
      \mapsto \lambda_x(y).
    \]
    Für $x = (v_1, \dotsc, v_r)$ und $y = \sum_j b_j w^{(j)}_1 \wedge \dotsb \wedge w^{(j)}_s$ gilt dabei
    \[
        m_2\left( (v_1, \dotsc, v_r), \sum_j b_j w^{(j)}_1 \wedge \dotsb \wedge w^{(j)}_s \right)
      = \lambda_x(y)
      = \sum_j b_j v_1 \wedge \dotsb \wedge v_r \wedge w^{(j)}_1 \wedge \dotsb \wedge w^{(j)}_s.
    \]
    
  \item
    Für jedes $y \in \exteriorpower{s}{V}$ betrachten wir nun die Abbildung
    \[
                \tilde{\rho}_y
      \coloneqq m_2(-, y)
      \colon    V^r
      \to       \exteriorpower{r+s}{V},
      \quad     x
      \mapsto   m_2(x,y).
    \]
    (Der Buchstabe $\rho$ steht hier für „Rechtsmultiplikation“.)
    Gilt $y = \sum_j b_j w^{(j)}_1 \wedge \dotsb \wedge w^{(j)}_s$, so gilt
    \[
        \tilde{\rho}_y(v_1, \dotsc, v_r)
      = \sum_j b_j v_1 \wedge \dotsb \wedge v_r \wedge w^{(j)}_1 \wedge \dotsb \wedge w^{(j)}_s
      \qquad
      \text{für alle $v_1, \dotsc, v_r \in V$}.
    \]
    Für jedes $j$ ist dabei der Ausdruck $v_1 \wedge \dotsb \wedge v_r \wedge w^{(j)}_1 \wedge \dotsb \wedge w^{(j)}_s$ multilinear und alternierend in den Einträgen $v_1, \dotsc, v_r$;
    deshalb ist auch $\tilde{\rho}_y$ multilinear und alternierend.
    Nach der universellen Eigenschaft der äußeren Potenz erhalten wir für jedes $y \in \exteriorpower{s}{V}$ eine (eindeutige) induzierte lineare Abbildung $\rho_y \colon \exteriorpower{r}{V} \to \exteriorpower{r+s}{V}$ mit
    \[
        \rho_y(v_1 \wedge \dotsb \wedge v_r)
      = \sum_j b_j v_1 \wedge \dotsb \wedge v_r \wedge w^{(j)}_1 \wedge \dotsb \wedge w^{(j)}_s
      \qquad
      \text{für alle $v_1, \dotsc, v_r \in V$}.
    \]
    Für eine beliebiges Element $x \in \exteriorpower{r}{V}$ von der Form $x = \sum_i a_i v^{(i)}_1 \wedge \dotsb \wedge v^{(i)}_r$ gilt dabei wegen der Linearität von $\rho_y$, dass
    \begin{align*}
          \rho_y(x)
      =   \rho_y\left( \sum_i a_i v^{(i)}_1 \wedge \dotsb \wedge v^{(i)}_r \right)
      &=  \sum_i a_i \left(
                       \sum_j b_j v^{(i)}_1 \wedge \dotsb \wedge v^{(i)}_r \wedge w^{(j)}_1 \wedge \dotsb \wedge w^{(j)}_s
                     \right)
      \\
      &=  \sum_{i,j} a_i b_j v^{(i)}_1 \wedge \dotsb \wedge v^{(i)}_r \wedge w^{(j)}_1 \wedge \dotsb \wedge w^{(j)}_s.
    \end{align*}
    Durch Zusammenfügen der Funktionen $\rho_y$, $y \in \exteriorpower{s}{V}$ erhalten wir eine wohldefinierte Funktion
    \[
              m_3
      \colon  \exteriorpower{r}{V} \times \exteriorpower{s}{V}
      \to     \exteriorpower{r+s}{V},
      \quad   (x,y)
      \mapsto \rho_y(x).
    \]
    Für $x = \sum_i a_i v^{(i)}_1 \wedge \dotsb \wedge v^{(i)}_r$ und $y = \sum_j b_j w^{(j)}_1 \wedge \dotsb \wedge w^{(j)}_s$ ist $m_3(x,y)$ durch
    \[
        m_3\left( \sum_i a_i v^{(i)}_1 \wedge \dotsb \wedge v^{(i)}_r, \sum_j b_j w^{(j)}_1 \wedge \dotsb \wedge w^{(j)}_s \right)
      = \sum_{i,j} a_i b_j v^{(i)}_1 \wedge \dotsb \wedge v^{(i)}_r \wedge w^{(j)}_1 \wedge \dotsb \wedge w^{(j)}_s
    \]
    gegeben.
\end{enumerate}

Es sei nun $\wedge \coloneqq m_3$.
Für alle $v_1, \dotsc, v_r, w_1, \dotsc, w_s \in V$ gilt nun, dass
\[
    \wedge( v_1 \wedge \dotsb \wedge v_r, w_1 \wedge \dotsb \wedge w_s )
  = v_1 \wedge \dotsb \wedge v_r \wedge w_1 \wedge \dotsb \wedge w_s.
\]
Es bleibt zu zeigen, dass $\wedge$ bilinear ist.
Wir zeigen die Linearität im ersten Argument, die Linearität im zweiten Argument ergibt sich dann analog:

Es seien $x, y \in \exteriorpower{r}{V}$, $z \in \exteriorpower{s}{V}$ und $\lambda \in K$.
Zum Rechnen wählen wir explizite Darstellungen $x = \sum_{i=1}^n a_i v^{(i)}_1 \wedge \dotsb \wedge v^{(i)}_r$, $y = \sum_{i=n+1}^m a_i v^{(i)}_1 \wedge \dotsb \wedge v^{(i)}_r$ und $z = \sum_{j=1}^l b_j w^{(j)}_1 \wedge \dotsb \wedge w^{(j)}_s$. (Man beachte die unterschiedlichen Laufindizes für die Darstellungen von $x$ und $y$.)
Damit erhalten wir nun zum einen, dass
\begin{align*}
   &\,  (x + y) \wedge z
  \\
  =&\,  \left(
            \sum_{i=1}^n a_i v^{(i)}_1 \wedge \dotsb \wedge v^{(i)}_r
          + \sum_{i=n+1}^m a_i v^{(i)}_1 \wedge \dotsb \wedge v^{(i)}_r
        \right)
        \wedge
        \left(
          \sum_{j=1}^l b_j w^{(j)}_1 \wedge \dotsb \wedge w^{(j)}_s
        \right)
  \\
  =&\,  \left(
          \sum_{i=1}^m a_i v^{(i)}_1 \wedge \dotsb \wedge v^{(i)}_r
        \right)
        \wedge
        \left(
          \sum_{j=1}^l b_j w^{(j)}_1 \wedge \dotsb \wedge w^{(j)}_s
        \right)
  \\
  =&\,  \sum_{i=1}^m \sum_{j=1}^l a_i b_j v^{(i)}_1 \wedge \dotsb \wedge v^{(i)}_r \wedge w^{(j)}_1 \wedge \dotsb \wedge w^{(j)}_s
  \\
  =&\,    \sum_{i=1}^n \sum_{j=1}^l a_i b_j v^{(i)}_1 \wedge \dotsb \wedge v^{(i)}_r \wedge w^{(j)}_1 \wedge \dotsb \wedge w^{(j)}_s
  \\
   &\,  + \sum_{i=n+1}^m \sum_{j=1}^l a_i b_j v^{(i)}_1 \wedge \dotsb \wedge v^{(i)}_r \wedge w^{(j)}_1 \wedge \dotsb \wedge w^{(j)}_s
  \\
  =&\,  \left(
          \sum_{i=1}^n a_i v^{(i)}_1 \wedge \dotsb \wedge v^{(i)}_r
        \right)
        \wedge
        \left(
          \sum_{j=1}^l b_j w^{(j)}_1 \wedge \dotsb \wedge w^{(j)}_s
        \right)
  \\
   &\,  +
        \left(
          \sum_{i=n+1}^m a_i v^{(i)}_1 \wedge \dotsb \wedge v^{(i)}_r
        \right)
        \wedge
        \left(
          \sum_{j=1}^l b_j w^{(j)}_1 \wedge \dotsb \wedge w^{(j)}_s
        \right)
  \\
  =&\,  (x \wedge z) + (y \wedge z),
\end{align*}
und zum anderen, dass
\begin{align*}
    (\lambda x) \wedge z
  &=  \left(
        \lambda
        \sum_{i=1}^m a_i v^{(i)}_1 \wedge \dotsb \wedge v^{(i)}_r
      \right)
      \wedge
      \left(
        \sum_{j=1}^l b_j w^{(j)}_1 \wedge \dotsb \wedge w^{(j)}_s
      \right)
  \\
  &=  \left(
        \sum_{i=1}^m \lambda a_i v^{(i)}_1 \wedge \dotsb \wedge v^{(i)}_r
      \right)
      \wedge
      \left(
        \sum_{j=1}^l b_j w^{(j)}_1 \wedge \dotsb \wedge w^{(j)}_s
      \right)
  \\
  &=  \sum_{i=1}^m \sum_{j=1}^l \lambda a_i b_j
                               v^{(i)}_1 \wedge \dotsb \wedge v^{(i)}_r \wedge w^{(j)}_1 \wedge \dotsb \wedge w^{(j)}_s
  \\
  &=  \lambda \sum_{i=1}^m \sum_{j=1}^l a_i b_j
                               v^{(i)}_1 \wedge \dotsb \wedge v^{(i)}_r \wedge w^{(j)}_1 \wedge \dotsb \wedge w^{(j)}_s
  \\  
  &=  \lambda \left(
                \left(
                  \sum_{i=1}^m a_i v^{(i)}_1 \wedge \dotsb \wedge v^{(i)}_r
                \right)
                \wedge
                \left(
                  \sum_{j=1}^l b_j w^{(j)}_1 \wedge \dotsb \wedge w^{(j)}_s
                \right)
              \right)
  = \lambda (x \wedge y).
\end{align*}
Zusammen zeigt das die Bilinearität von $\wedge$ im ersten Argument.

\begin{remark}
  Aus der von uns gewählten Konstruktion von $\wedge$ geht tatsächlich schon ohne weitere Rechnungen hervor, dass $\wedge$ im ersten Argument linear ist:
  Dies bedeutet nämlich gerade, dass für fixiertes $y \in \exteriorpower{s}{V}$ die Abbildung
  \[
            (-) \wedge y
    \colon  \exteriorpower{r}{V}
    \to     \exteriorpower{r+s}{V},
    \quad   x
    \mapsto x \wedge y
  \]
  linear ist.
  Bei dieser Abbildung handelt es sich aber genau um die lineare Abbildung $\rho_y$.
  
  Die Linearität im zweiten Argument erhalten wir leider nicht direkt aus unserer Konstruktion.
  Neben dem direkten Nachrechnen gibt es aber auch hier einen geschickten Ausweg:
  
  Wir haben in unserer Konstruktion im ersten Schritt das erste Argument von $m_1$ fixiert, und im zweiten Schritt dafür das zweite Argument von $m_2$.
  Stattdessen hätte man auch umgekehrt vorgehen können:
  
  Indem man in einem abgeänderten ersten Schritt das zweite Argument von $m_1$ fixiert, erhält man mithilfe der universelle Eigenschaft der äußeren Potenz eine Abbildung
  \begin{align*}
            \tilde{m}_2
     \colon \exteriorpower{r}{V} \times V^s
    &\to    \exteriorpower{r+s} V
    \\
              \left( \sum_i a_i v_1^{(i)} \wedge \dotsb \wedge v_r^{(i)}, (w_1, \dotsc, w_s) \right)
    &\mapsto  \sum_i a_i v^{(i)}_1 \wedge \dotsb \wedge v^{(i)}_r \wedge w_1 \wedge \dotsb \wedge w_s.
  \end{align*}
  Durch anschließendes Fixieren des ersten Arguments von $\tilde{m}_2$ erhält man mithilfe der universelle Eigeschaft der äußeren Potenz dann die gleiche Funktion $m_3$, wie in der von uns gewählten Konstruktion.
  
  Während wir bei der von uns gewählten Reihenfolge die Linearität im ersten Argument geschenkt bekommen, bekommt man bei dieser anderen Reihenfolge die Linearität im zweiten Argument geschenkt.
  Aus beiden möglichen Reihenfolgen erhält man deshalb insgesamt, dass $\wedge$ in beiden Argumenten linear ist.
\end{remark}

\begin{remark}
  \label{remark: wedge product with scalars}
  Für $r = 0$ ist $\exteriorpower{r}{V} = \exteriorpower{0}{V} = K$.
  Das Element $1 \in K$ entspricht dabei dem „leeren Dachprodukt“ $1 = v_1 \wedge v_2 \wedge \dotsb \wedge v_0$ mit $v_1, v_2, \dotsc, v_0 \in V$.
  Für alle $w_1, \dotsc, w_s \in V$ gilt deshalb, dass
  \[
      1 \wedge (w_1 \wedge \dotsb \wedge w_s)
    = \underbrace{(v_1 \wedge v_2 \wedge \dotsb \wedge v_0)}_{\text{$0$ Faktoren}} \wedge (w_1 \wedge \dotsb \wedge w_s)
    = w_1 \wedge \dotsb \wedge w_s.
  \]
  Da die Abbildung $1 \wedge (-) \colon \exteriorpower{s}{V} \to \exteriorpower{s}{V}$ linear ist, und die Elemente $w_1 \wedge \dotsb \wedge w_s$  ein Erzeugendensystem von $\exteriorpower{s}{V}$ bilden, ergibt sich daraus, dass allgemeiner $1 \wedge y = y$ für alle $y \in \exteriorpower{s}{V}$ gilt.
  Analog ergibt sich im Fall $s = 0$, dass $x \wedge 1 = x$ für alle $x \in \exteriorpower{r}{V}$ gilt
  
  Wegen der Bilinearität von $\wedge$ ergibt sich im Fall $r = 0$ allgemeiner, dass $\lambda \wedge y = \lambda(1 \wedge y) = \lambda y$ für alle $\lambda \in K$ und $y \in \exteriorpower{s}{V}$ gilt.
  Analog ergibt sich im Fall $s = 0$, dass $x \wedge \lambda = \lambda x$ für alle $\lambda \in K$ und $x \in \exteriorpower{r}{V}$ gilt.
  
  Insbesondere ergibt sich im Fall $r = s = 0$, dass $\lambda \wedge \mu = \lambda \mu$ für alle $\lambda, \mu \in K$ gilt.
\end{remark}

\begin{remark*}
  Beim Vorrechnen der Konstruktion im Tutorium habe ich jeweils die Vorfaktoren $a_i$ und $b_j$ weggelassen.
  Für $r, s > 0$ ist dies okay, denn für alle Skalare $a, b \in K$ und Vektoren $v_1, \dotsc, v_r, w_1, \dotsc, w_s \in V$ gilt
  \[
      \lambda (v_1 \wedge \dotsb \wedge v_r)
    = (\lambda v_1) \wedge \dotsb \wedge v_r
    \quad\text{und}\quad
      \lambda (w_1 \wedge \dotsb \wedge w_s)
    = (\lambda w_1) \wedge \dotsb \wedge w_s.
  \]
  Deshalb lassen sich in den von uns getätigten Rechnungen die Vorfaktoren jeweils in die Elemente reinziehen.
\end{remark*}





\subsection{}

Wir wissen bereits, dass $\bigwedge V = \bigoplus_{d=0}^n \exteriorpower{d}{V}$ ein $K$-Vektorraum bezüglich der komponentenweise Addition
\[
    (x_0, \dotsc, x_n) + (y_0, \dotsc, y_n) 
  = (x_0 + y_0, \dotsc, x_n + y_n)
  \quad
  \text{für alle $(x_0, \dotsc, x_n), (y_0, \dotsc, y_n) \in \exterioralgebra{V}$}
\]
und der komponentenweise Skalarmultiplikation
\[
    \lambda \cdot (x_0, \dotsc, x_n)
  = (\lambda x_0, \dotsc, \lambda x_n)
  \qquad
  \text{für alle $\lambda \in K$, $(x_0, \dotsc, x_n) \in \exterioralgebra{V}$}
\]
einen $K$-Vektorraum bildet.
Für alle $(x_0, \dotsc, x_n), (y_0, \dotsc, y_n) \in \exterioralgebra{V}$ definieren wir nun
\[
            (x_0, \dotsc, x_n) \wedge (y_0, \dotsc, y_n)
  \coloneqq (z_0, \dotsc, z_n)
  \quad\text{mit}\quad
            z_r = \sum_{s=0}^r (x_s \wedge y_{r-s})
                = \sum_{s + t = r} (x_s \wedge y_t).
\]
Diese Verknüpfung ist wohldefiniert, denn für $x_s \in \exteriorpower{s}{V}$ und $y_t \in \exteriorpower{t}{V}$ gilt $x_s \wedge y_t \in \exteriorpower{s+t}{V}$.

\begin{remark}
  Stellt man sich die Elemente $(x_0, \dotsc, x_n)$ und $(y_0, \dotsc, y_n)$ als formale Summen $\sum_{s=0}^n x_s$ und $\sum_{t=0}^n y_t$ vor, so ist die Verknüpfung $\wedge$ gerade als
  \[
      \left( \sum_{s=0}^n x_s \right) \wedge \left( \sum_{t=0}^n y_t \right)
    = \sum_{s,t = 0}^n (x_s \wedge y_t)
  \]
  definiert.
  Gilt dabei $s + t > n$, so ist $x_s \wedge y_t \in \exteriorpower{s+t}{V} = 0$.
  Deshalb treten tatsächlich nur die Summanden auf, für die $s + t \leq n$ gilt, weshalb die Summe $\sum_{s,t = 0}^n (x_s \wedge y_t)$ in $\bigoplus_{d=0}^n \exteriorpower{d}{V} = \exterioralgebra{V}$ lebt.
\end{remark}

Wir zeigen, dass der Vektorraum $\exterioralgebra{V}$ zusammen mit $\wedge$ zu einer $K$-Algebra wird:

\begin{itemize}
  \item
    Wir spalten den Nachweis der Assoziativität in drei Schritte auf:
    \begin{enumerate}
      \item
        Für alle $r, s, t \in \naturals$ und  $u_1, \dotsc, u_r, v_1, \dotsc, v_r, w_1, \dotsc, w_t \in V$ gilt
        \begin{align*}
           &\,  ( (u_1 \wedge \dotsb \wedge u_r) \wedge (v_1 \wedge \dotsb \wedge v_s) ) \wedge (w_1 \wedge \dotsb \wedge w_t)
          \\
          =&\,  ( u_1 \wedge \dotsb \wedge u_r \wedge v_1 \wedge \dotsb \wedge v_s ) \wedge (w_1 \wedge \dotsb \wedge w_t)
          \\
          =&\,  u_1 \wedge \dotsb \wedge u_r \wedge v_1 \wedge \dotsb \wedge v_s \wedge w_1 \wedge \dotsb \wedge w_t
          \\
          =&\,  \dotsb
          =     (u_1 \wedge \dotsb \wedge u_r) \wedge ((v_1 \wedge \dotsb \wedge v_s) \wedge (w_1 \wedge \dotsb \wedge w_t)).
        \end{align*}
      \item
        Für alle $r, s, t \in \naturals$ und $x \in \exteriorpower{r}{V}$, $y \in \exteriorpower{s}{V}$ und $z \in \exteriorpower{t}{V}$ folgt, dass $x \wedge (y \wedge z) = (x \wedge y) \wedge z$ gilt:
        Es seien $x = \sum_i a_i u^{(i)}$, $y = \sum_j b_j v^{(j)}$ und $z = \sum_k c_k w^{(k)}$, wobei wir die abkürzenden Notationen $u^{(i)} = u^{(i)}_1 \wedge \dotsb \wedge u^{(i)}_r$, $v^{(j)} = v^{(j)}_1 \wedge \dotsb \wedge v^{(j)}_s$ und $w^{(k)} = w^{(k)}_1 \wedge \dotsb \wedge w^{(k)}_t$ verwenden.
        Wir haben bereits gezeigt, dass $u^{(i)} \wedge (v^{(j)} \wedge w^{(k)}) = (u^{(i)} \wedge v^{(j)}) \wedge w^{(k)}$ für alle $i,j,k$ gilt;
        es muss also nicht zwischen diesen beiden Klammerungen unterschieden werden.
        Daraus folgt dann mit der Bilinearität von $\wedge$, dass
        \begin{align*}
              x \wedge (y \wedge z)
          &=          \left( \sum_i a_i u^{(i)} \right)
              \wedge  \left( \left( \sum_j b_j v^{(j)} \right) \wedge \left( \sum_k c_k w^{(k)} \right) \right)
          \\
          &=  \left( \sum_i a_i u^{(i)} \right) \wedge \left( \sum_{j,k} b_j c_k v^{(j)} w^{(k)} \right)
          =   \sum_{i,j,k} a_i b_j c_k u^{(i)} \wedge v^{(j)} \wedge w^{(k)}
          \\
          &=  \dotso
           =  (x \wedge y) \wedge z.
        \end{align*}
        \begin{remark}
          Man kann auch argumentieren, dass die beiden Abbildungen
          \[
                    \alpha_1, \alpha_2
            \colon  \exteriorpower{r}{V} \times \exteriorpower{s}{V} \times \exteriorpower{t}{V}
            \to     \exteriorpower{r+s+t}{V}
          \]
          mit $\alpha_1 \colon (x,y,z) \mapsto x \wedge (y \wedge z)$ und $\alpha_2 \colon (x,y,z) \mapsto (x \wedge y) \wedge z$ wegen der Bilinearität von $\wedge$ beide trilinear sind, und dann die Gleihheit der beiden Abbildungen mithilfe von Lemma~\ref{lemma: multilinear maps are uniquely determined by generating sets} aus dem ersten Schritt folgern.
          (Die obige Rechnung ist nur ein Spezialfall des Beweises von Lemma~\ref{lemma: multilinear maps are uniquely determined by generating sets}.)
        \end{remark}
      \item
        Für beliebige Elemente $x, y, z \in \exterioralgebra{V}$ mit Einträgen $x = (x_0, \dotsc, x_n)$, $y = (y_0, \dotsc, y_n)$ und $z = (z_0, \dotsc, z_n)$ ergibt sich nun mit der Bilinearität von $\wedge$, dass
        \begin{align*}
              (x \wedge (y \wedge z))_p
          &=  \sum_{r + q = p} x_r \wedge (y \wedge z)_q
           =  \sum_{r + q = p} x_r \wedge \left( \sum_{s + t = q} y_s \wedge z_t \right)
           =  \sum_{\substack{r + q = p \\ s + t = q}} x_r \wedge y_s \wedge z_t
          \\
          &=  \sum_{r + s +t = p} x_r \wedge y_s \wedge z_t
           =  \dotsb
           =  ((x \wedge y) \wedge z)_p.
          \end{align*}
        für alle $p = 0, \dotsc n$ gilt, und somit $x \wedge (y \wedge z) = (x \wedge y) \wedge z$.
    \end{enumerate}
    Das zeigt die Assoziativität.
  \item
    Für alle $\lambda \in K$ und $x = (x_0, \dotsc, x_n) \in \exterioralgebra{V}$ gilt
    \begin{equation}
      \label{equation: scalarmultiplication via zero wedges}
        (\lambda, 0, \dotsc, 0) \wedge x
      = (\lambda, 0, \dotsc, 0) \wedge (x_0, \dotsc, x_n)
      = (\lambda \wedge x_0, \dotsc, \lambda \wedge x_n)
      = (\lambda x_0, \dotsc, \lambda x_n)
      = \lambda x
    \end{equation}
    sowie analog auch $x \wedge (\lambda, 0, \dotsc, 0) = \lambda x$.
    (Hier nutzen wir Bemerkung~\ref{remark: wedge product with scalars}.)
    Insbesondere ist $(1, 0, \dotsc, 0)$ deshalb ein Einselement bezüglich der Multiplikation.
  \item
    Für alle $x, y, z \in \exterioralgebra{V}$ mit $x = (x_0, \dotsc, x_n)$, $y = (y_0, \dotsc, y_n)$ und $z = (z_0, \dotsc, z_n)$ ergibt sich aus der Bilinearität von $\wedge$, dass
    \begin{align*}
          ((x + y) \wedge z)_p
      &=  \sum_{r + s = p} (x + y)_r \wedge z_s
       =  \sum_{r + s = p} (x_r + y_r) \wedge z_s
       =  \sum_{r + s = p} \left( (x_r \wedge z_s) + (y_r \wedge z_s) \right)
      \\
      &=  \left( \sum_{r + s = p} x_r \wedge z_s \right) + \left( \sum_{r + s = p} y_r \wedge z_s \right)
       =  (x \wedge z)_p + (y \wedge z)_p
       =  ((x \wedge z) + (y \wedge z))_p
    \end{align*}
    für alle $p = 0, \dotsc, n$ gilt.
    Somit gilt bereits $(x + y) \wedge z = (x \wedge z) + (y \wedge z)$, was die Distributivität im ersten Argument zeigt.
    Die Distributivität im zweiten Argument ergibt sich analog.
\end{itemize}

Das zeigt, dass $\exterioralgebra{V}$ bezüglich $\wedge$ ein Ring ist, und es bleibt zu zeigen, dass $\exterioralgebra{V}$ durch $\varphi \colon K \to \exterioralgebra{V}$, $\lambda \mapsto (\lambda, 0, \dotsc, 0)$ zu einer $K$-Algebra wird.

\begin{itemize}[resume]
  \item
    Die Abbildung ist ein Ringhomomorphismus, denn für alle $\lambda, \mu \in K$ gilt
    \begin{gather*}
        \varphi(\lambda) + \varphi(\mu)
      = (\lambda, 0, \dotsc, 0) + (\mu, 0, \dotsc, 0)
      = (\lambda + \mu, 0, \dotsc, 0)
      = \varphi(\lambda + \mu) 
    \shortintertext{und}
        \varphi(\lambda) \wedge \varphi(\mu)
      = (\lambda, 0, \dotsc, 0) \wedge (\mu, 0, \dotsc, 0)
      = (\lambda \mu, 0, \dotsc, 0)
      = \varphi(\lambda \mu),
    \end{gather*}
    (hier haben wir \eqref{equation: scalarmultiplication via zero wedges} genutzt), und es gilt $\varphi(1_K) = (1, 0, \dotsc, 0) = 1_{\exterioralgebra{V}}$.
  \item
    Aus \eqref{equation: scalarmultiplication via zero wedges} erhalten wir, dass
    \[
        \varphi(\lambda) \wedge x
      = \lambda x
      = x \wedge \varphi(\lambda)
    \]
    für alle $\lambda \in K$ und $x \in \exterioralgebra{V}$ gilt.
\end{itemize}
Das zeigt, dass $\exterioralgebra{V}$ durch $\varphi$ zu einer $K$-Algebra wird.
