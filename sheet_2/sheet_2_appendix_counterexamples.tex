\section{Gegenbeispiele}

Es sei $d \in \naturals$ und $V$ und $W$ seien $K$-Vektorräume.





\subsection{Zerlegbarkeit von Elementen}

Nicht jedes Element $x \in \exteriorpower{d}{V}$ ist notwendigerweise von der Form $x = v_1 \wedge \dotsb \wedge v_d$ für passende $v_1, \dotsc, v_d \in V$.
Man betrachte etwa das Element $\alpha \coloneqq e_1 \wedge e_2 + e_3 \wedge e_4 \in \exteriorpower{2}{\real^4}$.
Für $\alpha \wedge \alpha \in \exteriorpower{4}{\real^4}$ gilt
\begin{align*}
        \alpha \wedge \alpha
  &=    (e_1 \wedge e_2 + e_3 \wedge e_4) \wedge (e_1 \wedge e_2 + e_3 \wedge e_4)
  \\
  &=      \underbrace{e_1 \wedge e_2 \wedge e_1 \wedge e_2}_{=0}
        + e_1 \wedge e_2 \wedge e_3 \wedge e_4
        + \underbrace{e_3 \wedge e_4 \wedge e_1 \wedge e_2}_{= e_1 \wedge e_2 \wedge e_3 \wedge e_4}
        + \underbrace{e_3 \wedge e_4 \wedge e_3 \wedge e_4}_{=0}
  \\
  &=    2 e_1 \wedge e_2 \wedge e_3 \wedge e_4
   \neq 0,
\end{align*}
da $\{e_1 \wedge e_2 \wedge e_3 \wedge e_4\}$ eine Basis von $\exteriorpower{4}{\real^4}$ ist, und somit insbesondere $e_1 \wedge e_2 \wedge e_3 \wedge e_4 \neq 0$ gilt.
Wäre aber $\alpha = v_1 \wedge v_2$ für passende $v_1, v_2 \in \real^2$, so würde
\[
    \alpha \wedge \alpha
  = v_1 \wedge v_2 \wedge v_1 \wedge v_2
  = 0
\]
gelten.





\subsection{Kerne auf einfachen Elementen nachrechnen}

Ist $f \colon \exteriorpower{d}{V} \to W$ eine lineare Abbildung, so dass
\begin{equation}
  \label{equation: kernel vanishes on simple wedges}
            f(v_1 \wedge \dotsb \wedge v_d) = 0
  \implies  v_1 \wedge \dotsb \wedge v_d = 0
  \qquad
  \text{für alle $v_1, \dotsc, v_d \in V$}
\end{equation}
gilt, so muss $f$ nicht notwendigerweise injektiv sein.
Man betrachte etwa für das Element $\alpha \coloneqq e_1 \wedge e_2 + e_3 \wedge e_4 \in \exteriorpower{2}{\real^4}$ die kanonische Projektion
\[
          f
  \colon  \exteriorpower{2}{\real^4}
  \to     \left( \exteriorpower{2}{\real^4} \right) / \generated{\alpha},
  \quad   x
  \mapsto \class{x}.
\]
Es seien $v_1, v_2 \in \real^4$ mit $v_1 \wedge v_2 \in \ker f = \generated{\alpha}$.
Im Fall $v_1 \wedge v_2 \neq 0$ wäre $\{ v_1 \wedge v_2 \}$ bereits eine Basis des eindimensionalen Untervektorraums $\generated{\alpha} \subseteq \exteriorpower{2}{\real^4}$.
Dann gebe es $\lambda \in K$ mit $\alpha = \lambda v_1 \wedge v_2 = (\lambda v_1) \wedge v_2$.
Wir wissen aber bereits, dass sich $\alpha$ nicht so darstellen lässt.
Also muss $v_1 \wedge v_2 = 0$ gelten, und somit $f$ die Bedingung \eqref{equation: kernel vanishes on simple wedges} erfüllen.
Da aber $0 \neq \alpha \in \ker f$ gilt, ist $f$ nicht injektiv.





\subsection{Injektivität auf einfachen Elementen nachrechnen}

Ist $f \colon \exteriorpower{d}{V} \to W$ eine lineare Abbildung, so dass
\begin{equation}
  \label{equation: injective on simple wedges}
            f(v_1 \wedge \dotsb \wedge v_d) = f(w_1 \wedge w_d)
  \implies  v_1 \wedge \dotsb \wedge v_d = w_1 \wedge w_d
\end{equation}
für alle $v_1, \dotsc, v_d, w_1, \dotsc, w_d \in V$ gilt, so muss $f$ nicht notwendigerweise injektiv sein.

Unser vorheriges Gegenbeispiel lässt sich nicht direkt anwenden, da in diesem die beiden Elemente $e_1 \wedge e_2$ und $- e_3 \wedge e_4 = e_4 \wedge e_3$ miteinander identifiziert werden.
Wir werden aber ein Gegenbeispiel nach dem gleichen Prinzip kontstruieren, d.h.\ wir finden einen passenden Vektorraum $V$ und ein Element $\alpha \in \exteriorpower{2}{V}$ mit $\alpha \neq 0$, so dass die kanonische Projektion
\[
          f
  \colon  \exteriorpower{2}{V}
  \to     \left( \exteriorpower{2}{V} \right) / \generated{\alpha},
  \quad   x
  \mapsto \class{x}
\]
die Bedingung \eqref{equation: injective on simple wedges} erfüllt.

Hierfür bemerke man, dass für alle $v_1, v_2, w_1, w_2 \in V$ die Bedingung $f(v_1 \wedge v_2) = f(w_1 \wedge w_2)$ äquivalent dazu ist, dass $v_1 \wedge v_2 - w_1 \wedge w_2 \in \ker f = \generated{\alpha}$ gilt.
Wäre dies für $v_1 \wedge v_2 \neq w_1 \wedge w_2$ eintreten, so wäre $\{ v_1 \wedge v_2 - w_1 \wedge w_2 \}$ eine Basis des eindimensionalen Untervektorraums $\generated{\alpha} \subseteq \exteriorpower{2}{V}$.
Dann gebe es $\lambda \in K$ mit $\alpha = \lambda(v_1 \wedge v_2 - w_1 \wedge w_2) = (\lambda v_1) \wedge v_2 + (-\lambda w_1) \wedge w_2$.

Es genügt deshalb, ein Element $\alpha \in \exteriorpower{2}{V}$ für einen passenden Vektorraum $V$ zu finden, so dass $\alpha \neq v_1 \wedge v_2 + w_1 \wedge w_2$ für alle $v_1, v_2, w_1, w_2 \in V$ gilt;
wenn man also $\alpha$ als $\alpha = \sum_i v^{(i)}_1 \wedge v^{(i)}_2$ darstellt, werden mindestens $3$ Summanden benötigt.

Um ein solches Element zu finden, bemerke man, dass für jeden Vektorraum $V$ und alle $v_1, v_2, w_1, w_2 \in V$ für das Element $\beta = v_1 \wedge v_2 + w_1 \wedge w_2 \in \exteriorpower{2}{V}$ gilt, dass
\begin{gather*}
    \beta \wedge \beta
  = (v_1 \wedge v_2 + w_1 \wedge w_2) \wedge (v_1 \wedge v_2 + w_1 \wedge w_2)
  = \dotsb
  = 2 v_1 \wedge v_2 \wedge w_1 \wedge w_2,
\shortintertext{und somit}
    (\beta \wedge \beta) \wedge (\beta \wedge \beta)
  = 4 v_1 \wedge v_2 \wedge w_1 \wedge w_2 \wedge v_1 \wedge v_2 \wedge w_1 \wedge w_2
  = 0.
\end{gather*}
Es genügt also, für einen passenden Vektorraum $V$ ein Element $\alpha \in \exteriorpower{2}{V}$ zu finden, so dass $(\alpha \wedge \alpha) \wedge (\alpha \wedge \alpha) \neq 0$ gilt.
Ein solches Element ist durch $\alpha \coloneqq e_1 \wedge e_2 + e_3 \wedge e_4 + e_5 \wedge e_6 + e_7 \wedge e_8 \in \exteriorpower{2}{\real^8}$ gegeben:
Es gilt
\begin{gather*}
  \begin{aligned}
      \alpha \wedge \alpha
    = \dotsb
    = 2(&   e_1 \wedge e_2 \wedge e_3 \wedge e_4
        +  e_1 \wedge e_2 \wedge e_5 \wedge e_6
        +  e_1 \wedge e_2 \wedge e_7 \wedge e_8
    \\
        &+ e_3 \wedge e_4 \wedge e_5 \wedge e_6
          + e_3 \wedge e_4 \wedge e_7 \wedge e_8
          + e_5 \wedge e_6 \wedge e_7 \wedge e_8 ),
  \end{aligned}
  \shortintertext{und somit}
          (\alpha \wedge \alpha) \wedge (\alpha \wedge \alpha)
    =     \dotsb
    =     48 e_1 \wedge e_2 \wedge e_3 \wedge e_4 \wedge e_5 \wedge e_6 \wedge e_7 \wedge e_8
    \neq  0.
\end{gather*}





% \subsection{Gleichheit von einfachen Elementen}
% 
% Sind $v_1, \dotsc, v_d, w_1, \dotsc w_d \in V$ mit $v_1 \wedge \dotsb \wedge v_d = w_1 \wedge \dotsb \wedge w_d$, so folgt noch nicht notwendigerweise, dass die bereits die beiden Tupel $(v_1, \dotsc, v_d)$ und $(w_1, \dotsc, w_d)$ bis auf Permutation gleich sind;
% d.h.\ es muss nicht notwendigerweise eine Permutation $\sigma \in S_d$ geben, so dass $w_i = v_{\sigma(i)}$ für alle $i = 1, \dotsc, d$ gilt.
% So gilt etwa für $e_1, e_2 \in \real^2$, dass $e_1 \wedge e_1 = 0 = e_2 \wedge e_2$.

% Die Frage, wann $v_1 \wedge \dotsb \wedge v_d = w_1 \wedge \dotsb \wedge w_d$ gilt, lässt sich wie folgt beantworten:
% 
% \begin{proposition}
%   \label{proposition: equality of simple wedges}
%   Es sei $V$ ein Vektorraum.
%   Für alle $v_1, \dotsc, v_d \in V$ und $w_1, \dotsc, w_d \in V$ gilt genau dann $v_1 \wedge \dotsb \wedge v_d = w_1 \wedge \dotsb \wedge w_d$, wenn mindestens eine der folgenden beiden Bedingungen erfüllt ist:
%   \begin{enumerate}
%     \item
%       Die Familien $(v_1, \dotsc, v_d)$ und $(w_1, \dotsc, w_d)$ sind linear abhängig.
%     \item
%       Es gilt $\generated{v_1, \dotsc, v_d} = \generated{w_1, \dotsc, w_d}$.
%   \end{enumerate}
% \end{proposition}
% 
% Zum Beweis von Proposition~\ref{proposition: equality of simple wedges} nutzen wir die folgende Beobachtung:
% 
% \begin{lemma}
%   Es sei $V$ ein Vektorraum.
%   Für alle $v_1, \dotsc, v_d \in V$ gilt genau dann $v_1 \wedge \dotsb \wedge v_d = 0$, wenn die Familie $(v_1, \dotsc, v_d)$ linear abhängig ist.
% \end{lemma}
% \begin{proof}
%   Sind $(v_1, \dotsc, v_d)$ linear abhängig, so ist der Vektorraum $\dim \generated{v_1, \dotsc, v_d} < d$.
%   Für passendes $d' \coloneqq d-1$ gibt es deshalb $w_1, \dotsc, w_{d'} \in \generated{v_1, \dotsc, v_d}$ mit $\generated{v_1, \dotsc, v_d} = \generated{w_1, \dotsc, w_{d'}}$.
%   Insbesondere gibt es für alle $i = 1, \dotsc, d$ eine Darstellung $v_i = \sum_{j=1}^{d'} \lambda^{(i)}_j w_j$.
%   Dann gilt
%   \[
%       v_1 \wedge \dotsb \wedge v_d
%     =         \left( \sum_{j_1 = 1}^{d'} \lambda^{(1)}_{j_1} w_{j_1} \right)
%       \wedge  \dotsb
%       \wedge  \left( \sum_{j_d = 1}^{d'} \lambda^{(d)}_{j_d} w_{j_d} \right)
%     = \sum_{j_1, \dotsc, j_{d-1} = 1}^{d'}
%         \lambda^{(1)}_{j_1} \dotsm \lambda^{(d-1)}_{j_{d-1}}                                    
%         w_{j_1} \wedge \dotsb \wedge w_{j_d}
%   \]
%   Dabei sind in dem Ausdruck $w_{j_1} \wedge \dotsb \wedge w_{j_d}$ immer mindestens zwei der Faktoren gleich, weshalb $w_{j_1} \wedge \dotsb \wedge w_{j_d} = 0$ für alle $j_1, \dotsc, j_d = 1, \dotsc, d$ gilt.
%   Somit gilt auch insgesamt $v_1 \wedge \dotsb \wedge v_d = 0$.
%   
%   Sind $v_1, \dotsc, v_d$ linear unabhängig, so lassen sie sich zu einer Basis $(v_i)_{i \in I}$ von $V$ ergänzen, wobei $\{1, \dotsc, n\} \subseteq I$.
%   Ist $\leq$ eine lineare Ordnung auf $I$ mit $1 \leq 2 \leq \dotsb \leq n$, so ergibt sich für $\exteriorpower{d}{V}$ die Basis
%   \[
%               \basis{B}
%     \coloneqq \left(
%                 v_{i_1} \dotsb v_{i_d}
%               \suchthatscale
%                 i_1, \dotsc, i_d \in I,
%                 i_1 < \dotsb < i_d
%               \right).
%   \]
%   Dann ist $v_1 \wedge \dotsb \wedge v_d$ ein Element von $\basis{B}$, und somit insbesondere $v_1 \wedge \dotsb \wedge v_d \neq 0$.
% \end{proof}


  




