\appendix


\section{Gegenbeispiele}

Nicht jedes Element $x \in \exteriorpower{d}{V}$ ist notwendigerweise von der Form $x = v_1 \wedge \dotsb \wedge v_d$.
Man betrachte etwa $\alpha \coloneqq e_1 \wedge e_2 + e_3 \wedge e_4 \in \exteriorpower{2}{\real}$.
Für $\alpha \wedge \alpha \in \exteriorpower{4}{\real}$ gilt dann
\[
    \alpha \wedge \alpha
  = (e_1 \wedge e_2 + e_3 \wedge e_4) \wedge (e_1 \wedge e_2 + e_3 \wedge e_4)
  = e_1 \wedge e_2 \wedge e_1 \wedge e_2 + e_3 \wedge e_4 \wedge e_1 \wedge e_2
\]