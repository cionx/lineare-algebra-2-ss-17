\section{}

Nach Annahme kommutiert das folgende Diagramm:
\begin{equation}
  \label{diagram: given one}
  \begin{tikzcd}
      V
      \arrow{r}{f}
      \arrow[swap]{d}{h}
    & V
      \arrow{d}{h}
    \\
      W
      \arrow{r}{g}
    & W
  \end{tikzcd}
\end{equation}

Wir zeigen die zu zeigenden Aussagen auf zwei Weisen:
Zum einen zeigen wir die Aussage mithilfe von Aufgabe~3, und zum anderen von Hand und durch leichtes Abändern des Diagramms.





\subsection*{Mithilfe von Aufgabe~3}

Wir betrachten zunächst den Fall, dass $h$ bijektiv, also ein Isomorphismus ist.
Dann sind $V$ und $W$ in gewisser Weise „gleich“, und die Kommutatvität des Diagramms~\eqref{diagram: given one} sorgt dafür, dass $f$ und $g$ in gewisser Weise „die gleiche“ Abbildung sind.


\subsubsection*{$h$ ist bijektiv}

Ist $h$ bijektiv, also ein Isomorphismus, so gilt $p_f(t) = p_g(t)$:

Es sei $\basis{B} = (v_1, \dotsc, v_n)$ eine Basis von $V$.
Dann ist $\basis{C} = (w_1, \dotsc, w_n)$ mit $w_i \coloneqq h(v_i)$ für alle $i = 1, \dotsc, n$ eine Basis von $W$, da $h$ ein Isomorphismus ist.
Es sei $A \coloneqq \repmatrix{f}{\basis{B}}{\basis{B}} \in \matrices{n}{K}$ die darstellende Matrix des Endomorphismus $f$ bezüglich der Basis $\basis{B}$.

\begin{claim}
  Es gilt auch $A = \repmatrix{g}{\basis{C}}{\basis{C}}$.
\end{claim}

\begin{proof}
  Wir geben zwei mögliche Beweise an.
  \begin{enumerate}
    \item
      Es sei $B \coloneqq \repmatrix{g}{\basis{C}}{\basis{C}} \in \matrices{n}{K}$.
      Die Einträge von $A$ und $B$ sind eindeutig dadurch festgelegt, dass
      \[
          f(v_j)
        = \sum_{i=1}^n A_{ij} v_i
        \quad\text{und}\quad
          g(w_j)
        = \sum_{i=1}^n B_{ij} w_j
        \qquad
        \text{für alle $j = 1, \dotsc, n$}
      \]
      gilt. Wegen der Kommutativität des Diagrams \eqref{diagram: given one} erhalten wir dabei aus den Gleichungen $f(v_j) = \sum_{i=1}^n A_{ij} v_i$ für $j = 1, \dotsc, n$, dass
      \[
          g( w_j )
        = g( h(v_j) )
        = h( f(v_j) )
        = h\left( \sum_{i=1}^n A_{ij} v_i \right)
        = \sum_{i=1}^n A_{ij} h(v_i)
        = \sum_{i=1}^n A_{ij} w_j
      \]
      für alle $j = 1, \dotsc, n$ gilt.
      Also gilt bereits $A_{ij} = B_{ij}$ für alle $i,j = 1, \dotsc, n$ und somit $A = B$.
    \item
      Wegen der Kommutativität von \eqref{diagram: given one} gilt $h \circ f = g \circ h$, woraus sich durch Komposition mit $h^{-1}$ von rechts ergibt, dass auch $h \circ f \circ h^{-1}$ gilt;
      es kommutiert also auch das folgende Diagramm:
      \begin{equation}
        \begin{tikzcd}
            V
            \arrow{r}{f}
          & V
            \arrow{d}{h}
          \\
            W
            \arrow{u}{h^{-1}}
            \arrow{r}{g}
          & W
        \end{tikzcd}
      \end{equation}
      Dass $h(v_i) = w_i$ für alle $i = 1, \dotsc, n$ gilt, ist äquivalent dazu, dass $\repmatrix{h}{\basis{B}}{\basis{C}} = I$ gilt, wobei $I \in \matrices{n}{K}$ die Einheitsmatrix bezeichnet.
      Daraus ergibt außerdem, dass auch $\repmatrix{h^{-1}}{\basis{C}}{\basis{B}} = (\repmatrix{h}{\basis{B}}{\basis{C}})^{-1} = I^{-1} = I$ gilt.
      Ingesamt erhalten wir somit, dass
      \[
          \repmatrix{g}{\basis{C}}{\basis{C}}
        = \repmatrix{h \circ f \circ h^{-1}}{\basis{C}}{\basis{C}}
        =       \repmatrix{h}{\basis{B}}{\basis{C}}
          \cdot \repmatrix{f}{\basis{B}}{\basis{B}}
          \cdot \repmatrix{h^{-1}}{\basis{C}}{\basis{B}}
        = I \cdot A \cdot I
        = A
      \]
      gilt.
    \qedhere
  \end{enumerate}
\end{proof}

Aus der Behauptung folgt nun insbesondere, dass $p_f(t) = p_A(t) = p_g(t)$ gilt.



\subsubsection{\texorpdfstring{$h$}{h} ist injektiv}

Die lineare Abbildung $h$ sei nun injektiv.
Dann schränkt sich $h$ zu einem Isomorphismus $h' \colon V \to \im h$, $v \mapsto h(v)$ ein.

Wegen der Kommutativität des Diagramms \eqref{diagram: given one} ist der Untervektorraum $\im h \subseteq W$ bereits $g$-invariant, denn es gilt
\[
            g(\im h)
  =         g(h(V))
  =         h(f(V))
  \subseteq \im h.
\]
Somit schränkt sich $g$ zu einem Endomorphismus $g|_{\im h} \colon \im h \to \im h$, $w \mapsto g(w)$ ein.

Aus der Kommutativität des Diagramms \eqref{diagram: given one} folgt dann die Kommutativität des eingeschränkten Diagramms
\[
  \begin{tikzcd}[column sep = large]
      V
      \arrow{r}{f}
      \arrow[swap]{d}{h'}
    & V
      \arrow{d}{h'}
    \\
      \im h
      \arrow{r}{g|_{\im h}}
    & \im h
  \end{tikzcd}
\]
Wegen der Bijektivität von $h'$ erhalten wir, dass $p_f(t) = p_{g|_{\im h}}(t)$, und aus Aufgabe~3 erhalten wir, dass $p_{g|_{\im h}}(t) \divides p_g(t)$.
Somit gilt $p_f(t) \mid p_g(t)$.
Insbesondere ist jede Nullstelle von $p_f(t)$ auch eine Nullstelle von $p_g(t)$, also jeder Eigenwert von $f$ auch ein Eigenwert von $g$.



\subsubsection*{\texorpdfstring{$h$}{h} ist surjektiv}

Die lineare Abbildung $h$ sei nun surjektiv.
Dann induziert $h$ einen Isomorphismus von $K$-Vektorräumen $\induced{h} \colon V/\ker h \to W$, $\class{v} \mapsto h(v)$.

Aus der Kommutativität des Diagramms \eqref{diagram: given one} folgt nun, dass der Untervektorraum $\ker h \subseteq V$ bereits $f$-invariant ist, denn es gilt
\[
    h(f(\ker h))
  = g(h(\ker h))
  = g(\{0\})
  = \{0\}.
\]
Nach Aufgabe~3 induziert $f$ einen Endomorphismus $\induced{f} \colon V/\ker h \to V/\ker h$, $\class{v} \mapsto \class{f(v)}$.

Aus der Kommutativität des Diagramms \eqref{diagram: given one} folgt die Kommutativität des induzierten Diagramms
\[
  \begin{tikzcd}
      V/\ker h
      \arrow{r}{\induced{f}}
      \arrow[swap]{d}{\induced{h}}
    & V/\ker h
      \arrow{d}{\induced{h}}
    \\
      W
      \arrow{r}{g}
    & W
  \end{tikzcd}
\]
Wegen der Bijektivität von $\induced{h}$ gilt $p_{\induced{f}}(t) = p_g(t)$.
Nach Aufgabe~3 gilt außerdem $p_{\induced{f}} \divides p_f(t)$.
Somit gilt $p_g(t) \divides p_f(t)$.
Deshalb ist jede Nullstelle von $p_g(t)$ auch eine Nullstelle von $p_f(t)$, also jeder Eigenwert von $g$ auch ein Eigenwert von $f$.





\subsection*{Beweis durch abgeändertes Diagramm}

Aus der Kommutativität des Diagramms \eqref{diagram: given one} folgt für alle $\lambda \in K$ die Kommutativität des folgenden abgeänderten Diagramms:
\begin{equation}
  \label{diagram: changed one}
  \begin{tikzcd}[column sep = huge]
      V
      \arrow{r}{f - \lambda \id_V}
      \arrow[swap]{d}{h}
    & V
      \arrow{d}{h}
    \\
      W
      \arrow{r}{g - \lambda \id_W}
    & W
  \end{tikzcd}
\end{equation}
Die Kommutativität von \eqref{diagram: given one} ist nämlich äquivalent zu der Gleichheit $h \circ f = g \circ h$.
Daraus folgt für alle $\lambda \in K$, dass
\[
    h \circ (f - \lambda \id_V)
  = h \circ f - \lambda h
  = g \circ h - \lambda h
  = (g - \lambda \id_W) \circ h
\]
gilt, was gerade die Kommutativität von \eqref{diagram: changed one} bedeutet.



\subsubsection{\texorpdfstring{$h$}{h} ist injektiv}

Es sei $h$ injektiv und $\lambda \in K$.
Ist $\lambda$ kein Eigenwert von $g$, so ist $g - \lambda \id_W$ injektiv.
Wegen der Injektivität von $h$ und Kommutativität von \eqref{diagram: changed one} ist damit auch $(g - \lambda \id_W) \circ h = h \circ (f - \lambda \id_V)$ injektiv.
Somit ist auch $f - \lambda \id_V$ injektiv, also $\lambda$ kein Eigenwert von $f$.



\subsubsection{\texorpdfstring{$h$}{h} ist surjektiv}

Es sei $h$ surjektiv und $\lambda \in K$.
Ist $\lambda$ kein Eigenwert von $f$, so ist $f - \lambda \id_V$ injektiv.
Wegen der Endlichdimensionalität von $V$ ist $f - \lambda \id_V$ somit auch surjektiv.
Wegen der Surjektivität von $h$ und der Kommutativität von \eqref{diagram: changed one} ist damit auch $h \circ (f - \lambda \id_V) = (g - \lambda \id_W) \circ h$ surjektiv.
Somit ist auch $g - \lambda \id_W$ surjektiv.
Wegen der Endlichdimensionalität ist $g - \lambda \id_W$ deshalb auch injektiv, also $\lambda$ kein Eigenwert von $g$.

