\section{}

Wir haben im Tutorium und auf dem Übungszettel die folgenden Aussagen gesehen:

\begin{lemma}
  Es sei $V$ ein endlichdimensionaler $K$-Vektorraum und $U \subseteq V$ ein Untervektorraum.
  Ist $(u_1, \dotsc, u_n, w_1, \dotsc, w_m)$ eine Basis von $V$, so dass $(u_1, \dotsc, u_n)$ eine Basis von $U$ ist, so ist $(\class{w_1}, \dotsc, \class{w_m})$ eine Basis von $V/U$.
\end{lemma}


\begin{lemma}
  Es sei $V$ ein $K$-Vektorraum, $f \colon V \to V$ ein Endomorphismus und $U \subseteq V$ ein $f$-invarianter Untervektorraum.
  Dann gibt eine eindeutige lineare Abbildung $\induced{f} \colon V/U \to V/U$, die das folgende Diagramm zum kommutieren bringt:
  \[
    \begin{tikzcd}
        V
        \arrow{r}{f}
        \arrow[swap]{d}{\pi}
      & V
        \arrow{d}{\pi}
      \\
        V/U
        \arrow[densely dashed]{r}{\induced{f}}
      & V/U
    \end{tikzcd}
  \]
  Dabei bezeichnet $\pi \colon V \to V/U$, $v \mapsto \class{v}$ die kanonische Projektion
  Die Abbildung $\induced{f}$ ist auf Elementen durch
  \[
    \induced{f}(\class{v}) = \class{f(v)}
    \qquad
    \text{für alle $v \in V$}
  \]
  gegeben.
\end{lemma}

\begin{lemma}
  \label{lemma: block form coming from invariant subspaces}
  Es sei $V$ eine endlichdimensionaler $K$-Vektorraum, $f \colon V \to V$ ein Endomorphismus und $U \subseteq V$ ein $f$-invarianter Untervektorraum.
  Es sei $\basis{B} = (u_1, \dotsc, u_n, w_1, \dotsc, w_m)$ eine Basis von $V$, so dass $\basis{B}_U = (u_1, \dotsc, u_n)$ eine Basis von $U$ ist.
  Dann gilt mit der Basis $\overline{\basis{B}} = (\class{w_1}, \dotsc, \class{w_m})$ von $V/U$, dass
  \[
      \repmatrix{f}{\basis{B}}{\basis{B}}
    = \begin{pmatrix}
        A & B
      \\
        0 & C
      \end{pmatrix},
  \]
  wobei $A = \repmatrix{f|_U}{\basis{B}_U}{\basis{B}_U} \in \matrices{n}{K}$, $C = \repmatrix{\induced{f}}{\overline{\basis{B}}}{\overline{\basis{B}}} \in \matrices{m}{K}$ und $B \in \mnatrices{n}{m}{K}$ gilt.
\end{lemma}


Wir möchten hier noch anmerken, dass sich die Blockform aus Lemma~\ref{lemma: block form coming from invariant subspaces} zu der folgenden nützlichen Beobachtung verallgemeinern lässt:

\begin{proposition}
  \label{proposition: better block form coming from chain of invariant subspaces}
  Es sei $V$ ein endlichdimensionaler $K$-Vektorraum und $f \colon V \to V$ ein Endomorphismus.
  Es sei
  \[
              0
    =         U_0
    \subseteq U_1
    \subseteq U_2
    \subseteq U_3
    \subseteq \dotsb
    \subseteq U_{t-1}
    \subseteq U_t
    =         V
  \]
  eine aufsteigende Kette von $f$-invarianten Untervektorräumen $U_i$.
  Es sei
  \[
      \basis{B}
    = \left(
        v^{(1)}_1, \dotsc, v^{(1)}_{n_1},
        v^{(2)}_1, \dotsc, v^{(2)}_{n_2},
        \dotsc,
        v^{(t)}_1, \dotsc, v^{(t)}_{n_t}
      \right)
  \]
  eine Basis von $V$, so dass für jedes $i = 1, \dotsc, t$ die Teilfamilie
  \[
      \basis{B}_i
    = \left(
        v^{(1)}_1, \dotsc, v^{(1)}_{n_1},
        v^{(2)}_1, \dotsc, v^{(2)}_{n_2},
        \dotsc,
        v^{(i)}_1, \dotsc, v^{(i)}_{n_i}
      \right)
  \]
  eine Basis von $U_i$ ist.
  \begin{enumerate}
    \item
      Die darstellende Matrix $\repmatrix{f}{\basis{B}}{\basis{B}}$ ist von der Form
      \begin{equation}
        \label{equation: general block matrix}
          \repmatrix{f}{\basis{B}}{\basis{B}}
        = \begin{pmatrix}
            A_1 &         & *
            \\
                & \ddots  &
            \\
            0   &         & A_t
          \end{pmatrix}
      \end{equation}
      mit $A_i \in \matrices{n_i}{K}$ für alle $i = 1, \dotsc, t$.
    \item
      Für alle $i = 1, \dotsc, t$ induziert $f$ einen Endomorphismus
      \[
                \induced{f}_i
        \colon  U_i/U_{i-1} \to U_i/U_{i-1},
        \quad   \class{v}
        \mapsto \class{f(v)}
      \]
      und bezüglich der Basis $\overline{\basis{B}}_i \coloneqq \left( \class{v^{(i)}_1}, \dotsc, \class{v^{(i)}_{n_i}} \right)$ von $U_i/U_{i-1}$ gilt $A_i = \repmatrix{\induced{f}_i}{\overline{\basis{B}}_i}{\overline{\basis{B}}_i}$.
      (Man bemerke, dass $U_0 = 0$ gilt, und somit $U_1/U_0 = U_1$ sowie $\induced{f}_1 = f|_{U_1}$.)
  \end{enumerate}
\end{proposition}

\begin{remark}
  Eine Basis $\basis{B}$ wie in Proposition~\ref{proposition: better block form coming from chain of invariant subspaces} entsteht etwa durch wiederholte Basisergänzung.
\end{remark}

\begin{example}
  Es sei $V$ ein endlichdimensionaler $K$-Vektorraum und $f \colon V \to V$ ein Endomorphismus.
  \begin{enumerate}
    \item
      Lemma~\ref{lemma: block form coming from invariant subspaces} ist ein Sonderfall von Proposition~\ref{proposition: better block form coming from chain of invariant subspaces}:
      Ist $U \subseteq V$ ein $f$-invarianter Untervektorraum, so erhalten wir die Kette von $f$-invarianten Untervektoräumen
      \[
                  0
        \subseteq U
        \subseteq V.
      \]
      Wendet man Proposition~\ref{proposition: better block form coming from chain of invariant subspaces} auf hierauf an, so ergibt sich Lemma~\ref{lemma: block form coming from invariant subspaces}.
    \item
      Es sei
      \[
                  0
        =         U_0
        \subseteq U_1
        \subseteq U_2
        \subseteq \dotsb
        \subseteq U_n
        =         V
      \]
      eine aufsteigende Folge von $f$-invarianten Untervektorräumen mit $\dim U_k = k$ für alle $k = 0, \dotsc, n$, d.h.\ es handle sich um eine $f$-stabile Fahne.
      Dann liefert Proposition~\ref{proposition: better block form coming from chain of invariant subspaces} die Trigonalisierbarkeit von $f$ (in \eqref{equation: general block matrix} sind die Matrizen $A_1, \dotsc, A_n$ dann alle von Größe $1 \times 1$, weshalb die Matrix selbst eine obere Dreicksmatrix ist).
  \end{enumerate}
\end{example}
