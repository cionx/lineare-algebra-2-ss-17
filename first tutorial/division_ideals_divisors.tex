\section{Polynomdivision, Hauptideale und größte gemeinsame Teiler}





\subsection{Teilbarkeit}

\begin{definition}
  Es sei $R$ ein kommutativer Ring und es seien $a, b \in R$.
  Dann ist $a$ ein \emph{Teiler} von $b$, bzw.\ $a$ \emph{teilt} $b$, falls es $c \in R$ mit $b = ac$ gibt.
  Man schreibt dann $a \divides b$.
\end{definition}

\begin{definition}
  Es sei $R$ eine kommutativer Ring, und es seien $a, b \in R$.
  Ein Element $d \in R$ heißt \emph{größter gemeinsamer Teiler} von $a$ und $b$, falls
  \begin{enumerate}
    \item
      $d$ ist ein gemeinsamer Teiler von $a$ und $b$, d.h.\ es gilt $d \divides a$ und $d \divides b$ und
    \item
      für jedes $d' \in R$ mit $d' \divides a$ und $d' \divides b$ gilt $d' \divides d$.
  \end{enumerate}
\end{definition}





\subsection{Polynomdivision}

Von nun an sei $K$ ein Körper.

\begin{theorem}[Polynomdivision, bzw.\ Teilen mit Rest]
  \label{theorem: polynomial division}
  Es seien $f, g \in K[t]$ mit $g \neq 0$.
  Dann gibt es eindeutige Polynome $q, r \in K[t]$ mit
  \begin{enumerate}
    \item
      $\deg(r) < \deg(g)$ und
    \item
      $f = qg + r$.
  \end{enumerate}
\end{theorem}

\begin{proof}
  \begin{itemize}
    \item
      Wir zeigen zunächst die Eindeutigkeit:
      
      Es seien $q, q', r, r' \in K[t]$ mit $\deg(r), \deg(r') < \deg(g)$ und $q g + r = f = q' g + r'$.
      Dann gilt
      \begin{equation}
        \label{equation: showing uniqueness}
        (q - q') g = r' - r,
      \end{equation}
      Zum einen gilt dabei, dass
      \[
              \deg(r' - r)
        \leq  \max\{ \deg(r'), \deg(-r) \}
        =     \max\{ \deg(r'), \deg(r) \}
        <     \deg(g),
      \]
      und zum anderen gilt
      \[
          \deg((q-q') g)
        = \deg(q-q') \deg(g).
      \]
      Daraus folgt, dass $\deg(q - q') \deg(g) < \deg(g)$ gilt.
      Somit muss $\deg(q - q') < 0$ gelten, also $\deg(q - q') = -\infty$ und deshalb $q - q' = 0$.
      Aus \eqref{equation: showing uniqueness} folgt damit, dass auch $r' - r = 0$ gilt.
    \item
      Wir zeigen nun die Existenz per Induktion über $n \coloneqq \deg f$.
      Dabei gilt $m \coloneqq \deg g \geq 0$, da $g \neq 0$ gilt.
      
      Als Induktionsanfang dient der Fall $n < m$.
      Dann lässt sich $q = 0$ und $r = f$ wählen.
      
      Es sei nun $n \geq m$, und es seien $f = a_n t^n + \sum_{i=0}^{n-1} a_i t^i$ und $g = b_m t^m + \sum_{j=0}^{m-1} b_j t^j$, wobei $b_m \neq 0$.
      Die beiden Polynome $f$ und $\frac{a_n}{b_m} t^{n-m} g$ haben dann den gleichen Grad (hierfür sorgt der Faktor $t^{n-m}$) sowie den gleichen Leitkoeffizienten (hierfür sorgt der Faktor $\frac{a_n}{b_m}$).
      In der Differenz $f - \frac{a_n}{b_m} t^{n-m} g$ löschen sich diese Leitkoeffizieten daher aus, weshalb
      \[
        \deg\left( f - \frac{a_n}{b_m} t^{n-m} g \right) < \deg(f)
      \]
      gilt.
      Nach Induktionsvoraussetzung gibt es deshalb $q, r \in K[t]$ mit $\deg(r) < \deg(g)$, so dass
      \begin{gather*}
          f - \frac{a_n}{b_m} t^{n-m} g
        = q g + r,
      \shortintertext{gilt, und somit auch}
          f
        = \left( \frac{a_n}{b_m} t^{n-m} + q \right) g + r.
      \end{gather*}
      Dies zeigt die Existenz.
    \qedhere
  \end{itemize}
\end{proof}

\begin{remark}
  Der obige Beweis von Satz~\ref{theorem: polynomial division} liefert ein konstruktives Verfahren zur Berechnung von $q$ und $r$.
\end{remark}





\subsection{Hauptideale}

Wir zeigen im Folgenden mithilfe der Polynomdivision den folgenden Satz:

\begin{theorem}
  \label{theorem: existence of greatest common divisors}
  Je zwei Polynome $f, g \in K[t]$ besitzen einen größten gemeinsamen Teiler $d \in K[t]$, und es gibt $a, b \in K[t]$ mit $d = af + bg$.
\end{theorem}

Wir führen einen Beweis mithilfe von Idealen.

\begin{definition}
  Eine Teilmenge $I \subseteq R$ eines kommutativen Rings $R$ heißt \emph{Ideal} falls $I$ eine Untergruppe der additiven Gruppe von $R$ ist, und $rx \in I$ für alle $r \in R$ und $x \in I$ gilt.
\end{definition}

\begin{example}
  Es sei $R$ ein kommutativer Ring.
  \begin{enumerate}
    \item
      Für jedes $a \in R$ ist $(a) = Ra \coloneqq \{ra \suchthat r \in R\}$ ein Ideal in $R$.
      Man bezeichnet ein Ideal dieser Form als \emph{Hauptideal}.
    \item
      Sind $I, J \subseteq R$ zwei Ideale, so ist auch $I + J \coloneqq \{x + y \suchthat x \in I, y \in J\}$ ein Ideal in $R$.
      Dies ist das kleinste Ideal in $R$, dass die beiden Ideale $I$ und $J$ enthält.
    \item
      Induktiv folgt, dass für alle Ideale $I_1, \dotsc, I_n \subseteq R$ auch
      \[
          I_1 + \dotsb + I_n
        = \{ x_1 + \dotsb + x_n \suchthat x_1 \in I_1, \dotsc, x_n \in I_n \}
      \]
      ein Ideal in $R$ ist.
      
      (Alternativ lässt sich auch direkt Nachrechnen, dass für jede Familie $(I_\lambda)_{\lambda \in \Lambda}$ von Idealen $I_\lambda \subseteq R$ die Summe
      \[
          \sum_{\lambda \in \Lambda} I_\lambda
        = \left\{
            \sum_{\lambda \in \Lambda} x_\lambda
          \suchthatscale
            \text{$x_\lambda \in I_\lambda$ für alle $\lambda \in \Lambda$},
            \text{$x_\lambda = 0$ für fast alle $\lambda \in \Lambda$}
          \right\}
      \]
      ein Ideal in $R$ ist.
      Dies ist das kleinste Ideal in $R$, dass alle $I_\lambda$ enhält.)
    \item
      Für alle $a_1, \dotsc, a_n$ ist
      \[
                  (a_1, \dotsc, a_n)
        \coloneqq (a_1) + \dotsb + (a_n)
        =         \{ r_1 a_1 + \dotsb + r_n a_n \suchthat r_1, \dotsc, r_n \in R \}
      \]
      ein Ideal in $R$.
  \end{enumerate}
\end{example}

\begin{proposition}
  \label{proposition: the polynomial ring is a principal ideal domain}
  Jedes Ideal $I \subseteq K[t]$ ist von der Form $I = (f)$ für ein $f \in I$, d.h.\ jedes Ideal in $K[t]$ ist ein Hauptideal.
\end{proposition}

\begin{proof}
  Ist $I = \{0\}$, so lässt sich $f = 0$ wählen.
  Wir betrachten daher im Folgenden nur den Fall $I \neq \{0\}$.
  
  Es sei $f \in I$ mit $f \neq 0$ von minimalen Grad.
  Dann gilt auch $af \in I$ für alle $a \in K[t]$, und somit $(f) \subseteq I$.
  Ist andererseits $g \in I$, so gibt es nach Satz~\ref{theorem: polynomial division} Polynome $q, r \in K[t]$ mit $\deg(r) < \deg(f)$ und $g = q f + r$.
  Dann gilt $r = g - qf \in I$.
  Wegen der Gradminimalität von $f$ muss bereits $r = 0$ gelten, und somit $g = q f \in (f)$.
\end{proof}

\begin{remark}
  Für einen kommutativen Ring $R$ sind die folgenden Bedingungen äquivalent:
  \begin{enumerate}
    \item
      $R$ ist ein Körper.
    \item
      Es gilt $R[t] \neq 0$ und in $R[t]$ ist ein „Teilen mit Rest“ wie in Satz~\ref{theorem: polynomial division} möglich, d.h.\ für alle $f, g \in R[t]$ mit $g \neq 0$ gibt es $q, r \in R[t]$ mit $\deg(r) < \deg(g)$ und $f = q g + r$ (die Eindeutigkeit von $q$ und $r$ wird hier nicht gefordert).
    \item
      Der Ring $R[t]$ ist ein Integritätsbereich und jedes Ideal $I \subseteq R[t]$ ist ein Hauptideal.
  \end{enumerate}
  Insbesondere lässt sich an den ringtheoretischen Eigenschaften des Polynomrings $R[t]$ schon erkennen, ob $R$ selbst ein Körper ist.
\end{remark}

Wir können nun den größten gemeinsamen Teiler zweier Polynome idealtheoretisch beschreiben, und erhalten als Korollar einen Beweis für Satz~\ref{theorem: existence of greatest common divisors}.

\begin{lemma}
  \label{lemma: characterization of the greatest common divisor via ideals}
  Es sei $R$ ein kommutativer Ring, und es seien $f, g \in R$.
  Gibt es $d \in R$ mit $(f, g ) = (d)$, so ist $d$ ein größter gemeinsamer Teiler von $f$ und $g$.
\end{lemma}

\begin{proof}
  Da $f, g \in (f,g) = (d)$ gilt, gibt es $a, b \in R$ mit $f = ad$ und $g = bd$, weshalb $d \divides f$ und $d \divides g$ gilt.
  Andererseits folgt aus $d \in (d) = (f,g)$, dass es $a, b \in R$ mit $d = af + bg$ gibt.
  Ist $d' \in R$ mit $d' \divides f$ und $d' \divides g$, so gilt deshalb auch $d' \divides (af + bg) = d$.
\end{proof}

\begin{proof}[Beweis von Satz~\ref{theorem: existence of greatest common divisors}]
  Nach Proposition~\ref{proposition: the polynomial ring is a principal ideal domain} gibt es $d \in K[t]$ mit $(f,g) = (d)$.
  Nach Lemma~\ref{lemma: characterization of the greatest common divisor via ideals} ist $d$ ein größter gemeinsamer Teiler von $f$ und $g$.
  Da $d \in (d) = (f,g)$ gilt, gibt es $a, b \in K[t]$ mit $d = af + bg$.
\end{proof}

\begin{remark}
  \begin{enumerate}
    \item
      Sind $d, d' \in K[t]$ zwei größte gemeinsame Teiler von $f, g \in K[t]$, so gibt es ein $\lambda \in K$, $\lambda \neq 0$ mit $d' = \lambda d$.
      Man spricht daher häufig von \emph{dem} größten gemeinsamen Teiler zweier Polynome, und bezeichnet diesen mit $\ggT(f,g)$.
    \item
      Da es in $K[t]$ eine Division mit Rest gibt (Satz~\ref{theorem: polynomial division}) lässt sich mithilfe des euklidischen Algorithmus ein größter gemeinsamer Teiler von $f, g \in K[t]$ berechnen, sowie entsprechende $a, b \in K[t]$ mit $\ggT(f,g) = af + bg$.
  \end{enumerate}
\end{remark}

\begin{remark}
  Auch in $\integer$ ist eine Division mit Rest möglich, d.h.\ für alle $n, m \in \integer$ mit $m \neq 0$ gibt es eindeutige $q, r \in \integer$ mit $n = qm + r$;
  dies lässt sich analog zu Satz~\ref{theorem: polynomial division} zeigen, wobei man anstelle des Grades $\deg$ mit dem Betrag $|\cdot|$ arbeitet.
  Analog zu Proposition~\ref{proposition: the polynomial ring is a principal ideal domain} lässt sich deshalb zeigen, dass jedes Ideal $I \subseteq \integer$ von der Form $I = (n)$ für ein $n \in \integer$ ist.
\end{remark}
