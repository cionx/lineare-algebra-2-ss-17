\section{Zur Definition des Polynomrings}

Es sei $R$ ein kommutativer Ring.
Auf der Menge der endlichen Folgen auf $R$
\begin{equation}
  \tag{D}
  \label{equation: definition}
  R[\natural]
  \coloneqq
  \{
    (a_n)_{n \in \natural}
  \mid
    \text{$a_i \in R$ für alle $i \in \natural$},
    \text{$a_i = 0$ für fast alle $i \in \natural$} 
  \}
\end{equation}
wird eine Addition
\begin{equation}
  \tag{A}
  \label{equation: addition}
    (a_n)_{n \in \natural} + (b_n)_{n \in \natural}
  = (c_n)_{n \in \natural}
  \quad\text{mit}\quad
  \text{$c_i = a_i + b_i$ für alle $i \in \natural$}
\end{equation}
und eine Multiplikation
\begin{equation}
  \tag{M}
  \label{equation: multiplication}
    (a_n)_{n \in \natural} \cdot (b_n)_{n \in \natural}
  = (c_n)_{n \in \natural}
  \quad\text{mit}\quad
  \text{$c_i = \sum_{j=0}^i a_j b_{i-j}$ für alle $i \in \natural$}
\end{equation}
definiert.
Zusammen mit dieser Addition und Multiplikation ist $R[\natural]$ ein kommutativer Ring.
Das Einselement ist gegeben durch $1_{R[\natural]} = (1, 0, 0, \dotsc)$.

Wir führen nun die Notation $t \coloneqq (0, 1, 0, 0, \dotsc)$ ein.
Induktiv ergibt sich für alle $n \geq 0$, dass
\begin{align*}
  t^0 &= (1, 0, 0, \dotsc),
  \\
  t^1 &= (0, 1, 0, 0, \dotsc),
  \\
  t^2 &= (0, 0, 1, 0, 0, \dotsc),
  \\
      &\vdots
  \\
  t^n &= (0, \dotsc, 0, 1, 0, 0, \dotsc),
\end{align*}
dass also $t^n = (\delta_{ni})_{i \in \natural}$.

Für alle $r, s \in R$ gilt
\begin{align*}
     (r, 0, 0, \dotsc) + (s, 0, 0, \dotsc)
  &= (r + s, 0, 0, \dotsc)
\shortintertext{und}
      (r, 0, 0, \dotsc) \cdot (s, 0, 0, \dotsc)
  &=  (r \cdot s, 0, 0, \dotsc).
\end{align*}
Wir können deshalb $R$ mit dem Unterring
\[
            \{ (r, 0, 0, \dotsc) \mid r \in R \}
  \subseteq R[\natural]
\]
identifzieren.
Für alle $r \in R$ und $(a_0, \dotsc, a_n, 0, 0, \dotsc) \in R[\natural]$ git dann, dass
\begin{equation}
  \tag{S}
  \label{equation: scalar multiplication} 
  \begin{aligned}
        r \cdot (a_0, a_1, \dotsc, a_n, 0, 0, \dotsc)
    &=  (r, 0, 0, \dotsc) \cdot (a_0, \dotsc, a_n, 0, 0, \dotsc)
    \\
    &=  (r a_0, \dotsc, r a_n, 0, 0, \dotsc).
  \end{aligned}
\end{equation}
Damit ergibt sich nun für jedes $(a_0, \dotsc, a_n, 0, 0, \dotsc) \in R[\natural]$, dass
\begin{align*}
   &\,  (a_0, a_1, \dotsc, a_n, 0, 0, \dotsc)
   \\
  =&\,  (a_0, 0, 0, \dotsc) + (0, a_1, 0, 0, \dotsc) + \dotsb + (0, \dotsc, 0, a_n, 0, 0, \dotsc)
  \\
  =&\,  a_0 (1, 0, 0, \dotsc) + a_1 (0, 1, 0, 0, \dotsc) + \dotsb + a_n (0, \dotsc, 0, 1, 0, 0, \dotsc)
  \\
  =&\,  a_0 t^0 + a_1 t^1 + \dotsb + a_n t^n
  =     \sum_{i=0}^n a_i t^i.
\end{align*}
Da $a_i = 0$ für alle $i > n$ gilt, lässt sich statt $\sum_{i=0}^n a_i t^i$  auch $\sum_{i=0}^\infty a_i t^i$ schreiben.

Anstelle von \eqref{equation: definition} schreibt man nun
\begin{equation}
  \tag{D'}
    R[t]
  = \left\{
      \sum_{i=0}^\infty a_i t^i
    \,\middle|\,
      \text{$a_i \in R$ für alle $i \in \natural$},
      \text{$a_i = 0$ für fast alle $i \in \natural$}
    \right\}.
\end{equation}
Die Addition \eqref{equation: addition} ist in dieser Schreibweise durch
\begin{equation}
  \tag{A'}
    \left( \sum_{i=0}^\infty a_i t^i \right) + \left( \sum_{i=0}^\infty b_i t^i \right)
  = \sum_{i=0}^\infty (a_i + b_i) t^i
\end{equation}
gegeben, und die Multiplikation \eqref{equation: multiplication} durch
\begin{equation}
  \tag{M'}
    \left( \sum_{i=0}^\infty a_i t^i \right) \cdot \left( \sum_{i=0}^\infty b_i t^i \right)
  = \sum_{i=0}^\infty \left( \sum_{j=0}^i a_i b_{j-i} \right) t^i.
\end{equation}
Für $r \in R$ und $\sum_{i=0}^\infty a_i t^i \in R[t]$ ist \eqref{equation: scalar multiplication} gegeben durch
\begin{equation}
  \tag{S'}
  \label{equation: scalar multiplication new}
    r \cdot \left( \sum_{i=0}^\infty a_i t^i \right)
  = \sum_{i=0}^\infty (r a_i) t^i.
\end{equation}


\begin{remark*}
  \begin{itemize}
    \item
      Man bezeichnet das obige Element $t$ as „Variable“.
    \item
      Statt „$t$“ lassen sich auch andere Buchstaben verwenden;
      beliebt sind $T$, $x$, $X$, $y$ und $Y$.
    \item
      Die Multiplikation auf $R[t]$ ist eindeutig dadurch bestimmt, dass
      \begin{enumerate}
        \item
          $t^i \cdot t^j = t^{i+j}$ für alle $i,j \in \natural$,
        \item
          $r \cdot (f \cdot g) = (r \cdot f) \cdot g = f \cdot (r \cdot g)$ für alle $r \in R$ und $f,g \in R[t]$,
        \item
          Die Multiplikation ist distributiv in beiden Argumenten.
      \end{enumerate}
    \item
      Ist $K$ ein Körper, so definiert \eqref{equation: scalar multiplication new} eine Skalarmultiplikation von $K$ auf $K[t]$, die zu einer $K$-Vektorraumstruktur auf $K[t]$ führt.
      Eine $K$-Basis von $K[t]$ ist dann durch die Familie $(t^n)_{n \in \natural}$ gegeben.
  \end{itemize}
\end{remark*}