\section{}
Es sei $d \in \naturals$.
Wir geben mehrere mögliche Beweise für die zu zeigenden Aussagen an:

Zunächst geben wir einen Beweis unter der zusätzlichen Annahme, dass $V$ und $W$ endlichdimensional sind.
Diese Annahme wurde in der Aufgabenstellung vergessen;
die zu zeigenden Aussagen werden dadurch zwar nicht falsch, aber schwieriger zu zeigen.

Anschließend zeigen wir, wie sich mithilfe des endlichdimensionalen Falls der allgemeine Fall beweisen lässt.

Schließlich geben wir auch den Beweis aus dem Tutorium noch einmal an, der ohne Rücksicht auf die Dimensionen von $V$ und $W$ funktioniert.





\subsection{Beweis im Endlichdimensionalen}

Die Vektorräume $V$ und $W$ seien zunächst endlichdimensional.



\subsubsection{Surjektivität}

Es sei $f \colon V \to W$ eine surjektive lineare Abbildung.
Es seien $w_1, \dotsc, w_d \in W$.
Wegen der Surjektivität von $f$ gibt es $v_1, \dotsc, v_d \in V$ mit $f(v_i) = w_i$ für alle $i = 1, \dotsc, d$.
Dann gilt auch
\[
      w_1 \wedge \dotsb \wedge w_d
  =   f(v_1) \wedge \dotsb \wedge f(v_d)
  =   \left( \exteriorpower{d}{f} \right)(v_1 \wedge \dotsb \wedge v_d)
  \in \im \exteriorpower{d}{f}.
\]
Da $\exteriorpower{d}{f}$ von den Elementen $w_1 \wedge \dotsb \wedge w_d$ mit $w_1, \dotsc, w_d$ erzeugt wird, gilt für den Untervektorraum $\im \exteriorpower{d}{f} \subseteq W$ deshalb bereits $\exteriorpower{d}{f} = W$.



\subsubsection{Injektivität}

Wir erinnern an die folgende Aussage aus der Linearen~Algebra~I:

\begin{lemma}
  \label{lemma: checking injectivity on a basis}
  Es sei $f \colon V \to W$ eine lineare Abbildung.
  Ist $(v_i)_{i \in I}$ eine Basis von $V$, so dass die Familie $(f(v_i))_{i \in I}$ ebenfalls linear unabhängig ist, so ist $f$ injektiv.
\end{lemma}

\begin{proof}
  Ist $v \in \ker f$ mit $v = \sum_{i \in I} \lambda_i v_i$, so folgt durch Anwenden von $f$, dass
  \[
      0
    = f(v)
    = f\left( \sum_{i \in I} \lambda_i v_i \right)
    = \sum_{i \in I} \lambda_i f(v_i)
  \]
  gilt.
  Aus der linearen Unabhängigkeit der Familie $(f(v_i))_{i \in I}$ folgt damit, dass $\lambda_i = 0$ für alle $i \in I$, und somit bereits $v = 0$.
  Das zeigt, dass $\ker f = 0$ gilt.
\end{proof}

Die Abbildung $f$ sei injektiv.
Es sei $(v_1, \dotsc, v_n)$ eine Basis von $V$.
Wegen der Injektivität von $f$ ist dann auch die Familie $(f(v_1), \dotsc, f(v_n))$ linear unanabhängig, und lässt sich deshalb zu einer Basis $(w_1, \dotsc, w_m)$ von $W$ mit $w_i = f(v_i)$ für alle $i = 1, \dotsc, n$ ergänzen.

Die Familie
\[
    \basis{B}
  = (
      v_{i_1} \wedge \dotsb \wedge v_{i_d}
      \suchthat
      1 \leq i_1 < \dotsb < i_d \leq n
    )
\]
ist dann eine Basis von $\exteriorpower{d}{V}$, und die Familie
\[
    \basis{C}
  = (
      w_{j_1} \wedge \dotsb \wedge w_{j_d}
      \suchthat
      1 \leq j_1 < \dotsb < j_d \leq m
    )
\]
ist eine Basis von $\exteriorpower{d}{W}$.
Für die lineare Abbildung $\exteriorpower{d}{f} \colon \exteriorpower{d}{V} \to \exteriorpower{d}{W}$ gilt dann
\[
    \left(\exteriorpower{d}{f}\right)(v_{i_1} \wedge v_{i_d})
  = f(v_{i_1}) \wedge f(v_{i_d})
  = w_{i_1} \wedge w_{i_d}
  \qquad
  \text{für alle $1 \leq i_1 < \dotsb < i_d \leq n$}.
\]
Die lineare Abbildung $\exteriorpower{d}{f}$ bildet also die Basis $\basis{B}$ injektiv auf eine Teilfamilie von $\basis{C}$ ab, und somit insbesondere auf eine linear unabhängige Familie.
Nach Lemma~\ref{lemma: checking injectivity on a basis} ist $\exteriorpower{d}{f}$ somit injektiv.



\subsection{Beweis für beliebig-dimensionale Räume}

Es sei $f \colon V \to W$ eine lineare Abbildung, wobei $V$ und $W$ zwei nicht notwendigerweise endlichdimensionale Vektorräume sind.



\subsection{Surjektivität}

Falls $f$ surjektiv ist, so ist $\exteriorpower{d}{f}$ nach der gleichen Arugmentation wie im endlichdimensionalen Fall ebenfalls surjektiv.



\subsection{Injektivität}

Die Abbildung $f$ sei nun injektiv.
Wir führen die Injektvität von $\exteriorpower{d}{f}$ auf den endlichdimensionalen Fall zurück.
Hierfür nutzen wir die konkrete Konstruktion von $\exteriorpower{d}{V}$ und $\exteriorpower{d}{W}$ als Quotienten von freien Vektorräumen:

Es seien $F_V \coloneqq K^{(V^d)}$ und $F_W \coloneqq K^{(W^d)}$ die freien Vektorräume auf den Mengen $V^d$ und $W^d$.
Für alle $v_1, \dotsc, v_n \in V$ und $w_1, \dotsc, w_d \in W$ bezeichnen wir die entsprechenden Basiselemente mit $e_{(v_1, \dotsc, v_d)} \in F_V$ und $e_{(w_1, \dotsc, w_d)} \in F_W$.
Dann gilt $\exteriorpower{d}{V} = F_V/N_V$ und $\exteriorpower{d}{V} = F_W/N_W$, wobei $N_V \subseteq F_V$ den Untervektorraum bezeichnet, den von den Elementen der Form
\begin{itemize}
  \item
    $e_{(v_1, \dotsc, v_{i-1}, a v_i + b v_i', v_{i+1}, \dotsc, v_n)}$ mit $v_1, \dotsc, v_{i-1}, v_i, v_i', v_{i+1}, \dotsc, v_n \in V$ und $a, b \in K$, und
  \item
    $e_{(v_1, \dotsc, v_n)}$ mit $v_1, \dotsc, v_n \in V$, so dass $v_i = v_j$ für passende $i \neq j$
\end{itemize}
erzeugt wird, und $N_W$ den analogen Untervektorraum von $F_W$.
Es seien
\[
          p_V
  \colon  F_V
  \to     F_V/N_V
  =       \exteriorpower{d}{V},
  \quad   y
  \mapsto \class{y}
  \quad\text{und}\quad
  p_W
  \colon  F_W
  \to     F_W/N_W
  =       \exteriorpower{d}{V},
  \quad   y
  \mapsto \class{y}
\]
die kanonischen Projektionen.

Nach der universellen Eigenschaft des freien Vektorraums induziert die lineare Abbildung $f$ eine eindeutige lineare Abbildung $f_* \colon F_V \to F_W$ mit
\[
    f_*( e_{(v_1, \dotsc, v_d)} )
  = e_{(f(v_1), \dotsc, f(v_d))}
  \qquad
  \text{für alle $v_1, \dotsc, v_d \in V$}.
\]
Wegen der Injektivität von $f$ bildet $f_*$ die Basis $(e_{(v_1, \dotsc, v_d)} \suchthat v_1, \dotsc, v_d \in V)$ von $F_V$ injektiv in die Basis $(e_{(w_1, \dotsc, w_d)} \suchthat w_1, \dotsc, w_d \in W)$ ab;
nach Lemma~\ref{lemma: checking injectivity on a basis} ist $f_*$ somit injektiv.
Wir haben nun das folgende kommutative Diagram:
\[
  \begin{tikzcd}[column sep = large]
      F_V
      \arrow{r}{f_*}
      \arrow[swap]{d}{p_V}
    & F_W
      \arrow{d}{p_W}
    \\
      \exteriorpower{d}{V}
      \arrow{r}{\exteriorpower{d}{f}}
    & \exteriorpower{d}{W}
  \end{tikzcd}
\]
Die Kommutativität ergibt sich durch direktes Nachrechnen:
Für alle $v_1, \dotsc, v_d \in V$ gilt
\begin{align*}
      p_W( f_*( e_{(v_1, \dotsc, v_d)} )
  &=  p_W( e_{(f(v_1), \dotsc, f(v_d))} )
   =  f(v_1) \wedge \dotsb \wedge f(v_d)
  \\
  &=  \left( \exteriorpower{d}{f} \right)( v_1 \wedge \dotsb \wedge v_d )
   =  \left( \exteriorpower{d}{f} \right)( p_V( e_{(v_1, \dotsc, v_d)} ) ).
\end{align*}
Das zeigt, dass die beiden Kompositionon $p_W \circ f_*$ und $(\exteriorpower{d}{f}) \circ p_V$ bereits auf der Basis $(e_{(v_1, \dotsc, v_d)} \suchthat v_1, \dotsc, v_d \in V)$ von $F_V$ übereinstimmen.
Wegen der Linearität der beiden Abbildungen stimmen sie damit schon auf ganz $F_V$ überein.











\subsection{Beweis durch Links- und Rechtsinverse}

Vollständigkeitshalber geben wir auch die im Tutorium genutzte Argumentation noch einmal.

\begin{recall}
  Es seien $A$ und $B$ zwei Mengen.
  Eine Funktion $f \colon A \to B$ ist genau dann injektiv, wenn sie ein Linksinverses besitzt, d.h.\ wenn es eine Funktion $g \colon B \to A$ mit $g \circ f = \id_A$ gibt.
  Analog ist $f$ genau dann surjektiv, wenn $f$ ein Rechtsinverses besitzt, d.h.\ wenn es eine Funktion $g \colon B \to A$ mit $f \circ g = \id_B$ gibt.
\end{recall}

\begin{remark}
  Dass jede Surjekton $f \colon A \to B$ ein Rechtsinverses besitzt, ist äquivalent zum Auswahlaxiom.
\end{remark}

Für lineare Injektionen, bzw.\ Surjektionen lässen sich ein entsprechendes Links- bzw.\ Rechtsinverses ebenfalls linear wählen.

\begin{lemma}
  \label{lemma: inverses can be choosen linearly}
  Es seien $V$ und $W$ zwei $K$-Vektorräume und $f \colon V \to W$ linear.
  \begin{enumerate}
    \item
      $f$ ist genau dann injektiv, wenn $f$ ein lineares Linksinverses besitzt, d.h.\ wenn es eine lineare Abbildung $g \colon W \to V$ mit $g \circ f = \id_V$ gibt.
    \item
      $f$ ist genau dann surjektiv, wenn $f$ ein lineares Rechtsinverses besitzt, d.h.\ wenn es eine lineare Abbildung $g \colon W \to V$ mit $f \circ g = \id_W$ gibt.
  \end{enumerate}
\end{lemma}

\begin{warning}
  Ist $f \colon V \to W$ linear und injektiv (bzw.\ surjektiv) und ist $g \colon W \to V$ ein Linksinverses (bzw.\ Rechtsinverses) zu $f$, so muss $g$ im Allgemeinen nicht linear sein.
  Lemma~\ref{lemma: inverses can be choosen linearly} sagt, dass es ein lineares Linksinverses (bzw.\ Rechtsinverses) gibt, nicht aber, dass bereits jedes Linksinverse (bzw.\ Rechtsinverse) schon linear ist.
\end{warning}

\begin{proposition}
  \label{proposition: functoriality of exterior power}
  Es seien $U$, $V$ und $W$ drei $K$-Vektorräume, und es sei $d \in \naturals$.
  \begin{enumerate}
    \item
      Für alle linearen Abbildungen $f \colon U \to V$ und $g \colon V \to W$ gilt
      \[
          \exteriorpower{d}{(f \circ g)}
        = \exteriorpower{d}{f} \circ \exteriorpower{d}{g}.
      \]
    \item
      Es gilt $\exteriorpower{d}{ \id_V } = \id_{\exteriorpower{d}{V}}$.
  \end{enumerate}
\end{proposition}

Wir haben im Tutorium bereits einen Beweis hierfür gegeben;
wir geben den gleichen Beweis hier noch einmal an, allerdings in einer diagrammatischen Form.

\begin{proof}
  \begin{enumerate}
    \item
      Die beiden Abbildungen $\exteriorpower{d}{f}$ und $\exteriorpower{d}{g}$ sind die eindeutigen lineare Abbildung $\exteriorpower{d}{U} \to \exteriorpower{d}{V}$, bzw.\ $\exteriorpower{d}{V} \to \exteriorpower{d}{W}$, so dass die beiden Diagramme
      \[
        \begin{tikzcd}[column sep = large]
            U^d
            \arrow{r}{f^{\times d}}
            \arrow[swap]{d}{\wedge}
          & V^d
            \arrow{d}{\wedge}
          \\
            \exteriorpower{d}{U}
            \arrow[swap, dashed]{r}{\exteriorpower{d}{f}}
          & \exteriorpower{d}{V}
        \end{tikzcd}
        \quad\text{und}\quad
        \begin{tikzcd}[column sep = large]
            V^d
            \arrow{r}{g^{\times d}}
            \arrow[swap]{d}{\wedge}
          & W^d
            \arrow{d}{\wedge}
          \\
            \exteriorpower{d}{V}
            \arrow[swap, dashed]{r}{\exteriorpower{d}{g}}
          & \exteriorpower{d}{W}
        \end{tikzcd}
      \]
      kommutieren, wobei $\wedge$ die jeweils kanonischen Abbildungen bezeichnet.
      Durch Zusammenfügen dieser beiden kommutativen Diagramme ergibt sich das folgende kommutative Diagramm:
      \[
        \begin{tikzcd}[column sep = large]
            U^d
            \arrow{r}{f^{\times d}}
            \arrow[swap]{d}{\wedge}
          & V^d
            \arrow{r}{g^{\times d}}
            \arrow{d}{\wedge}
          & W^d
            \arrow{d}{\wedge}
          \\
            \exteriorpower{d}{U}
            \arrow[swap, dashed]{r}{\exteriorpower{d}{f}}
          & \exteriorpower{d}{V}
            \arrow[swap, dashed]{r}{\exteriorpower{d}{g}}
          & \exteriorpower{d}{W}
        \end{tikzcd}
      \]
      Durch Entfernen des mittleren vertikalen Pfeils und die Gleichheit $g^{\times d} \circ f^{\times d} = (g \circ f)^{\times d}$ erhalten wir hieraus das folgende kommutative Diagramm:
      \[
        \begin{tikzcd}[column sep = 6em]
            U^d
            \arrow{r}{(g \circ f)^{\times d}}
            \arrow[swap]{d}{\wedge}
          & W^d
            \arrow{d}{\wedge}
          \\
            \exteriorpower{d}{U}
            \arrow[swap, dashed]{r}{\exteriorpower{d}{g}\circ \exteriorpower{d}{f}}
          & \exteriorpower{d}{W}
        \end{tikzcd}
      \]
      Dabei ist $\exteriorpower{d}{g} \circ \exteriorpower{d}{f}$ als Komposition zweier linearer Abbildungen ebenfalls linear.
      Nun ist aber $\exteriorpower{d}{(g \circ f)}$ die eindeutige lineare Abbildung $\exteriorpower{d}{U} \to \exteriorpower{d}{W}$ die dieses Diagramm zum kommutieren bringt.
      Folglich müssen beide Abbildungen übereinstimmen, d.h.\ es gilt $\exteriorpower{d}{g} \circ \exteriorpower{d}{f} = \exteriorpower{d}{(g \circ f)}$.
      
    \item
      Die Abbildung $\exteriorpower{d}{\id_V}$ ist die eindeutige lineare Abbildung $\exteriorpower{d}{V} \to \exteriorpower{d}{V}$ die das folgende Diagramm zum Kommutieren bringt:
      \[
        \begin{tikzcd}[column sep = large]
            V^d
            \arrow{r}{\id_V^{\times d}}
            \arrow[swap]{d}{\wedge}
          & V^d
            \arrow{d}{\wedge}
          \\
            \exteriorpower{d}{V}
            \arrow[swap, dashed]{r}{\exteriorpower{d}{\id_V}}
          & \exteriorpower{d}{V}
        \end{tikzcd}
      \]
      Dabei gilt $\id_V^{\times d} = \id_{V^d}$, weshalb auch die lineare Abbildung $\id_{\exteriorpower{d}{V}}$ das Diagramm zum kommutieren bringt.
      Es folgt, dass beide Abbildungen bereits übereinstimmen, dass also $\exteriorpower{d}{\id_V} = \id_{\exteriorpower{d}{V}}$ gilt.
    \qedhere
  \end{enumerate}
\end{proof}

Ist nun $f \colon V \to W$ eine lineare Abbildung.
Ist $f$ injektiv, so gibt es nach Lemma~\ref{lemma: inverses can be choosen linearly} eine lineare Abbildung $g \colon W \to V$ mit $g \circ f = \id_V$;
dann gilt nach Proposition~\ref{proposition: functoriality of exterior power} auch $\exteriorpower{d}{g} \circ \exteriorpower{d}{f} = \id_{\exteriorpower{d}{V}}$ und somit nach Proposition~\ref{proposition: functoriality of exterior power}, dass $\exteriorpower{d}{f}$ injektiv ist.
Ist $f$ surjektiv, so ergibt sich durch analoge Argumentation mithilfe eines Rechtsinversen von $f$, dass auch $\exteriorpower{d}{f}$ surjektiv ist.





% \subsection{Beweis mit Elementen}
% 
% 
% 
% \subsubsection{Surjektivität}
% Es sei $f \colon V \to W$ eine surjektive lineare Abbildung.
% Für alle $w_1, \dotsc, w_d \in W$ gibt es dann $v_1, \dotsc, v_d \in V$ mit $f(v_i) = w_i$ für alle $i = 1, \dotsc, n$, weshalb
% \[
%       w_1 \wedge \dotsb \wedge w_d
%   =   f(v_1) \wedge \dotsb \wedge f(v_d)
%   =   \left(\exteriorpower{d}{f}\right)(v_1, \dotsc, v_d)
%   \in \im \exteriorpower{d}{f}
% \]
% gilt.
% Da $\exteriorpower{d}{W}$ von den Elmenten $w_1 \wedge \dotsb \wedge w_d$ mit $w_1, \dotsc, w_d \in W$ erzeugt wird, folgt hieraus, dass bereits $\im \exteriorpower{d}{f} = W$ gilt.
% Also ist auch $\exteriorpower{d}{f}$ surjektiv.
% 
% 
% 
% \subsubsection{Injektivität}
% 
% \begin{proposition}
%   Es sei $V$ ein Vektorraum und $(v_i)_{i \in I}$ eine Basis von $V$, wobei $(I, \leq)$ eine linear geordnete Menge ist.
%   Dann bilden die Elemente
%   \[
%     v_{i_1} \wedge \dotsb \wedge v_{i_d}
%     \quad\text{mit}\quad
%     i_1, \dotsc, i_n \in I,
%     \,
%     i_1 < \dotsb < i_d
%   \]
%   eine Basis von $\exteriorpower{d}{V}$.
% \end{proposition}
% 
% \begin{remark}
%   Mithilfe des Auswahlaxioms lässt sich zeigen, dass es auf jeder Menge $I$ eine lineare Ordnung $\leq$ gibt.
%   Tatsächlich besagt der zum Auswahlaxiom äquivalente \emph{Wohlordnungsatz}, dass es auf jeder Menge $I$ eine Wohlordnung gibt, d.h.\ eine lineare Ordnung, bezüglich dessen jede nicht-leere Teilmenge $J \subseteq I$ ein minimales Element besitzt.
% \end{remark}
% 
% Wir erinnern auch noch an die folgende Aussage aus der Linearen~Algebra~I:
% 
% \begin{lemma}
%   \label{lemma: checking injectivity on a basis}
%   Es sei $f \colon V \to W$ eine lineare Abbildung.
%   Ist $(v_i)_{i \in I}$ eine Basis von $V$, so dass die Familie $(f(v_i))_{i \in I}$ ebenfalls linear unabhängig ist, so ist $f$ injektiv.
% \end{lemma}
% 
% \begin{proof}
%   Ist $v \in \ker f$ mit $v = \sum_{i \in I} \lambda_i v_i$, so folgt durch Anwenden von $f$, dass
%   \[
%       0
%     = f(v)
%     = f\left( \sum_{i \in I} \lambda_i v_i \right)
%     = \sum_{i \in I} \lambda_i f(v_i)
%   \]
%   gilt.
%   Aus der linearen Unabhängigkeit der Familie $(f(v_i))_{i \in I}$ folgt damit, dass $\lambda_i = 0$ für alle $i \in I$, und somit bereits $v = 0$.
%   Das zeigt, dass $\ker f = 0$ gilt.
% \end{proof}
% 
% 
% Es sei $f \colon V \to W$ eine injektive lineare Abbildung.
% Es sei $(v_i)_{i \in I}$ eine Basis von $V$.
% Wegen der Injektivität von $f$ ist auch die Familie $(f(v_i))_{i \in I}$ linear unabhängig;
% diese Familie lässt sich deshalb zu einer Basis $(w_j)_{j \in J}$ von $W$ mit $I \subseteq J$ und $w_i = f(v_i)$ für alle $i \in I$ ergänzen.
% Es sei $\leq$ eine lineare Ordnung auf $J$.
% Dann erhalten wir eine Basis von $\exteriorpower{d}{V}$ durch
% \[
%             \basis{B}
%   \coloneqq (
%               v_{i_1} \wedge \dotsb \wedge v_{i_d}
%             \suchthat 
%               i_1, \dotsc, i_d \in I,
%               i_1 < \dotsb < i_d
%             )
% \]
% und eine Basis von $\exteriorpower{d}{W}$ durch
% \[
%             \basis{C}
%   \coloneqq (
%               w_{j_1} \wedge \dotsb \wedge w_{j_d}
%             \suchthat 
%               j_1, \dotsc, j_d \in I,
%               j_1 < \dotsb < j_d
%             ).
% \]
% Für alle $i_1, \dotsc, i_d \in I$ gilt dabei, dass
% \[
%     \left( \exteriorpower{d}{f} \right)(v_{i_1} \wedge \dotsb \wedge v_{i_d})
%   = f(v_{i_1}) \wedge \dotsb \wedge f(v_{i_d})
%   = w_{i_1} \wedge \dotsb \wedge w_{i_d}.
% \]
% Also bildet $\exteriorpower{d}{f}$ die Basis $\basis{B}$ von $\exteriorpower{d}{V}$ injektiv auf eine Teilfamilie der Basis $\basis{C}$ von $\exteriorpower{d}{W}$ ab, die wegen der linearen Unabhängigkeit von $\basis{C}$ ebenfalls linear unabhängig ist. Somit ist $\exteriorpower{d}{f}$ nach Lemma~\ref{lemma: checking injectivity on a basis} injektiv.


