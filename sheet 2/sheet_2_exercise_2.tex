\section{}

\subsection{Zauberbeweis}

\begin{recall}
  Es seien $A$ und $B$ zwei Mengen.
  Eine Funktion $f \colon A \to B$ ist genau dann injektiv, wenn sie ein Linksinverses besitzt, d.h.\ wenn es eine Funktion $g \colon B \to A$ mit $g \circ f = \id_A$ gibt.
  Analog ist $f$ genau dann surjektiv, wenn $f$ ein Rechtsinverses besitzt, d.h.\ wenn es eine Funktion $g \colon B \to A$ mit $f \circ g = \id_B$ gibt.
\end{recall}

\begin{remark}
  Dass jede Surjekton $f \colon A \to B$ ein Rechtsinverses besitzt, ist äquivalent zum Auswahlaxiom.
\end{remark}

Für lineare Injektionen, bzw.\ Surjektionen lässen sich ein entsprechendes Links- bzw.\ Rechtsinverses ebenfalls linear wählen.

\begin{lemma}
  \label{lemma: inverses can be choosen linearly}
  Es seien $V$ und $W$ zwei $K$-Vektorräume und $f \colon V \to W$ linear.
  \begin{enumerate}
    \item
      $f$ ist genau dann injektiv, wenn $f$ ein lineares Linksinverses besitzt, d.h.\ wenn es eine lineare Abbildung $g \colon W \to V$ mit $g \circ f = \id_V$ gibt.
    \item
      $f$ ist genau dann surjektiv, wenn $f$ ein lineares Rechtsinverses besitzt, d.h.\ wenn es eine lineare Abbildung $g \colon W \to V$ mit $f \circ g = \id_W$ gibt.
  \end{enumerate}
\end{lemma}

\begin{warning}
  Ist $f \colon V \to W$ linear und injektiv (bzw.\ surjektiv) und ist $g \colon W \to V$ ein Linksinverses (bzw.\ Rechtsinverses) zu $f$, so muss $g$ im Allgemeinen nicht linear sein.
  Lemma~\ref{lemma: inverses can be choosen linearly} sagt, dass es ein lineares Linksinverses (bzw.\ Rechtsinverses) gibt, nicht aber, dass bereits jedes Linksinverse (bzw.\ Rechtsinverse) schon linear ist.
\end{warning}

\begin{proposition}
  \label{proposition: functoriality of exterior power}
  Es seien $U$, $V$ und $W$ drei $K$-Vektorräume, und es sei $d \in \naturals$.
  \begin{enumerate}
    \item
      Für alle linearen Abbildungen $f \colon U \to V$ und $g \colon V \to W$ gilt
      \[
          \exteriorpower{d}{(f \circ g)}
        = \exteriorpower{d}{f} \circ \exteriorpower{d}{g}.
      \]
    \item
      Es gilt $\exteriorpower{d}{ \id_V } = \id_{\exteriorpower{d}{V}}$.
  \end{enumerate}
\end{proposition}

Wir haben im Tutorium bereits einen Beweis hierfür gegeben;
wir geben den gleichen Beweis hier noch einmal an, allerdings in einer diagrammatischen Form.

\begin{proof}
  \begin{enumerate}
    \item
      Die beiden Abbildungen $\exteriorpower{d}{f}$ und $\exteriorpower{d}{g}$ sind die eindeutigen lineare Abbildung $\exteriorpower{d}{U} \to \exteriorpower{d}{V}$, bzw.\ $\exteriorpower{d}{V} \to \exteriorpower{d}{W}$, so dass die beiden Diagramme
      \[
        \begin{tikzcd}[column sep = large]
            U^d
            \arrow{r}{f^{\times d}}
            \arrow[swap]{d}{\wedge}
          & V^d
            \arrow{d}{\wedge}
          \\
            \exteriorpower{d}{U}
            \arrow[swap, dashed]{r}{\exteriorpower{d}{f}}
          & \exteriorpower{d}{V}
        \end{tikzcd}
        \quad\text{und}\quad
        \begin{tikzcd}[column sep = large]
            V^d
            \arrow{r}{g^{\times d}}
            \arrow[swap]{d}{\wedge}
          & W^d
            \arrow{d}{\wedge}
          \\
            \exteriorpower{d}{V}
            \arrow[swap, dashed]{r}{\exteriorpower{d}{g}}
          & \exteriorpower{d}{W}
        \end{tikzcd}
      \]
      kommutieren, wobei $\wedge$ die jeweils kanonischen Abbildungen bezeichnet.
      Durch Zusammenfügen dieser beiden kommutativen Diagramme ergibt sich das folgende kommutative Diagramm:
      \[
        \begin{tikzcd}[column sep = large]
            U^d
            \arrow{r}{f^{\times d}}
            \arrow[swap]{d}{\wedge}
          & V^d
            \arrow{r}{g^{\times d}}
            \arrow{d}{\wedge}
          & W^d
            \arrow{d}{\wedge}
          \\
            \exteriorpower{d}{U}
            \arrow[swap, dashed]{r}{\exteriorpower{d}{f}}
          & \exteriorpower{d}{V}
            \arrow[swap, dashed]{r}{\exteriorpower{d}{g}}
          & \exteriorpower{d}{W}
        \end{tikzcd}
      \]
      Durch Entfernen des mittleren vertikalen Pfeils und die Gleichheit $g^{\times d} \circ f^{\times d} = (g \circ f)^{\times d}$ erhalten wir hieraus das folgende kommutative Diagramm:
      \[
        \begin{tikzcd}[column sep = 6em]
            U^d
            \arrow{r}{(g \circ f)^{\times d}}
            \arrow[swap]{d}{\wedge}
          & W^d
            \arrow{d}{\wedge}
          \\
            \exteriorpower{d}{U}
            \arrow[swap, dashed]{r}{\exteriorpower{d}{g}\circ \exteriorpower{d}{f}}
          & \exteriorpower{d}{W}
        \end{tikzcd}
      \]
      Dabei ist $\exteriorpower{d}{g} \circ \exteriorpower{d}{f}$ als Komposition zweier linearer Abbildungen ebenfalls linear.
      Nun ist aber $\exteriorpower{d}{(g \circ f)}$ die eindeutige lineare Abbildung $\exteriorpower{d}{U} \to \exteriorpower{d}{W}$ die dieses Diagramm zum kommutieren bringt.
      Folglich müssen beide Abbildungen übereinstimmen, d.h.\ es gilt $\exteriorpower{d}{g} \circ \exteriorpower{d}{f} = \exteriorpower{d}{(g \circ f)}$.
      
    \item
      Die Abbildung $\exteriorpower{d}{\id_V}$ ist die eindeutige lineare Abbildung $\exteriorpower{d}{V} \to \exteriorpower{d}{V}$ die das folgende Diagramm zum Kommutieren bringt:
      \[
        \begin{tikzcd}[column sep = large]
            V^d
            \arrow{r}{\id_V^{\times d}}
            \arrow[swap]{d}{\wedge}
          & V^d
            \arrow{d}{\wedge}
          \\
            \exteriorpower{d}{V}
            \arrow[swap, dashed]{r}{\exteriorpower{d}{\id_V}}
          & \exteriorpower{d}{V}
        \end{tikzcd}
      \]
      Dabei gilt $\id_V^{\times d} = \id_{V^d}$, weshalb auch die lineare Abbildung $\id_{\exteriorpower{d}{V}}$ das Diagramm zum kommutieren bringt.
      Es folgt, dass beide Abbildungen bereits übereinstimmen, dass also $\exteriorpower{d}{\id_V} = \id_{\exteriorpower{d}{V}}$ gilt.
    \qedhere
  \end{enumerate}
\end{proof}

Ist nun $f \colon V \to W$ eine lineare Abbildung.
Ist $f$ injektiv, so gibt es nach Lemma~\ref{lemma: inverses can be choosen linearly} eine lineare Abbildung $g \colon W \to V$ mit $g \circ f = \id_V$;
dann gilt nach Proposition~\ref{proposition: functoriality of exterior power} auch $\exteriorpower{d}{g} \circ \exteriorpower{d}{f} = \id_{\exteriorpower{d}{V}}$ und somit nach Proposition~\ref{proposition: functoriality of exterior power}, dass $\exteriorpower{d}{f}$ injektiv ist.
Ist $f$ surjektiv, so ergibt sich durch analoge Argumentation mithilfe eines Rechtsinversen von $f$, dass auch $\exteriorpower{d}{f}$ surjektiv ist.





\subsection{Beweis mit Elementen}

\begin{proposition}
  Es sei $V$ ein Vektorraum und $(v_i)_{i \in I}$ eine Basis von $V$, wobei $(I, \leq)$ eine linear geordnete Menge ist.
  Dann bilden die Elemente
  \[
    v_{i_1} \wedge \dotsb \wedge v_{i_d}
    \quad\text{mit}\quad
    i_1, \dotsc, i_n \in I,
    \,
    i_1 < \dotsb < i_d
  \]
  eine Basis von $\exteriorpower{d}{V}$.
\end{proposition}
