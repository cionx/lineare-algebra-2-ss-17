\section{}

Es sei $d \in \naturals$.
Für jeden Vektorraum $V$ sei $\wedge_V \colon V^d \to \exteriorpower{d}{V}$, $(v_1, \dotsc, v_d) \mapsto v_1 \wedge \dotsb \wedge v_d$ die kanonische Projektion.
Für alle $K$-Vektorräume $V$ und $W$ haben wir einen Isomorphismus von $K$-Vektorräumen
\[
          \Phi_{V, W}
  \colon  \Alt( V^d, W )
  \to     \Hom\left( \exteriorpower{d}{V}, W \right),
  \quad   f
  \to     \induced{f},
\]
wobei $\induced{f} \colon \exteriorpower{d}{V} \to W$ die eindeutige lineare Abbildung ist, die das folgende Diagramm zum kommutieren bringt:
\[
  \begin{tikzcd}
      V^d
      \arrow{r}{f}
      \arrow[swap]{d}{\wedge_V}
    & W
    \\
      \exteriorpower{d}{V}
      \arrow[swap]{ru}{\induced{f}}
    & {}
  \end{tikzcd}
\]
Das Inverse $\Phi_{V,W}^{-1}$ ist durch
\[
          \Phi_{V, W}^{-1}
  \colon  \Hom\left( \exteriorpower{d}{V}, W \right)
  \to     \Alt( V^d, W ),
  \quad   g
  \to     g \circ \wedge_V,
\]
gegeben.

\begin{remark}
  Ist $g \colon V \to \tilde{V}$ eine lineare Abbildung, so erhalten wir eine induzierte Abbildung
  \[
            g^{\times d}
    \colon  V^d
    \to     \tilde{V}^d,
    \quad   (v_1, \dotsc, v_d)
    \mapsto (g(v_1), \dotsc, g(v_d)),
  \]
  und somit eine induzierte lineare Abbildung
  \[
            \left( g^{\times d} \right)^*
    \colon  \Alt( \tilde{V}^d, W )
    \to     \Alt( V^d, W ),
    \quad   f
    \mapsto f \circ g^{\times d}.
  \]
  Außerdem erhalen wir auch die induzierte lineare Abbildung
  \[
    \exteriorpower{d}{g}
    \colon  \exteriorpower{d}{V}
    \to     \exteriorpower{d}{\tilde{V}},
    \quad   v_1 \wedge \dotsb \wedge v_d
    \mapsto g(v_1) \wedge \dotsb \wedge g(v_d)
    \qquad
    \text{für alle $v_1, \dotsc, v_d \in V$},
  \]
  und somit eine induzierte lineare Abbildung
  \[
            \left( \exteriorpower{d}{g} \right)^*
    \colon  \Hom\left( \exteriorpower{d}{V}, W \right)
    \to     \Hom\left( \exteriorpower{d}{\tilde{V}}, W \right),
    \quad   f
    \mapsto f \circ \exteriorpower{d}{g}.
  \]

  Für jede lineare Abbildung $h \colon W \to \tilde{W}$ erhalten wir ebenfalls induzierte lineare Abbildungen
  \begin{gather*}
            h^{\Alt}_*
    \colon  \Alt( V^d, W )
    \to     \Alt( V^d, \tilde{W} ),
    \quad   f
    \mapsto h \circ f
  \shortintertext{und}
            h^{\Hom}_*
    \colon  \Hom\left( \exteriorpower{d}{V}, W \right)
    \to     \Hom\left( \exteriorpower{d}{V}, \tilde{W} \right),
    \quad   f
    \mapsto h \circ f.
  \end{gather*}
  
  Die Familie von Isomorphismen $(\Phi_{V,W})_{V,W}$ ist insofern mit diesen induzierten Abbildungen verträglich, als dass für alle $K$-Vektorräume $V$, $W$, $\tilde{V}$ und $\tilde{W}$ und linearen Abbildungen $g \colon V \to \tilde{V}$ und $h \colon W \to \tilde{W}$ die beiden Diagramme
  \begin{gather}
    \label{diagramm: induced map in the first entry}
    \begin{tikzcd}[sep = large, ampersand replacement = \&]
          \Alt(V^d, W)
          \arrow{r}{\Phi_{V,W}}
      \&  \Hom\left( \exteriorpower{d}{V}, W \right)
      \\
          \Alt( \tilde{V}^d, W )
          \arrow{u}{\left( g^{\times d} \right)^*}
          \arrow{r}{\Phi_{\tilde{V}, W}}
      \& \Hom\left( \exteriorpower{d}{\tilde{V}}, W \right)
          \arrow[swap]{u}{ \left( \exteriorpower{d}{g} \right)^* }
    \end{tikzcd}
  \shortintertext{und}
    \label{diagramm: induced map in the second entry}
    \begin{tikzcd}[sep = large, ampersand replacement = \&]
          \Alt(V^d, W)
          \arrow{r}{\Phi_{V,W}}
          \arrow[swap]{d}{h^{\Alt}_*}
      \&  \Hom\left( \exteriorpower{d}{V}, W \right)
          \arrow{d}{h^{\Hom}_*}
      \\
          \Alt( V^d, \tilde{W} )
          \arrow{r}{\Phi_{V,\tilde{W}}}
      \&  \Hom\left( \exteriorpower{d}{V}, \tilde{W} \right)
    \end{tikzcd}
  \end{gather}
  kommutieren:
  
  \begin{claim}
    Für alle $K$-Vektorräume $V$, $W$, $\tilde{V}$, $\tilde{W}$ und linearen Abbildungen $g \colon V \to \tilde{V}$ und $h \colon W \to \tilde{W}$ kommutieren die beiden Diagramme \eqref{diagramm: induced map in the first entry} und \eqref{diagramm: induced map in the second entry}.
  \end{claim}
  \begin{proof}
    Wir müssen zeigen, dass die Gleichheiten
    \[
        \left( \exteriorpower{d}{g} \right)^* \circ \Phi_{\tilde{V}, W}
      = \Phi_{V, W} \circ \left( g^{\times d} \right)^*
      \qquad\text{und}\qquad
        h^{\Hom}_* \circ \Phi_{V, W}
      = \Phi_{V, \tilde{W}} \circ h^{\Alt}_*
    \]
    gelten.
    Indem wir die erste Gleichung von links mit $\Phi_{V,W}^{-1}$ komponieren und von rechts mit $\Phi_{\tilde{V},W}$, sowie die zweite Gleichung von rechts mit $\Phi_{V,W}^{-1}$ und von links mit $\Phi_{V,\tilde{W}}^{-1}$, erhalten wir die jeweils äquivalenten Gleichungen
    \[
        \Phi_{V, W}^{-1} \circ \left( \exteriorpower{d}{g} \right)^* 
      = \left( g^{\times d} \right)^* \circ \Phi_{\tilde{V}, W}^{-1}
      \qquad\text{und}\qquad
        \Phi_{V, \tilde{W}}^{-1} \circ h^{\Hom}_*
      = h^{\Alt}_* \circ \Phi_{V, W}^{-1}.
    \]
    Die Kommutativität der Diagramme \eqref{diagramm: induced map in the first entry} und \eqref{diagramm: induced map in the second entry} ist deshalb äquivalent dazu, dass die beiden Diagramme
    \begin{gather}
      \label{diagramm: induced map in the first entry swapped}
      \begin{tikzcd}[sep = large, ampersand replacement = \&]
            \Alt(V^d, W)
        \&  \Hom\left( \exteriorpower{d}{V}, W \right)
            \arrow[swap]{l}{\Phi_{V,W}^{-1}}
        \\
            \Alt( \tilde{V}^d, W )
            \arrow{u}{\left( g^{\times d} \right)^*}
        \& \Hom\left( \exteriorpower{d}{\tilde{V}}, W \right)
            \arrow[swap]{l}{\Phi_{\tilde{V}, W}^{-1}}
            \arrow[swap]{u}{ \left( \exteriorpower{d}{g} \right)^* }
      \end{tikzcd}
    \shortintertext{und}
      \label{diagramm: induced map in the second entry swapped}
      \begin{tikzcd}[sep = large, ampersand replacement = \&]
            \Alt(V^d, W)
            \arrow[swap]{d}{h^{\Alt}_*}
        \&  \Hom\left( \exteriorpower{d}{V}, W \right)
            \arrow[swap]{l}{\Phi_{V,W}^{-1}}
            \arrow{d}{h^{\Hom}_*}
        \\
            \Alt( V^d, \tilde{W} )
        \&  \Hom\left( \exteriorpower{d}{V}, \tilde{W} \right)
            \arrow{l}{\Phi_{V,\tilde{W}}^{-1}}
      \end{tikzcd}
    \end{gather}
    kommutieren.
    (Die Kommutativät dieser abgeänderten Diagramme \eqref{diagramm: induced map in the first entry swapped} und \eqref{diagramm: induced map in the second entry swapped} ist leichter zu überprüfen als die Kommutativität der ursprünlichen Diagramme \eqref{diagramm: induced map in the first entry} und \eqref{diagramm: induced map in the second entry}, da etwa das Inverse $\Phi_{V,W}^{-1}$ eine einfachere Form hat als $\Phi_{V,W}$ selbst.)
    
    Die Kommutativität der Diagramme \eqref{diagramm: induced map in the first entry swapped} und \eqref{diagramm: induced map in the second entry swapped} lässt sich nun direkt nachrechnen:
  \end{proof}
  
  Wegen dieser Verträglichkeit mit den induzierten Abbildungen bezeichnet man die Familie von Isomorphismen $(\Phi_{V,W})_{V,W}$ auch als \emph{natürlich}.
  Abkürzend sagt man auch, dass für alle $K$-Vektorräume $V$ und $W$ der Isomorphismus $\Phi_{V,W}$ natürlich ist;
  dabei sollte man aber im Hinterkopf behalten, dass diese Natürlichkeiten eine Aussage über alle Isomorphismen gleichzeit ist, und nicht nur über einen einzelnen.
  
  Einer der Vorteile dieses Begriffes eines „natürlichen Isomorphismus“ ist, dass man ihn anstelle des schwammigen „kanonischen Isomorphismus“ verwenden kann.
  
  
\end{remark}














