\section{}





\subsection{}
Die Kommutativität des Diagrams
\[
  \begin{tikzcd}
      R
      \arrow{r}{\varphi}
      \arrow{d}{\pi}
    & S
    \\
      R/I
      \arrow[swap]{ru}{\induced{\varphi}}
    & {}
  \end{tikzcd}
\]
ist äquivalent dazu, dass $\induced{\varphi}(\class{x}) = \varphi(x)$ für alle $x \in R/I$.
Dies zeigt die Eindeutigkeit von $\induced{\varphi}$.

Zum Beweis der Existenz gilt es zu zeigen, dass durch
\[
          \induced{\varphi}
  \colon  R/I
  \to     S,
  \quad   \class{x}
  \mapsto \varphi(x)
\]
ein wohldefinierter Ringhomomorphismus gegeben ist:

Für $x, y \in R$ mit $\class{x} = \class{y}$ gilt $x - y \in \ker \varphi$ und somit $0 = \varphi(x-y) = \varphi(x) - \varphi(y)$, also $\varphi(x) = \varphi(y)$.
Dies zeigt die Wohldefiniertheit von $\induced{\varphi}$.
Dass $\induced{\varphi}$ ein Ringhomomorphismus ist, ergibt sich durch direktes Nachrechnen, denn für alle $\class{x}, \class{y} \in R/I$ gilt
\begin{gather*}
    \induced{\varphi}(\class{x} + \class{y})
  = \induced{\varphi}(\class{x + y})
  = \varphi(x + y)
  = \varphi(x) + \varphi(y)
  = \induced{\varphi}(\class{x}) + \induced{\varphi}(\class{y}).
\shortintertext{und}
    \induced{\varphi}(\class{x} \cdot \class{y})
  = \induced{\varphi}(\class{x \cdot y})
  = \varphi(x \cdot y)
  = \varphi(x) \cdot \varphi(y)
  = \induced{\varphi}(\class{x}) \cdot \induced{\varphi}(\class{y}),
\end{gather*}
und es gilt $\induced{\varphi}(1_{R/I}) = \induced{\varphi}(\class{1_R}) = \varphi(1_R) = 1_S$.





\subsection{}

Es gilt
\[
    \im \induced{\varphi}
  = \{ \induced{\varphi}(\class{x}) \mid \class{x} \in R/I \}
  = \{ \varphi(x) \mid x \in R \}
  = \im \varphi,
\]
also ist $\induced{\varphi}$ genau dann surjektiv, wenn $\varphi$ surjektiv ist.
Außerdem gilt
\begin{align*}
      \ker \induced{\varphi}
  &=  \{ \class{x} \in R/I \mid \induced{\varphi}( \class{x} ) = 0 \}
   =  \{ \class{x} \in R/I \mid \varphi(x) = 0 \}
  \\
  &=  \{ \class{x} \in R/I \mid x \in \ker \varphi \}
   =  \{ \class{x} \mid x \in \ker \varphi \}
   =  \{ x + I \mid x \in \ker \varphi \}
   =  (\ker \varphi)/I.
\end{align*}
Deshalb gilt
\[
        \text{$\induced{\varphi}$ ist injektiv}
  \iff  \ker \induced{\varphi} = \{0\}
  \iff  (\ker \varphi)/I = \{0\}
  \iff  \ker \varphi = I.
\]

Das Bild $\im \varphi$ ist ein kommutativer Unterring von $S$:

Es gilt $0 = \varphi(0) \in \im \varphi$.
Für $y \in \im \varphi$ gibt es $x \in R$ mit $y = \varphi(x)$, weshalb auch $-y = -\varphi(x) = \varphi(-x) \in \im \varphi$.
Für $y_1, y_2 \in \im \varphi$ gibt es $x_1, x_2 \in R$ mit $y_1 = \varphi(x_1)$ und $y_2 = \varphi(x_2)$, weshalb auch $y_1 + y_2 = \varphi(x_1) + \varphi(x_2) = \varphi(x_1 + x_2) \in \im \varphi$.
Das zeigt, dass $\im \varphi$ eine Untergruppe der additiven Gruppe von $S$ ist.

Es gilt $1_S = \varphi(1_R) \in \im \varphi$.
Für alle $y_1, y_2 \in \im \varphi$ gibt es $x_1, x_2 \in R$ mit $y_1 = \varphi(x_1)$ und $y_2 = \varphi(x_2)$, weshalb auch $y_1 y_2 = \varphi(x_1) \varphi(x_2) = \varphi(x_1 x_2) \in \im \varphi$.
Das zeigt, dass $\im \varphi$ bereits ein Unterring von $S$ ist.

Wir können nun $\varphi$ als einen surjektiven Ringhomorphismus $\varphi \colon R \to \im \varphi$ auffassen.
Aus den bereits gezeigten Aussagen erhalten wir, dass $\varphi$ einen bijektiven Ringhomomorphismus, also einen Ringisomorphismus $\induced{\varphi} \colon R/I \to \im \varphi$, $\class{x} \mapsto \varphi(x)$ induziert.
Somit gilt $R/I \cong \im \varphi$.


\begin{example}
  \begin{enumerate}
    \item
      Nach der universellen Eigenschaft des Polynomrings gibt es einen eindeutigen Homomorphismus von $\real$-Algebren $\varphi \colon \real[X] \to \complex$ mit $\varphi(X) = i$.
      Für alle $a, b \in \real$ gilt $a + ib = \varphi(a + bX) \in \im \varphi$, weshalb $\varphi$ surjektiv.
      
      \begin{claim}
        Es gilt $\ker \varphi = (X^2 + 1) = \{f \cdot (X^2 + 1) \mid f \in \real[X]\}$.
      \end{claim}
      
      Damit ergibt sich, dass $\varphi$ einen Ringisomorphismus $\induced{\varphi} \colon \real[X]/(X^2 + 1) \to \complex$, $f \mapsto f(i)$ induziert.
      Anschaulich bedeutet dies, dass $\complex$ aus $\real$ durch hinzufügen eines Elements $X$ mit $X^2 + 1$ entsteht.
      
    \item
      Es sei
      \[
                  C
        \coloneqq \{
                    (a_n)_{n \in \natural}
                  \mid
                    \text{$a_n \in \rational$ für alle $n \in \natural$},
                    \text{$(a_n)_{n \in \natural}$ ist eine Cauchyfolge}
                  \}
      \]
      der Raum der rationalen Cauchyfolgen.
      Sind $(a_n)_{n \in \natural}, (b_n)_{n \in \natural} \in C$ zwei rationale Cauchyfolgen, so sind auch $(a_n + b_n)_{n \in \natural}$ und $(a_n \cdot b_n)_{n \in \natural}$ rationale Cauchyfolgen.
      Zusammen mit dieser Addition und Multiplikation bildet $C$ einen kommutativen Ring;
      das Einselement ist durch die konstante $1$-Folge $(1)_{n \in \natural}$ gegeben.
      
      Da jede rationale Cauchyfolge in $\real$ kovergiert, ergibt es eine wohldefinierte Abbildung
      \[
                \lim
        \colon  C
        \to     \real,
        \quad   (a_n)_{n \in \natural}
        \mapsto \lim_{n \to \infty} a_n.
      \]
      Die Abbildung $\lim$ ist ein Ringhomomorphismus, denn für alle $(a_n)_{n \in \natural}, (b_n)_{n \in \natural} \in C$ gilt
      \begin{gather*}
        \begin{aligned}
              \lim( (a_n)_{n \in \natural} + (b_n)_{n \in \natural} )
          &=  \lim( (a_n + b_n)_{n \in \natural} )
          =  \lim_{n \to \infty} (a_n + b_n)
        \\
          &=  (\lim_{n \to \infty} a_n) + (\lim_{n \to \infty} b_n)
          =  \lim( (a_n)_{n \in \natural} ) + \lim( (b_n)_{n \in \natural} ).
        \end{aligned}
      \shortintertext{und}
        \begin{aligned}
              \lim( (a_n)_{n \in \natural} \cdot (b_n)_{n \in \natural} )
          &=  \lim( (a_n b_n)_{n \in \natural} )
          =  \lim_{n \to \infty} (a_n b_n)
        \\
          &=  (\lim_{n \to \infty} a_n) \cdot (\lim_{n \to \infty} b_n)
          =  \lim( (a_n)_{n \in \natural} ) \cdot \lim( (b_n)_{n \in \natural} ).
        \end{aligned}
      \end{gather*}
      Da jede reelle Zahle als Grenzwert einer rationalen Cauchyfolge geschrieben werden kann (da $\rational$ als Teilmenge von $\real$ dicht ist) ist $\lim$ surjektiv.
      Dabei gilt
      \[
          \ker \lim
        = \{ (a_n)_{n \in \natural} \mid \lim( (a_n)_{n \in \natural} ) = 0 \}
        = \{ (a_n)_{n \in \natural} \mid \lim_{n \to \infty} a_n = 0 \},
      \]
      d.h.\ der Kern von $\lim$ besteht aus den rationalen Cauchyfolgen, die auch Nullfolgen sind.
      Da aber jede Nullfolge bereits eine Cauchyfolge ist, ist der Kern von $\lim$ durch
      \[
                  N
        \coloneqq \{
                    (a_n)_{n \in \natural}
                  \mid
                    \text{$a_n \in \rational$ für alle $n \in \natural$},
                    \text{$a_n \to 0$ für $n \to \infty$}
                  \}
      \]
      gegeben.
      Somit ist $N$ ein Ideal in $C$, und $\lim$ induziert einen Ringisomorphismus
      \[
                C/N
        \to     \real,
        \quad   \class{ (a_n)_{n \in \natural} }
        \mapsto \lim_{n \to \infty} a_n.
      \]
      
      Dies führt dazu, dass sich die reellen Zahlen als Äquivalenzklassen von rationalen Cauchyfolgen konstruieren lassen, wobei zwei rationale Cauchyfolgen $(a_n)_{n \in \natural}$ und $(b_n)_{n \in \natural}$ genau dann äquivalent sind, wenn $(a_n)_{n \in \natural} - (b_n)_{n \in \natural}$ eine Nullfolge ist.
  \end{enumerate}
\end{example}









