\section{}





\subsection{}
Die Kommutativität des Diagrams
\[
  \begin{tikzcd}
      R
      \arrow{r}{\varphi}
      \arrow{d}{\pi}
    & S
    \\
      R/I
      \arrow[swap]{ru}{\overline{\varphi}}
    & {}
  \end{tikzcd}
\]
ist äquivalent dazu, dass $\overline{\varphi}(\class{x}) = \varphi(x)$ für alle $x \in R/I$.
Dies zeigt die Eindeutigkeit von $\overline{\varphi}$.

Zum Beweis der Existenz gilt es zu zeigen, dass durch
\[
          \overline{\varphi}
  \colon  R/I
  \to     S,
  \quad   \class{x}
  \mapsto \varphi(x)
\]
ein wohldefinierter Ringhomomorphismus gegeben ist:

Für $x, y \in R$ mit $\class{x} = \class{y}$ gilt $x - y \in \ker \varphi$ und somit $0 = \varphi(x-y) = \varphi(x) - \varphi(y)$, also $\varphi(x) = \varphi(y)$.
Dies zeigt die Wohldefiniertheit von $\overline{\varphi}$.
Dass $\overline{\varphi}$ ein Ringhomomorphismus ist, ergibt sich durch direktes Nachrechnen, denn für alle $\overline{x}, \overline{y} \in R/I$ gilt
\begin{gather*}
    \overline{\varphi}(\class{x} + \class{y})
  = \overline{\varphi}(\class{x + y})
  = \varphi(x + y)
  = \varphi(x) + \varphi(y)
  = \overline{\varphi}(\class{x}) + \overline{\varphi}(\class{y}).
\shortintertext{und}
    \overline{\varphi}(\class{x} \cdot \class{y})
  = \overline{\varphi}(\class{x \cdot y})
  = \varphi(x \cdot y)
  = \varphi(x) \cdot \varphi(y)
  = \overline{\varphi}(\class{x}) \cdot \overline{\varphi}(\class{y}),
\end{gather*}
und es gilt $\overline{\varphi}(1_{R/I}) = \overline{\varphi}(\overline{1_R}) = \varphi(1_R) = 1_S$.





\subsection{}

Es gilt
\[
    \im \overline{\varphi}
  = \{ \overline{\varphi}(\overline{x}) \mid \overline{x} \in R/I \}
  = \{ \varphi(x) \mid x \in R \}
  = \im \varphi,
\]
also ist $\overline{\varphi}$ genau dann surjektiv, wenn $\varphi$ surjektiv ist.
Außerdem gilt
\begin{align*}
      \ker \overline{\varphi}
  &=  \{ \class{x} \in R/I \mid \overline{\varphi}( \class{x} ) = 0 \}
   =  \{ \class{x} \in R/I \mid \varphi(x) = 0 \}
  \\
  &=  \{ \class{x} \in R/I \mid x \in \ker \varphi \}
   =  \{ \class{x} \mid x \in \ker \varphi \}
   =  \{ x + I \mid x \in \ker \varphi \}
   =  (\ker \varphi)/I.
\end{align*}
Deshalb gilt
\[
        \text{$\overline{\varphi}$ ist injektiv}
  \iff  \ker \overline{\varphi} = \{0\}
  \iff  (\ker \varphi)/I = \{0\}
  \iff  \ker \varphi = I.
\]

Das Bild $\im \varphi$ ist ein kommutativer Unterring von $S$:

Es gilt $0 = \varphi(0) \in \im \varphi$.
Für $y \in \im \varphi$ gibt es $x \in R$ mit $y = \varphi(x)$, weshalb auch $-y = -\varphi(x) = \varphi(-x) \in \im \varphi$.
Für $y_1, y_2 \in \im \varphi$ gibt es $x_1, x_2 \in R$ mit $y_1 = \varphi(x_1)$ und $y_2 = \varphi(x_2)$, weshalb auch $y_1 + y_2 = \varphi(x_1) + \varphi(x_2) = \varphi(x_1 + x_2) \in \im \varphi$.
Das zeigt, dass $\im \varphi$ eine Untergruppe der additiven Gruppe von $S$ ist.

Es gilt $1_S = \varphi(1_R) \in \im \varphi$.
Für alle $y_1, y_2 \in \im \varphi$ gibt es $x_1, x_2 \in R$ mit $y_1 = \varphi(x_1)$ und $y_2 = \varphi(x_2)$, weshalb auch $y_1 y_2 = \varphi(x_1) \varphi(x_2) = \varphi(x_1 x_2) \in \im \varphi$.
Das zeigt, dass $\im \varphi$ bereits ein Unterring von $S$ ist.

Wir können nun $\varphi$ als einen surjektiven Ringhomorphismus $\varphi \colon R \to \im \varphi$ auffassen.
Aus den bereits gezeigten Aussagen erhalten wir, dass $\varphi$ einen bijektiven Ringhomomorphismus, also einen Ringisomorphismus $\overline{\varphi} \colon R/I \to \im \varphi$, $\class{x} \mapsto \varphi(x)$ induziert.
Somit gilt $R/I \cong \im \varphi$.









