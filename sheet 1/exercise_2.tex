\section{}
Es sei $R$ ein Integritätsbereich.

\begin{remark}
  Der Quotientenkörpers des Integritätsbereichs der ganzen Zahlen $\integer$ ist der Körper der rationalen Zahlen $\rational$, d.h.\ $\fraction{\integer} = \rational$.
\end{remark}

\begin{remark}
  Die Abbildung $i \colon R \to \fraction{R}$, $r \mapsto r/1$ ist ein injektiver Ringhomomorphismus:
  Es handelt sich um einen Ringhomomorphismus, denn für alle $r_1, r_2 \in R$ gilt
  \begin{gather*}
      i(r_1) + i(r_2)
    = \frac{r_1}{1} + \frac{r_2}{1}
    = \frac{r_1 + r_2}{1}
    = i(r_1 + r_2)
  \shortintertext{und}
      i(r_1) \cdot i(r_2)
    = \frac{r_1}{1} \cdot \frac{r_2}{1}
    = \frac{r_1 r_2}{1}
    = i(r_1 r_2),
  \end{gather*}
  und es gilt $i(1_R) = 1_R/1_R = 1_{\fraction{R}}$.
  Ist $r \in \ker i$, so gilt $r/0 = 0/1$ und somit $r \cdot 1 = 0 \cdot 0$, also $r = 0$.
  Deshalb ist $\ker i = \{0\}$ und $i$ somit injektiv.

  Da $i$ ein injektiver Ringhomomorphismus ist, lässt sich $i$ durch Einschränkung als ein Ringisomorphismus $R \to \im i$ auffassen, d.h.\ $\im i$ ist ein zu $R$ isomorpher Unterring von $\fraction{R}$, und ein entsprechender Isomorphismus ist durch $r \mapsto r/1$ gegeben.
\end{remark}

Anschaulich gesehen ist $\fraction{R}$ der „kleinstmögliche“ Körper, der $R$ enthält.
Dies lässt sich durch die \emph{universelle Eigenschaft des Quotientenkörpers} formalisieren:

Ist $K$ ein beliebiger Körper und $j \colon R \to K$ ein injektiver Ringhomomorphismus, so gibt es einen eindeutigen Ringhomomorphismus $\induced{j} \colon \fraction{R} \to K$, der das folgende Diagramm zum Kommutieren bringt:
\begin{equation}
  \label{equation: commutative diagram for the universal property of the quotient field}
  \begin{tikzcd}
      {}
    & R
      \arrow[swap]{dl}{i}
      \arrow{dr}{j}
    & {}
    \\
      \fraction{R}
      \arrow{rr}{\induced{j}}
    & {}
    & K
  \end{tikzcd}
\end{equation}

\begin{proof}
  Für alle $r/s \in \fraction{R}$ muss
  \[
      \induced{j}\left( \frac{r}{s} \right)
    = \induced{j}\left( \frac{r}{1} \left(\frac{s}{1}\right)^{-1} \right)
    = \induced{j}( i(r) i(s)^{-1} )
    = \induced{j}(i(r)) \induced{j}(i(s))^{-1}
    = j(r) j(s)^{-1}
    = \frac{j(r)}{j(s)}
  \]
  gelten, was die eindeutig von $\induced{j}$ zeigt.
  (Hier nutzen wir, dass $\induced{j}(x^{-1}) = \induced{j}(x)^{-1}$ für alle $x \in \fraction{R}$ mit $x \neq 0$ gelten muss, da $1_K = \induced{j}(1_{\fraction{R}}) = \induced{j}(x x^{-1}) = \induced{j}(x) \induced{j}(x^{-1})$ gilt.)
  
  Andererseits definiert
  \[
            \induced{j}
    \colon  \fraction{R}
    \to     K,
    \quad   \frac{r}{s}
    \mapsto \frac{j(r)}{j(s)}
  \]
  einen wohldefinierten Ringhomomorphismus:
  
  Für $r/s = r'/s' \in \fraction{R}$ gilt $r s' = r' s$ und somit auch
  \begin{equation}
    \label{equation: not reduced equality}
      j(r) j(s')
    = j(r s')
    = j(r' s)
    = j(r') j(s).
  \end{equation}
  Wegen der Injektivität von $j$ folgt aus $s, s' \neq 0$, dass auch $j(s), j(s') \neq 0$ gilt.
  Die Gleichung \eqref{equation: not reduced equality} lässt sich deshalb durch $j(s)$ und durch $j(s')$ teilen, wodurch sich $j(r)/j(s) = j(r')/j(s')$ ergibt.
  Dies zeigt die Wohldefiniertheit von $\induced{j}$.
  
  Dass $\induced{j}$ ein Ringhomomorphismus ist, ergibt sich durch direktes Nachrechnen, denn für alle $r_1/s_1, r_2/s_2 \in \fraction{R}$ gilt
  \begin{gather*}
    \begin{aligned}
          \induced{j}\left( \frac{r_1}{s_1} \right) + \induced{j}\left( \frac{r_2}{s_2} \right)
      &=  \frac{j(r_1)}{j(s_1)} + \frac{j(r_2)}{j(s_2)}
      =  \frac{j(r_1) j(s_2) + j(r_2) j(s_1)}{j(s_1) j(s_2)}
      \\
      &=  \frac{j(r_1 s_2 + r_2 s_1)}{j(s_1 s_2)}
      =  \induced{j}\left( \frac{r_1 s_2 + r_2 s_1}{s_1 s_2} \right)
      =  \induced{j}\left( \frac{r_1}{s_1} + \frac{r_2}{s_2} \right).
    \end{aligned}
    \shortintertext{sowie}
    \begin{aligned}
          \induced{j}\left( \frac{r_1}{s_1} \right) \cdot \induced{j}\left( \frac{r_2}{s_2} \right)
       =  \frac{j(r_1)}{j(s_1)} \cdot \frac{j(r_2)}{j(s_2)}
       =  \frac{j(r_1) j(r_2)}{j(s_1) j(s_2)}
       =  \frac{j(r_1 r_2)}{j(s_1 s_2)}
      &=  \induced{j}\left( \frac{r_1 r_2}{s_1 s_2} \right)
      \\
      &=  \induced{j}\left( \frac{r_1}{s_1} \cdot \frac{r_2}{s_2} \right),
    \end{aligned}
  \end{gather*}
  und es gilt $\induced{j}(1_{\fraction{R}}) = \induced{j}(1_R / 1_R) = j(1_R) / j(1_R) = 1_K / 1_K = 1_K$.
\end{proof}

\begin{remark}
  Ringhomomorphismen zwischen zwei Körpern sind stets injektiv:
  Es seien $K$ und $L$ zwei Körper.
  Gebe es einen nicht-injektiven Ringhomomorphismus $\varphi \colon K \to L$, so würde $\ker \varphi \neq 0$ gelten.
  Dann gebe es ein $x \in K$ mit $x \neq 0$ und $\varphi(x) = 0$.
  Dann würde auch
  \[
      0
    = 0 \cdot \varphi(x^{-1})
    = \varphi(x) \cdot \varphi(x^{-1})
    = \varphi(x \cdot x^{-1})
    = \varphi(1)
    = 1,
  \]
  gelten, was aber in Körpern per Definition nicht gilt.
  Folglich muss $\ker \varphi = 0$ gelten, und $\varphi$ somit injektiv sein.
  
  Inbesondere erhalten wir in dem kommutativen Diagram \eqref{equation: commutative diagram for the universal property of the quotient field}, dass auch der Ringhomomorphismus $\induced{j}$ injektiv ist, und somit einen Isomorphismus von Körpern $\fraction{R} \to \im \induced{j}$ induziert.
  Dies entspricht der Anschauung, dass jeder Körper, der den Integritätsbereich $R$ enthält, auch schon den Quotientenkörper $\fraction{R}$ enthalten muss.
  
  So muss etwa jeder Körper, der die ganzen Zahlen $\integer$ enthält, auch schon die rationalen Zahlen $\fraction{\integer} = \rational$ enthalten.
\end{remark}


\begin{remark}
  Auf die übliche Weise ergibt sich, dass das Paar $(\fraction{R}, i)$ durch die obige universelle Eigenschaft bis auf eindeutigen Isomorphismus eindeutig bestimmt ist:
  
  Es sei $(Q', i')$ ein weiteres Paar, bestehend aus einem Körper $Q'$ und einem injektiven Ringhomomorphismus $i' \colon R \to Q'$, so dass es für jeden Körper $K$ und jeden injektiven Ringhomomorphismus $j \colon R \to K$ einen eindeutigen Ringhomomorphismus $\induced{j} \colon Q' \to K$ gibt, so dass das folgende Diagramm kommutiert:
  \begin{equation}
    \begin{tikzcd}
        {}
      & R
        \arrow[swap]{dl}{i'}
        \arrow{dr}{j}
      & {}
      \\
        Q'
        \arrow{rr}{\induced{j}}
      & {}
      & K
    \end{tikzcd}
  \end{equation}
  Dann gibt es einen eindeutigen Ringhomomorphismus $\varphi \colon \fraction{R} \to Q'$, der das Diagramm
  \begin{equation}
    \label{equation: first morphism from uniqueness}
    \begin{tikzcd}
        {}
      & R
        \arrow[swap]{dl}{i}
        \arrow{dr}{i'}
      & {}
      \\
        \fraction{R}
        \arrow{rr}{\varphi}
      & {}
      & Q'
    \end{tikzcd}
  \end{equation}
  zum Kommutieren bringt, und $\varphi$ ist ein Isomorphismus:
  
  Die Existenz und Eindeutigkeit von $\varphi$ ergeben sich dadurch, dass man die universelle Eigenschaft des Paars $(\fraction{R}, i)$ auf den injektiven Ringhomomorphismus $i' \colon R \to Q'$ anwendet.
  Analog ergibt sich Anwenden der analogen Eigenschaft von $(Q', i')$ auf den injektiven Ringhomomorphismus $i \colon R \to \fraction{Q}$, dass es einen eindeutigen Ringhomomorphismus $\psi \colon Q' \to \fraction{R}$ gibt, so dass das Diagramm
  \begin{equation}
    \label{equation: second morphism from uniqueness}
    \begin{tikzcd}
        {}
      & R
        \arrow[swap]{dl}{i}
        \arrow{dr}{i'}
      & {}
      \\
        \fraction{R}
      & {}
      & Q'
        \arrow{ll}{\psi}
    \end{tikzcd}
  \end{equation}
  kommutiert.
  Durch Zusammenfügen von \eqref{equation: first morphism from uniqueness} und \eqref{equation: second morphism from uniqueness} ergibt sich das folgende kommutative Diagramm:
  \begin{equation}
    \begin{tikzcd}
        {}
      & R
        \arrow[swap]{dl}{i}
        \arrow{d}{i'}
        \arrow{dr}{i}
      & {}
      \\
        \fraction{R}
        \arrow{r}{\varphi}
      & Q'
        \arrow{r}{\psi}
      & \fraction{R}
    \end{tikzcd}
  \end{equation}
  Hieraus ergibt sich durch Vergessen des mittleren vertikalen Pfeils das folgende kommutative Diagramm:
  \begin{equation}
    \label{equation: third morphism from uniqueness}
    \begin{tikzcd}
        {}
      & R
        \arrow[swap]{dl}{i}
        \arrow{dr}{i}
      & {}
      \\
        \fraction{R}
        \arrow{rr}{\psi \circ \varphi}
      & {}
      & \fraction{R}
    \end{tikzcd}
  \end{equation}
  Nach der universellen Eigenschaft von $(\fraction{R}, i)$ ist $\psi \circ \varphi$ dabei bereits der \emph{eindeutige} Ringhomomorphismus $\fraction{R} \to \fraction{R}$, der das Diagramm \eqref{equation: third morphism from uniqueness} zum Kommutieren bringt.
  Andererseits bringt auch $\id_{\fraction{R}} \colon \fraction{R} \to \fraction{R}$ das Diagramm zum Kommutieren.
  Somit muss bereits $\psi \circ \varphi = \id_{\fraction{R}}$ gelten.
  Analog ergibt sich, dass auch $\varphi \circ \psi = \id_{Q'}$ gilt.
  Also ist $\varphi$ ein Isomorphismus mit $\varphi^{-1} = \psi$.
\end{remark}
























