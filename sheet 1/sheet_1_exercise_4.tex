\section{}





\subsection{}
Die Kommutativität des Diagrams
\[
  \begin{tikzcd}
      R
      \arrow{r}{\varphi}
      \arrow[swap]{d}{\pi}
    & S
    \\
      R/I
      \arrow[swap]{ru}{\induced{\varphi}}
    & {}
  \end{tikzcd}
\]
ist äquivalent dazu, dass $\induced{\varphi}(\class{x}) = \varphi(x)$ für alle $x \in R/I$ gilt.
Dies zeigt bereits die Eindeutigkeit von $\induced{\varphi}$.

Zum Beweis der Existenz von $\induced{\varphi}$ gilt es zu zeigen, dass durch
\[
          \induced{\varphi}
  \colon  R/I
  \to     S,
  \quad   \class{x}
  \mapsto \varphi(x)
\]
ein wohldefinierter Ringhomomorphismus gegeben ist:

Für $x, y \in R$ mit $\class{x} = \class{y}$ gilt $x - y \in \ker \varphi$ und somit $0 = \varphi(x-y) = \varphi(x) - \varphi(y)$, also $\varphi(x) = \varphi(y)$.
Dies zeigt die Wohldefiniertheit von $\induced{\varphi}$.
Durch direktes Nachrechnen ergibt sich, dass $\induced{\varphi}$ ein Ringhomomorphismus ist, denn für alle $\class{x}, \class{y} \in R/I$ gilt
\begin{gather*}
    \induced{\varphi}(\class{x} + \class{y})
  = \induced{\varphi}(\class{x + y})
  = \varphi(x + y)
  = \varphi(x) + \varphi(y)
  = \induced{\varphi}(\class{x}) + \induced{\varphi}(\class{y}).
\shortintertext{und}
    \induced{\varphi}(\class{x} \cdot \class{y})
  = \induced{\varphi}(\class{x \cdot y})
  = \varphi(x \cdot y)
  = \varphi(x) \cdot \varphi(y)
  = \induced{\varphi}(\class{x}) \cdot \induced{\varphi}(\class{y}),
\end{gather*}
und es gilt $\induced{\varphi}(1_{R/I}) = \induced{\varphi}(\class{1_R}) = \varphi(1_R) = 1_S$.

\begin{remark}
  Ist $R$ ein nicht-kommutativer Ring und $I \subseteq R$ ein beidseitiges Ideal, so gilt der Homomorphiesatz in gleicher Form und mit unveränderten Beweis.
\end{remark}





\subsection{}

Es gilt
\[
    \im \induced{\varphi}
  = \{ \induced{\varphi}(\class{x}) \suchthat \class{x} \in R/I \}
  = \{ \varphi(x) \suchthat x \in R \}
  = \im \varphi.
\]
Insbesondere ist $\induced{\varphi}$ genau dann surjektiv, wenn $\varphi$ surjektiv ist.
Außerdem gilt
\begin{align*}
      \ker \induced{\varphi}
  &=  \{ \class{x} \in R/I \suchthat \induced{\varphi}( \class{x} ) = 0 \}
   =  \{ \class{x} \in R/I \suchthat \varphi(x) = 0 \}
  \\
  &=  \{ \class{x} \in R/I \suchthat x \in \ker \varphi \}
   =  \{ \class{x} \suchthat x \in \ker \varphi \}
   =  \{ x + I \suchthat x \in \ker \varphi \}
   =  (\ker \varphi)/I.
\end{align*}
Deshalb gilt
\[
        \text{$\induced{\varphi}$ ist injektiv}
  \iff  \ker \induced{\varphi} = \{0\}
  \iff  (\ker \varphi)/I = \{0\}
  \iff  \ker \varphi = I.
\]

\begin{claim}
  Das Bild $\im \varphi$ ist ein kommutativer Unterring von $S$.
\end{claim}
\begin{proof}
  Es gilt
  \[
        0
    =   \varphi(0)
    \in \im \varphi.
  \]
  Für $y \in \im \varphi$ gibt es $x \in R$ mit $y = \varphi(x)$, weshalb auch
  \[
        -y
    =   -\varphi(x)
    =   \varphi(-x)
    \in \im \varphi
  \]
  gilt
  Für $y_1, y_2 \in \im \varphi$ gibt es $x_1, x_2 \in R$ mit $y_1 = \varphi(x_1)$ und $y_2 = \varphi(x_2)$, weshalb auch
  \[
        y_1 + y_2
    =   \varphi(x_1) + \varphi(x_2)
    =   \varphi(x_1 + x_2)
    \in \im \varphi
  \]
  gilt.
  Das zeigt, dass $\im \varphi$ eine Untergruppe der additiven Gruppe von $S$ ist.

  Es gilt
  \[
    1_S = \varphi(1_R) \in \im \varphi.
  \]
  Für alle $y_1, y_2 \in \im \varphi$ gibt es $x_1, x_2 \in R$ mit $y_1 = \varphi(x_1)$ und $y_2 = \varphi(x_2)$, weshalb auch
  \[
        y_1 y_2
    =   \varphi(x_1) \varphi(x_2)
    =   \varphi(x_1 x_2)
    \in \im \varphi
  \]
  gilt
  Das zeigt, dass $\im \varphi$ bereits ein Unterring von $S$ ist.
  
  Für $y_1, y_2 \in \im \varphi$ gibt es $x_1, x_2 \in R$ mit $y_1 = \varphi(x_1)$ und $y_2 = \varphi(x_2)$.
  Aus der Kommutativität von $\varphi$ ergibt sich dann, dass
  \[
      y_1 y_2
    = \varphi(x_1) \varphi(x_2)
    = \varphi(x_1 x_2)
    = \varphi(x_2 x_1)
    = \varphi(x_2) \varphi(x_1)
    = y_2 y_1.
  \]
  Also ist auch $\im \varphi$ kommutativ.
\end{proof}

Wir können nun $\varphi$ als einen surjektiven Ringhomorphismus $R \to \im \varphi$, $r \mapsto \varphi(r)$ auffassen.
Aus den bereits gezeigten Aussagen erhalten wir, dass $\varphi$ einen bijektiven Ringhomomorphismus, also einen Ringisomorphismus $\induced{\varphi} \colon R/I \to \im \varphi$, $\class{x} \mapsto \varphi(x)$ induziert.
Somit gilt $R/I \cong \im \varphi$.

\begin{remark}
  Ist $R$ ein nicht-kommutativer Ring und $I \subseteq R$ ein Ideal, bzw.\ $\varphi \colon R \to S$ ein Ringhomomorphismus, so gelten die Aussagen unverändert mit gleichen Beweis.
  (Der einzige Unterschied besteht darin, dass $\im \varphi$ nicht mehr notwendigerweise kommutativ ist.)
\end{remark}

\begin{example}
  \begin{enumerate}
    \item
      Nach der universellen Eigenschaft des Polynomrings gibt es einen eindeutigen Homomorphismus von $\real$-Algebren $\varphi \colon \real[X] \to \complex$ mit $\varphi(X) = i$;
      konkret ist $\varphi$ durch $\varphi(p) = p(i)$ für alle $p \in \real[X]$ gegeben.
      Für alle $a, b \in \real$ gilt $a + ib = \varphi(a + bX) \in \im \varphi$, weshalb $\varphi$ surjektiv ist.
      
      \begin{claim}
        Es gilt $\ker \varphi = (X^2 + 1) = \{f \cdot (X^2 + 1) \suchthat f \in \real[X]\}$.
      \end{claim}
      \begin{proof}
        Für $p \in (X^2 + 1)$ gibt es $f \in \real[X]$ mit $p = f \cdot (X^2 + 1)$.
        Dann gilt
        \[
            \varphi(p)
          = p(i)
          = f(i) \cdot (i^2 + 1)
          = f(i) \cdot 0
          = 0,
        \]
        und somit $p \in \ker \varphi$.
        
        Gilt andererseits $p \in \ker \varphi$, so ist $p(i) = \varphi(p) = 0$.
        Durch Polynomdivison ergeben sich $f, r \in K[X]$ mit $p = f \cdot (X^2 + 1) + r$ und $\deg r < \deg(X^2 + 1) = 2$.
        Also ist $r$ von der Form $r = a + bX$ mit $a, b \in \real$.
        Wir erhalten, dass
        \[
            0
          = p(i)
          = f(i) \cdot \underbrace{(i^2 + 1)}_{=0} + r(i)
          = r(i)
          = a + bi.
        \]
        Es folgt, dass $a = b = 0$ gilt, und somit $r = 0$.
        Also ist bereits $p = f \cdot (X^2 + 1) \in (X^2 + 1)$.
      \end{proof}
      
      Damit ergibt sich, dass $\varphi$ einen Ringisomorphismus
      \[
                \induced{\varphi}
        \colon  \real[X]/(X^2 + 1)
        \to     \complex,
        \quad   p
        \mapsto p(i)
      \]
      induziert.
      Dieser Isomorphismus kann so verstanden werden, dass $\complex$ aus $\real$ durch hinzufügen eines Elements $\class{X}$ mit $\class{X}{}^2 + 1 = 0$ entsteht, wobei $\class{X}$ der üblichen komplexen Zahl $i$ entspricht.
      
    \item
      Es sei
      \[
                  \cauchy
        \coloneqq \left\{
                    (a_n)_{n \in \mathbb{N}}
                  \suchthatscale
                    \text{$a_n \in \rational$ für alle $n \in \naturals$},
                    \text{$(a_n)_{n \in \naturals}$ ist eine Cauchyfolge}
                  \right\}
      \]
      der Raum der rationalen Cauchyfolgen.
      Sind $(a_n)_{n \in \naturals}, (b_n)_{n \in \naturals} \in \cauchy$ zwei rationale Cauchyfolgen, so sind auch $(a_n + b_n)_{n \in \naturals}$ und $(a_n \cdot b_n)_{n \in \naturals}$ rationale Cauchyfolgen.
      Zusammen mit dieser Addition und Multiplikation bildet $\cauchy$ einen kommutativen Ring;
      das Einselement ist durch die konstante $1$-Folge $(1)_{n \in \naturals}$ gegeben ($\cauchy$ enthält alle konstanten Folgen, da diese insbesondere konvergent und somit Cauchy sind).
      
      Da jede rationale Cauchyfolge in $\real$ kovergiert, ergibt sich eine wohldefinierte Abbildung
      \[
                \lim
        \colon  \cauchy
        \to     \real,
        \quad   (a_n)_{n \in \naturals}
        \mapsto \lim_{n \to \infty} a_n.
      \]
      Die Abbildung $\lim$ ist ein Ringhomomorphismus, denn für alle $(a_n)_{n \in \naturals}, (b_n)_{n \in \naturals} \in \cauchy$ gilt
      \begin{align*}
            \lim( (a_n)_{n \in \naturals} + (b_n)_{n \in \naturals} )
        &=  \lim( (a_n + b_n)_{n \in \naturals} )
          =  \lim_{n \to \infty} (a_n + b_n)
        \\
        &=  \left( \lim_{n \to \infty} a_n \right) + \left( \lim_{n \to \infty} b_n \right)
          =  \lim( (a_n)_{n \in \naturals} ) + \lim( (b_n)_{n \in \naturals} ).
      \end{align*}
      und
      \begin{align*}
            \lim( (a_n)_{n \in \naturals} \cdot (b_n)_{n \in \naturals} )
        &=  \lim( (a_n b_n)_{n \in \naturals} )
        =  \lim_{n \to \infty} (a_n b_n)
        \\
        &=  \left( \lim_{n \to \infty} a_n \right) \cdot \left( \lim_{n \to \infty} b_n \right)
        =  \lim( (a_n)_{n \in \naturals} ) \cdot \lim( (b_n)_{n \in \naturals} ).
      \end{align*}
      Da jede reelle Zahl als Grenzwert einer rationalen Cauchyfolge geschrieben werden kann (dies ist gerade die Dichtheit von $\rational$ in $\real$ ), ist $\lim$ surjektiv.
      Außerdem gilt
      \[
          \ker \lim
        = \{
            (a_n)_{n \in \naturals} \in \cauchy
          \suchthat
            \lim( (a_n)_{n \in \naturals} ) = 0
          \}
        = \left\{
            (a_n)_{n \in \naturals} \in \cauchy
          \suchthatscale
            \lim_{n \to \infty} a_n = 0
          \right\},
      \]
      d.h.\ der Kern von $\lim$ besteht aus genau den rationalen Cauchyfolgen, die auch Nullfolgen sind.
      Da aber jede Nullfolge bereits eine Cauchyfolge ist, ist der Kern von $\lim$ durch
      \[
                  N
        \coloneqq \{
                    (a_n)_{n \in \naturals}
                  \suchthat
                    \text{$a_n \in \rational$ für alle $n \in \naturals$},
                    \text{$a_n \to 0$ für $n \to \infty$}
                  \}
      \]
      gegeben.
      Somit ist $N$ ein Ideal in $\cauchy$, und $\lim$ induziert einen Isomorphismus von Ringen
      \[
                \cauchy/N
        \to     \real,
        \quad   \class{ (a_n)_{n \in \naturals} }
        \mapsto \lim_{n \to \infty} a_n.
      \]
      
      Dies führt dazu, dass sich die reellen Zahlen als Äquivalenzklassen von rationalen Cauchyfolgen konstruieren lassen, wobei zwei rationale Cauchyfolgen $(a_n)_{n \in \naturals}$ und $(b_n)_{n \in \naturals}$ genau dann äquivalent sind, wenn $(a_n)_{n \in \naturals} - (b_n)_{n \in \naturals}$ eine Nullfolge ist.
  \end{enumerate}
\end{example}









