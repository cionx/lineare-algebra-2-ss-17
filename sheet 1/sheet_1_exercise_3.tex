\section{}





\subsection{}
Da $I$ eine Untergruppe der unterliegenden additiven Gruppe von $R$ ist, ist aus der Linearen Algebra I bekannt, dass
\begin{itemize}
  \item
    $\sim$ eine Äquivalenzrelation auf $R$ definiert,
  \item
    durch $\class{x} + \class{y} \coloneqq \class{x+y}$ eine wohldefinierte binäre Verknüpfung von $R/I$ definiert wird,
  \item
    $R/I$ durch $+$ zu einer abelschen Gruppe wird.
\end{itemize}
Es bleibt zu zeigen, dass
\begin{itemize}
  \item
    durch $\class{x} \cdot \class{y} \coloneqq \class{xy}$ eine wohldefinierte binäre Verknüpfüng auf $R/I$ definiert wird,
  \item
    diese Multiplikation assoziativ ist,
  \item
    diese Multiplikation kommutativ ist,
  \item
    es für diese Multiplikation ein Einselement gibt,
  \item
    die Distributivgesetze für die Addition $+$ und Multiplikation $\cdot$ auf $R/I$ gelten.
\end{itemize}

Für die Wohldefiniertheit der Multiplikation seien $x, x, y, y' \in R$ with $\class{x} = \class{x'}$ und $\class{y} = \class{y'}$.
Dann gilt $x - x', y - y' \in I$ und somit auch
\[
      xy - x'y'
  =   xy - xy' + xy' - x'y'
  =   x \underbrace{(y-y')}_{\in I} + \underbrace{(x-x')}_{\in I} y'
  \in I,
\]
also $\class{xy} = \class{x'y'}$.
Das zeigt die Wohldefiniertheit der Multiplikation.
Für alle $\class{x}, \class{y}, \class{z} \in R/I$ gilt
\[
    \class{x} \cdot (\class{y} \cdot \class{z})
  = \class{x} \cdot \class{yz}
  = \class{xyz}
  = \class{xy} \cdot \class{z}
  = (\class{x} \cdot \class{y}) \cdot \class{z},
\]
was die Assoziativität der Multiplikation zeigt.
Für alle $\class{x}, \class{y} \in R/I$ gilt
\[
    \class{x} \cdot \class{y}
  = \class{xy}
  = \class{yx}
  = \class{y} \cdot \class{x},
\]
was die Kommutativität der Multiplikation zeigt.
Das Element $\class{1} \in R/I$ ist ein Einselement für die Multiplikation, denn für alle $\class{x} \in R/I$ gilt
\[
    \class{1} \cdot \class{x}
  = \class{1 \cdot x}
  = \class{x}.
\]
Die Distributivität der Multiplikation im ersten Argument ergibt sich darus, dass für alle $\class{x}, \class{y}, \class{z} \in R/I$ die Gleichheit
\[
  (\class{x} + \class{y}) \cdot \class{z}
  = \class{x + y} \cdot \class{z}
  = \class{(x + y) z}
  = \class{xz + yz}
  = \class{xz} + \class{yz}
  = \class{x} \cdot \class{z} + \class{y} \cdot \class{z}
\]
gilt.
Da die Multiplikation auf $R/I$ kommutativ ist, ergibt sich hieraus auch die Distributivität im zweiten Argument.

Ingesamt zeigt dies, dass $R/I$ mit der gegeben Addition und Multiplikation ein kommutativer Ring ist.

\begin{remark}
  Im Falle $R = \integer$ und $I = (n) = n \integer = \{ an \suchthat a \in \integer \}$ ist die Konstruktion von $R/I = \integer/n\integer$ bereits aus der Linearen Algebra I bekannt.
\end{remark}

\begin{remark}
  Die Konstruktion des Quotientenringes $R/I$ funktioniert auch für einen nicht-kommutativen Ring, sofern man fordert, dass $I$ ein beidseitiges Ideal ist, d.h.\ dass $r x , x r \in I$ für alle $x \in I$ und $r \in R$ gilt.
\end{remark}





\subsection{}

Als Ringhomomorphismus ist $\varphi$ insbesondere ein Gruppenhomomorphismus zwischen den unterliegenden additiven Gruppen von $R$ und $S$;
deshalb gilt
\[
  \varphi(0) = 0
\]
und somit $0 \in \ker \varphi$, und für jedes $x \in \ker \varphi$ gilt
\[
    \varphi(-x)
  = - \varphi(x)
  = - 0
  = 0
\]
und somit auch $-x \in \ker \varphi$.
Für alle $x, y \in \ker \varphi$ gilt außerdem
\[
    \varphi(x + y)
  = \varphi(x) + \varphi(y)
  = 0 + 0
  = 0,
\]
und somit auch $x + y \in \ker \varphi$.
Das zeigt, dass $\ker \varphi$ eine Untergruppe der additiven Gruppe von $R$ ist.
(Eventuell wurde dies auch schon in der Linearen Algebra I gezeigt.)
Für alle $r \in R$ und $x \in \ker \varphi$ gilt
\[
    \varphi(rx)
  = \varphi(r) \varphi(x)
  = \varphi(r) \cdot 0
  = 0,
\]
und somit auch $rx \in \ker \varphi$.
Somit ist $\ker \varphi$ bereits ein Ideal in $R$.

\begin{example}
  Ist $K$ ein Körper, so sind $\{0\}, K \subseteq K$ die einzigen beiden Ideale in $K$:
  Ist nämlich $I \subseteq K$ ein Ideal mit $I \neq \{0\}$, so gibt es ein $x \in I$ mit $x \neq 0$.
  Dann gilt auch $1 = x^{-1} x \in I$, und für jedes $y \in K$ somit auch $y = y \cdot 1 \in I$.
  Deshalb gilt dann bereits $I = K$.
  
  Ist nun $R$ ein Ring und $\varphi \colon K \to R$ ein Ringhomomorphismus, so ist $\ker \varphi \subseteq K$ ein Ideal.
  Ist $R \neq 0$, so gilt $0_R \neq 1_R$ und deshalb $1_K \notin \ker \varphi$.
  Somit muss nach der obigen Beobachtung bereits $\ker \varphi = 0$ gelten, und $\varphi$ deshalb injektiv sein.
  
  Ringhomomorphismen aus Körpern heraus sind also injektiv, sofern sie nicht ausgerechnet in den Nullring gehen.
\end{example}





\subsection{}

Wir betrachten den kommutativen Ring $R/I$ und die Abbildung $\pi \colon R \to R/I$, $x \mapsto \class{x}$.
Dies ist ein Ringhomomorphismus, denn für alle $x, y \in R$ gilt
\begin{gather*}
    \pi(x + y)
  = \class{x + y}
  = \class{x} + \class{y}
  = \pi(x) + \pi(y)
\shortintertext{sowie}
    \pi(x \cdot y)
  = \class{x \cdot y}
  = \class{x} \cdot \class{y}
  = \pi(x) \cdot \pi(y),
\end{gather*}
und es gilt $\pi(1_R) = \class{1_R} = 1_{R/I}$.
Für den Ringhomomorphismus $\pi$ gilt
\begin{align*}
      \ker \pi
  =  \{ x \in R \suchthat \pi(x) = 0 \}
   =  \left\{ x \in R \suchthatscale \class{x} = \class{0} \right\}
  =  \{ x \in R \suchthat x - 0 \in I \}
   =  \{ x \in R \suchthat x \in I \}
   =  I,
\end{align*}
was die gegebene Behauptung zeigt.


\begin{remark}
  Analog zu den letzten beiden Aufgabenteilen ergibt sich für einen nicht-kommutativen Ring $R$, dass $I \subseteq R$ genau dann beidseitiges Ideal ist, wenn es einen Ring $S$ und einen Ringhomomorphismus $\varphi \colon R \to S$ mit $\ker \varphi = I$ gibt.
\end{remark}







