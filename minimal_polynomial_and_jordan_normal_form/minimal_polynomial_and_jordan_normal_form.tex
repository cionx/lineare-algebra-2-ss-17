\documentclass[a4paper,10pt]{scrartcl}

\usepackage{../generalstyle}

\title{Minimalpolynom und Jordan-Normalform}
\author{Jendrik Stelzner}
\date{\today}

\begin{document}

\maketitle

Wir fixieren einen Körper $K$.
Wir zeigen im Folgenden die folgende Aussage:

\begin{theorem}
  \label{theorem: minimalpolynomial}
  Es sei $A \in \matrices{n}{K}$, so dass $A$ über $K$ eine Jordan-Normalform besitzt.
  Dann ist das Minimalpolynom von $A$ durch $m_A(t) = (t - \lambda_1)^{m_1} \dotsm (t - \lambda_r)^{m_r}$ gegeben, wobei $\lambda_1, \dotsc, \lambda_r$ die paarweise verschiedenen Eigenwerte von $A$ sind, und für alle $i = 1, \dotsc, n$ die Potenz $m_i$ mit der Größe des größten Jordanblocks zum Eigenwert $\lambda_i$ in der Jordannormalform von $A$ übereinstimmt.
\end{theorem}

Wir beginnen mit der folgenden Beobachtung:

\begin{lemma}
  \label{lemma: multiplication of block diagonal matrices}
  Es seien $D, D' \in \matrices{n}{K}$ Blockdiagonalmatrizen der Form
  \[
      D
    = \begin{pmatrix}
        A_1 &         &       \\
            & \ddots  &       \\
            &         & A_r
      \end{pmatrix}
    \quad\text{und}\quad
      D'
    = \begin{pmatrix}
        A'_1  &         &       \\
              & \ddots  &       \\
              &         & A'_r
      \end{pmatrix}
  \]
  mit $A_i, A_i' \in \matrices{n_i}{K}$ für alle $i = 1, \dotsc, r$, wobei $n = n_1 + \dotsb + n_r$.
  Dann gilt
  \[
      D D'
    = \begin{pmatrix}
        A_1 &         &     \\
            & \ddots  &     \\
            &         & A_r
      \end{pmatrix}
      \begin{pmatrix}
        A'_r  &         &       \\
              & \ddots  &       \\
              &         & A'_r
      \end{pmatrix}
    = \begin{pmatrix}
        A_1 A'_1  &         &           \\
                  & \ddots  &           \\
                  &         & A_r A'_r
      \end{pmatrix}.
  \]
\end{lemma}

\begin{proof}
  Es sei $V$ ein Vektorraum mit Basis $\basis{B} = (v_1, \dotsc, v_n)$.
  Für alle $i = 1, \dotsc, r$ ist die Teilfamilie $\basis{B}_i \coloneqq (v_{n_1 + \dotsb + n_{i-1} + 1}, \dotsc, v_{n_1 + \dotsb + n_i})$ eine Basis des Untervektorraums $U_i \coloneqq \generated{ \basis{B}_i }$, und für diese Untervektorräume gilt $V = U_1 \oplus \dotsb \oplus U_r$.
  Für alle $i = 1, \dotsc, r$ gibt es eindeutige Endomorphismen $f_i, f'_i \colon U_i \to U_i$ mit $\repmatrix{\basis{B}_i}{ \basis{B}_i}{f_i} = A_i$ und $\repmatrix{\basis{B}_i}{\basis{B}_i}{f'_i} = A'_i$.
  
  Für alle Endomorphismen $g_1 \colon U_1 \to U_1, \dotsc, g_r \colon U_r \to U_r$ gibt es einen eindeutigen Endomorphismus $g_1 \oplus \dotsb \oplus g_r \colon V \to V$ mit
  \[
      (g_1 \oplus \dotsb \oplus g_r)(u_1 + \dotsb + u_r)
    = g_1(u_1) + \dotsb + g_r(u_r)
    \qquad
    \text{für alle $u_1 \in U_1, \dotsc, u_r \in U_r$},
  \]
  und für diesen gilt
  \[
      \repmatrix{\basis{B}}{\basis{B}}{g_1 \oplus \dotsb \oplus g_r}
    = \begin{pmatrix}
        \repmatrix{\basis{B}_1}{\basis{B}_1}{g_1} &         &
      \\
                                                  & \ddots  &
      \\
                                                  &         & \repmatrix{\basis{B}_r}{\basis{B}_r}{g_r}
      \end{pmatrix}.
  \]
  Für alle Endomorphismen $g_1, g'_1 \colon U_1 \to U_1, \dotsc, g_r, g'_r \colon U_r \to U_r$ gilt
  \begin{align*}
        (g_1 \oplus \dotsb \oplus g_r)\left( (g'_1 \oplus \dotsb \oplus g'_r)(u_1 + \dotsb + u_r) \right)
    &=  g_1(g'_1(u_1)) + \dotsb + g_r(g'_r(u_r))
    \\
    &=  (g_1 \circ g'_1)(u_1) + \dotsb + (g_r \circ g'_r)(u_r)
  \end{align*}
  für alle $u_1 \in U_1, \dotsc, u_r \in U_r$, und deshalb gilt
  \[
    (g_1 \oplus \dotsb \oplus g_r) \circ (g'_1 \oplus \dotsb \oplus g'_r)
    = (g_1 \circ g'_1) \oplus \dotsb \oplus (g_r \circ g'_r).
  \]
  
  Damit erhalten wir nun, dass
  \begin{align*}
          D D'
    &=  \begin{pmatrix}
          A_1 &         &     \\
              & \ddots  &     \\
              &         & A_r
        \end{pmatrix}
        \begin{pmatrix}
          A'_r  &         &       \\
                & \ddots  &       \\
                &         & A'_r
        \end{pmatrix}
    \\
    &=  \begin{pmatrix}
          \repmatrix{\basis{B}_1}{\basis{B}_1}{f_1} &         &     \\
                                                    & \ddots  &     \\
                                                    &         & \repmatrix{\basis{B}_r}{\basis{B}_r}{f_r}
        \end{pmatrix}
        \begin{pmatrix}
          \repmatrix{\basis{B}_1}{\basis{B}_1}{f'_1}  &         &     \\
                                                      & \ddots  &     \\
                                                      &         & \repmatrix{\basis{B}_r}{\basis{B}_r}{f'_r}
        \end{pmatrix}
    \\
    &=  \repmatrix{\basis{B}}{\basis{B}}{f_1 \oplus \dotsb \oplus f_r}
        \repmatrix{\basis{B}}{\basis{B}}{f'_1 \oplus \dotsb \oplus f'_r}
      = \repmatrix{\basis{B}}{\basis{B}}{(f_1 \oplus \dotsb \oplus f_r) \circ (f'_1 \oplus \dotsb \oplus f'_r)}
    \\
    &=  \repmatrix{\basis{B}}{\basis{B}}{(f_1 \circ f'_1) \oplus \dotsb \oplus (f_r \circ f'_r)}
     =  \begin{pmatrix}
            \repmatrix{\basis{B}_1}{\basis{B}_1}{f_1 \circ f'_1} 
          & {}
          & {}
          \\
            {}
          & \ddots
          & {}
          \\
            {}
          & {}
          & \repmatrix{\basis{B}_r}{\basis{B}_r}{f_r \circ f'_r}
        \end{pmatrix}
    \\
    &=  \begin{pmatrix}
            \repmatrix{\basis{B}_1}{\basis{B}_1}{f_1} \repmatrix{\basis{B}_1}{\basis{B}_1}{f'_1}
          & {}
          & {}
          \\
            {}
          & \ddots
          & {}
          \\
            {}
          & {}
          & \repmatrix{\basis{B}_r}{\basis{B}_r}{f_r} \repmatrix{\basis{B}_r}{\basis{B}_r}{f'_r}
        \end{pmatrix}
     =  \begin{pmatrix}
          A_1 A'_1  &         &           \\
                    & \ddots  &           \\
                    &         & A_r A'_r
        \end{pmatrix}
  \end{align*}
  gilt, was die Aussage zeigt.
\end{proof}


Wir zeigen Satz~\ref{theorem: minimalpolynomial} nun schrittweise.
Da das Minimalpolynom invariant unter Ähnlichkeit ist, können wir dabei o.B.d.A.\ davon ausgehen, dass $A$ in Jordan-Normalform ist.
\begin{itemize}
  \item
    Es sei zunächst $A$ ein einzelner Jordanblock zum Eigenwert $0$, d.h.\ es gelte
    \[
        A
      = \begin{pmatrix}
          0 & 1 &         &         &   \\
            & 0 & 1       &         &   \\
            &   & \ddots  & \ddots  &   \\
            &   &         & \ddots  & 1 \\
            &   &         &         & 0
        \end{pmatrix}
      \in \matrices{n}{K}.
    \]
    Dann gilt $A^n = 0$ aber $A^k \neq 0$ für alle $k < n$, weshalb $m_A(t) = t^n$.
  \item
    Es sei nun $A$ ein einzelner Jordanblock zum Eigenwert $\lambda \in K$, d.h.\ es gelte
    \[
        A
      = \begin{pmatrix}
          \lambda & 1       &         &         &         \\
                  & \lambda & 1       &         &         \\
                  &         & \ddots  & \ddots  &         \\
                  &         &         & \ddots  & 1       \\
                  &         &         &         & \lambda
        \end{pmatrix}
      \in \matrices{n}{K}.
    \]
    Dann ist $A - \lambda I$ ein Jordanblock zum Eigenwert $0$, weshalb $(A - \lambda I)^n = 0$ aber $(A - \lambda I)^k \neq 0$ für alle $k < n$ gilt.
    Deshalb gilt $m_A(t) = (t - \lambda)^n$.
  \item
    Es sei nun $A$ in Jordan-Normalform
    \[
        A
      = \begin{pmatrix}
          J_{n_1}(\mu_1)  &         &                 \\
                          & \ddots  &                 \\
                          &         & J_{n_s}(\mu_s)
        \end{pmatrix}
      \in \matrices{n}{K},
    \]
    wobei für alle $n' \geq 1$ und $\mu \in K$ die Matrix $J_{n'}(\mu) \in \matrices{n'}{K}$ den Jordanblock von Größe $n' \times n'$ zum Eigenwerte $\mu$ bezeichnet, d.h.\
    \[
        J_n(\mu)
      = \begin{pmatrix}
          \mu & 1   &         &         &         \\
              & \mu & 1       &         &         \\
              &     & \ddots  & \ddots  &         \\
              &     &         & \ddots  & 1       \\
              &     &         &         & \mu
        \end{pmatrix}
        \in \matrices{n'}{\mu}.
    \]
    Aus Lemma~\ref{lemma: multiplication of block diagonal matrices} ergibt sich, dass
    \[
        A^k
      = \begin{pmatrix}
          J_{n_1}(\mu_1)^k  &         &                   \\
                            & \ddots  &                   \\
                            &         & J_{n_s}(\mu_s)^k
        \end{pmatrix}
      \qquad
      \text{für alle $k \geq 0$}
    \]
    gilt. Damit ergibt sich allgemeiner, dass
    \[
        p(A)
      = \begin{pmatrix}
          p( J_{n_1}(\mu_1) ) &         &                     \\
                              & \ddots  &                     \\
                              &         & p( J_{n_s}(\mu_s) )
        \end{pmatrix}
      \qquad
      \text{für alle $p(t) \in K[t]$}
    \]
    gilt.
    Inbesondere gilt deshalb für jedes $p(t) \in K[t]$, dass
    \[
            p(A) = 0
      \iff  \text{$p( J_{n_i}(\mu_i) ) = 0$ für alle $i = 1, \dotsc, s$}
      \iff  \text{$m_{ J_{n_i}(\mu_i) }(t) \divides p(t)$ für alle $i = 1, \dotsc, s$}.
    \]
    Wir haben bereits gezeigt, dass $m_{J_{n'}(\mu)} = (t - \mu)^{n'}$ für alle $n' \geq 1$ und $\mu \in K$ gilt.
    Deshalb gilt
    \begin{equation}
      \label{equation: characterization of vanishing polynomials}
            p(A) = 0
      \iff  \text{$(t - \mu_i)^{n_i} \divides p(t)$ für alle $i = 1, \dotsc, s$}.
    \end{equation}
    
    Es seien $\lambda_1, \dotsc, \lambda_r \in K$ die paarweise verschiedenen Eigenwerte von $A$, d.h.\ es gelte $\lambda_i \neq \lambda_j$ für alle $i \neq j$ und $\{\mu_1, \dotsc, \mu_s\} = \{\lambda_1, \dotsc, \lambda_r\}$, und für alle $i = 1, \dotsc, r$ sei
    \[
      m_i \coloneqq \max\{ n_j \suchthat \text{$1 \leq j \leq s$ mit $\mu_j = \lambda_i$} \}.
    \]
    
    Dann ist $\tilde{m}_A(t) \coloneqq (t - \lambda_1)^{m_1} \dotsb (t - \lambda_r)^{m_r}$ das minimale normierte, vom Nullpolynom verschiedene Polynom mit $(t - \mu_i)^{n_i} \divides p(t)$ für alle $i = 1, \dotsc, s$.
    Nach \eqref{equation: characterization of vanishing polynomials} ist $\tilde{m}_A(t)$ somit das Minimalpolynom von $A$.
    Dabei ist $m_i$ für alle $i = 1, \dotsc, r$ die maxmiale Größe eines Jordanblocks zum Eigenwert $\lambda_i$.
\end{itemize}












\end{document}
