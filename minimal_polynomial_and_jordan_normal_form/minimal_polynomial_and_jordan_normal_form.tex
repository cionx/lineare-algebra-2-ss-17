\documentclass[a4paper,10pt]{scrartcl}

\usepackage{../generalstyle}

\title{Minimalpolynom und Jordan-Normalform}
\author{Jendrik Stelzner}
\date{\today}

\begin{document}

\maketitle

Wir fixieren einen Körper $K$.
Wir zeigen im Folgenden die folgende Aussage:

\begin{theorem}
  \label{theorem: minimalpolynomial}
  Es sei $A \in \matrices{n}{K}$, so dass $A$ über $K$ eine Jordan-Normalform besitzt, d.h.\ das charakteristische Polynom von $A$ zerfalle über $K$ in Linearfaktoren:
  \[
    p_A(t) = (-1)^n (t - \lambda_1)^{n_1} \dotsm (t  \lambda_r)^{n_r}
  \]
  
  Dann ist das Minimalpolynom von $A$ durch
  \[
      m_A(t)
    = (t - \lambda_1)^{m_1} \dotsm (t - \lambda_r)^{m_r}
  \]
  gegeben, wobei $\lambda_1, \dotsc, \lambda_r$ die paarweise verschiedenen Eigenwerte von $A$ sind, und für alle $i = 1, \dotsc, n$ die Potenz $m_i$ mit der Größe des größten Jordanblocks zum Eigenwert $\lambda_i$ in der Jordannormalform von $A$ übereinstimmt.
  
  Insbesondere gelten $m_i \geq 1$ und $m_i \leq n_i$ für alle $i = 1, \dotsc, r$.
\end{theorem}

\begin{proof}
  Da das Minimalpolynom invariant unter Ähnlichkeit ist, können wir dabei o.B.d.A.\ davon ausgehen, dass $A$ in Jordan-Normalform ist.
  Wir gehen nun schrittweise vor.
  \begin{itemize}
    \item
      Es sei zunächst $A$ ein einzelner Jordanblock zum Eigenwert $0$, d.h.\ es gelte
      \[
          A
        = \begin{pmatrix}
            0 & 1 &         &         &   \\
              & 0 & 1       &         &   \\
              &   & \ddots  & \ddots  &   \\
              &   &         & \ddots  & 1 \\
              &   &         &         & 0
          \end{pmatrix}
        \in \matrices{n}{K}.
      \]
      Dann gilt $A^n = 0$ aber $A^k \neq 0$ für alle $k < n$, weshalb $m_A(t) = t^n$ gilt.
    \item
      Es sei nun $A$ ein einzelner Jordanblock zum Eigenwert $\lambda \in K$, d.h.\ es gelte
      \[
          A
        = \begin{pmatrix}
            \lambda & 1       &         &         &         \\
                    & \lambda & 1       &         &         \\
                    &         & \ddots  & \ddots  &         \\
                    &         &         & \ddots  & 1       \\
                    &         &         &         & \lambda
          \end{pmatrix}
        \in \matrices{n}{K}.
      \]
      Dann ist $A - \lambda I$ ein Jordanblock zum Eigenwert $0$, weshalb $(A - \lambda I)^n = 0$ aber $(A - \lambda I)^k \neq 0$ für alle $k < n$ gilt.
      Deshalb gilt $m_A(t) = (t - \lambda)^n$.
    \item
      Es sei nun $A$ in Jordan-Normalform
      \[
          A
        = \begin{pmatrix}
            J_{n_1}(\mu_1)  &         &                 \\
                            & \ddots  &                 \\
                            &         & J_{n_s}(\mu_s)
          \end{pmatrix}
        \in \matrices{n}{K},
      \]
      wobei für alle $n' \geq 1$ und $\mu \in K$ die Matrix $J_{n'}(\mu) \in \matrices{n'}{K}$ den Jordanblock von Größe $n' \times n'$ zum Eigenwerte $\mu$ bezeichnet, d.h.\
      \[
          J_{n'}(\mu)
        = \begin{pmatrix}
            \mu & 1   &         &         &         \\
                & \mu & 1       &         &         \\
                &     & \ddots  & \ddots  &         \\
                &     &         & \ddots  & 1       \\
                &     &         &         & \mu
          \end{pmatrix}
          \in \matrices{n'}{\mu}.
      \]
      
%       (Die Skalare $\mu_1, \dotsc, \mu_s$ sind die nicht-notwendigerweise paarweise verschiedenen Eigenwerte von $A$, während $\lambda_1, \dotsc, \lambda_r$ die paarweise verschiedenen Eigenwerte von $A$ bezeichnen.
%       Es gilt also $\{\mu_1, \dotsc, \mu_s\} = \{\lambda_1, \dotsc, \lambda_r\}$, aber i.A.\ gilt $s > r$.)
      
      Aus
      \[
          A^k
        = \begin{pmatrix}
            J_{n_1}(\mu_1)^k  &         &                   \\
                              & \ddots  &                   \\
                              &         & J_{n_s}(\mu_s)^k
          \end{pmatrix}
        \qquad
        \text{für alle $k \geq 0$}
      \]
      erhalten wir allgemeiner, dass
      \[
          p(A)
        = \begin{pmatrix}
            p( J_{n_1}(\mu_1) ) &         &                     \\
                                & \ddots  &                     \\
                                &         & p( J_{n_s}(\mu_s) )
          \end{pmatrix}
        \qquad
        \text{für alle $p(t) \in K[t]$}
      \]
      gilt.
      Inbesondere gilt deshalb für jedes $p(t) \in K[t]$, dass
      \begin{align*}
              p(A) = 0
        &\iff \text{$p( J_{n_i}(\mu_i) ) = 0$ für alle $i = 1, \dotsc, s$}
        \\
        &\iff \text{$m_{ J_{n_i}(\mu_i) }(t) \divides p(t)$ für alle $i = 1, \dotsc, s$}.
      \end{align*}
      Wir haben bereits gezeigt, dass $m_{J_{n'}(\mu)} = (t - \mu)^{n'}$ für alle $n' \geq 1$ und $\mu \in K$ gilt.
      Deshalb gilt
      \begin{equation}
        \label{equation: characterization of vanishing polynomials}
              p(A) = 0
        \iff  \text{$(t - \mu_i)^{n_i} \divides p(t)$ für alle $i = 1, \dotsc, s$}.
      \end{equation}
      
      Es seien $\lambda_1, \dotsc, \lambda_r \in K$ die paarweise verschiedenen Eigenwerte von $A$, d.h.\ es gelte $\lambda_i \neq \lambda_j$ für alle $i \neq j$ und $\{\mu_1, \dotsc, \mu_s\} = \{\lambda_1, \dotsc, \lambda_r\}$, und für alle $i = 1, \dotsc, r$ sei
      \[
        m_i \coloneqq \max\{ n_j \suchthat \text{$1 \leq j \leq s$ mit $\mu_j = \lambda_i$} \},
      \]
      d.h.\ $m_i$ ist die Größe des größten Jordanblocks zum Eigenwert $\lambda_i$.
      
      Dann ist $\tilde{m}_A(t) \coloneqq (t - \lambda_1)^{m_1} \dotsb (t - \lambda_r)^{m_r}$ das minimale normierte, vom Nullpolynom verschiedene Polynom mit $(t - \mu_i)^{n_i} \divides p(t)$ für alle $i = 1, \dotsc, s$.
      Nach \eqref{equation: characterization of vanishing polynomials} ist $\tilde{m}_A(t)$ somit das Minimalpolynom von $A$.
    \qedhere
  \end{itemize}
\end{proof}












\end{document}
