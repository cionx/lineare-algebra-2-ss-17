\section{}





\subsection{}

Für alle $x \in \real^n$ mit $x \neq 0$ gilt
\[
        \beta(x,x)
  =     \bilinear{x}{x} + \bilinear{A x}{A x} + \dotsb + \bilinear{A^{k-1} x}{A^{k-1} x}
  =     \|x\|^2 + \underbrace{ \|A x\|^2 + \dotsb + \|A^{k-1} x\|^2 }_{\geq 0}
  \geq  \|x\|^2
  >     0,
\]
da das Standardskalarprodukt $\bilinear{-}{-}$ positiv definit ist.
Also ist $\beta$ eine positiv definitive, symmetrische Bilinearform, d.h.\ ein Skalarprodukt.

Ferner gilt $\bilinear{A^k x}{A^k y} = \bilinear{x}{y}$ für alle $x, y \in \real^n$ da $A^k$ orthogonal ist, und somit auch
\begin{align*}
     \beta(A x, A y)
  &= \bilinear{A x}{A y} + \bilinear{A^2 x}{A^2 x} + \dotsb + \bilinear{A^k x}{A^k y}
  \\
  &= \bilinear{A x}{A y} + \bilinear{A^2 x}{A^2 x} + \dotsb + \bilinear{x}{y}
   = \beta(x, y).
\end{align*}





\subsection{}

Es sei $f \colon \real^n \to \real^n$, $x \mapsto Ax$ die zu $A$ (bezüglich der Standardbasis) gehörige lineare Abbildung.
Bezüglich der Standardbasis $\basis{S} = (e_1, \dotsc, e_n)$ gilt $A = \repmatrix{f}{\basis{S}}{\basis{S}}$.

Im vorherigen Aufgabenteil haben wir gezeigt, dass $\beta$ ein Skalarprodukt auf $\real^n$ ist, und dass $f \in \orthogonal{\real^n, \beta}$ gilt.
Ist $\basis{B} = (v_1, \dotsc, v_n)$ eine Orthonormalbasis von $\real^n$ bezüglich $\beta$, so ist deshalb die Matrix $U \coloneqq \repmatrix{f}{\basis{B}}{\basis{B}}$ orthogonal.

Für die Basiswechselmatrix $C \coloneqq \repmatrix{\id_{\real^n}}{\basis{S}}{\basis{B}} \in \GL{n}{\real}$ gilt somit
\[
      U
  =   \repmatrix{f}{\basis{B}}{\basis{B}}
  =   \repmatrix{\id_{\real^n}}{\basis{S}}{\basis{B}}
  \,  \repmatrix{f}{\basis{S}}{\basis{S}}
  \,  \repmatrix{\id_{\real^n}}{\basis{B}}{\basis{S}}
  =   \repmatrix{\id_{\real^n}}{\basis{S}}{\basis{B}}
  \,  \repmatrix{f}{\basis{S}}{\basis{S}}
  \,  \repmatrix{\id_{\real^n}}{\basis{S}}{\basis{B}}^{-1}
  =   C A C^{-1}.
\]

