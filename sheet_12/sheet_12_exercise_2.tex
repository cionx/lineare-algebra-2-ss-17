\section{}
Wir betrachten im Folgenden den Fall $K = \complex$.
Für den Fall $K = \real$ ersetze man jeweils $\complex$ durch $\real$, und $\unitary{-}$ durch $\orthogonal{-}$.
Für $A \in \mnatrices{m}{n}{\complex}$ schreiben im Folgenden abkürzend $A^*$ anstelle von $A^*$.

\begin{lemma}
  \label{lemma: decomposition}
  Es sei $A \in \mnatrices{m}{n}{\complex}$ und $r \coloneqq \rank A$.
  Dann gibt $A' \in \GL{r}{\complex}$ sowie $U_1 \in \unitary{m}$ und $U_2 \in \unitary{n}$, so dass
  \[
    A = U
        \begin{pmatrix}
          A'  & 0 \\
          0   & 0
        \end{pmatrix}
        V.
  \]
\end{lemma}
\begin{proof}
  Es sei $f \colon \complex^n \to \complex^m$, $x \mapsto Ax$ die zu $A$ (bezüglich der Standardbasen) gehörige lineare Abbildung.
  Bezüglich der Standardbasen $\basis{S}_n = (e_1, \dotsc, e_n)$ von $\complex^n$ und $\basis{S}_m = (e_1, \dotsc e_m)$ von $\complex^m$ gilt $A = \repmatrix{f}{\basis{S}_n}{\basis{S}_m}$.
  
  Es sei $p \colon \complex^n \to \complex^n / \ker f$, $x \mapsto \class{x}$ die kanonische Projektion und $i \colon \im f \to \complex^m$, $y \mapsto y$ die kanonische Inklusion.
  Nach dem Homomorphiesatz induziert $f$ einen wohldefinierten Isomorphismus
  \[
            \induced{f}
    \colon  \complex^n / \ker f
    \to     \im f,
    \quad   \class{x}
    \mapsto f(x).
  \]
  Dabei ist $\induced{f}$ die eindeutige Abbildung, die das folgende Diagram zum kommutieren bringt:
  \[
    \begin{tikzcd}
        \complex^n
        \arrow{r}{f}
        \arrow[swap]{d}{p}
      & \complex^m
      \\
        \complex^n / \ker f
        \arrow{r}{\induced{f}}
      & \im f
        \arrow[swap]{u}{i}
    \end{tikzcd}
  \]
  
  Es sei $\basis{B}'' = (v_1, \dotsc, v_{n-r})$ eine Basis von $\ker f = \ker A$.
  Wir ergänzen $\basis{B}''$ zu einer Basis $\basis{B} = (v_1, \dotsc, v_n)$ von $\complex^n$.
  Dann ist $\basis{B}' \coloneqq (\class{v_{n-r+1}}, \dotsc, \class{v_n})$ eine Basis von $\complex^n / \ker f$.
  Ferner sei $\basis{C}' = (w_1, \dotsc, w_r)$ eine Basis von $\im f = \im A$, und $\basis{C} = (w_1, \dotsc, w_m)$ eine ergänzte Basis von $\complex^m$.
  
  Es gilt nun, dass
  \[
        \repmatrix{f}{\basis{B}}{\basis{C}}
    =   \repmatrix{i}{\basis{C}'}{\basis{C}}
    \,  \repmatrix{\induced{f}}{\basis{B}'}{\basis{C}'}
    \,  \repmatrix{p}{\basis{B}}{\basis{B}'}.
  \]
  Dabei ist die darstellende Matrix $A \coloneqq \repmatrix{\induced{f}}{\basis{B}'}{\basis{C}'} \in \matrices{r}{\complex}$ ist invertierbar, da $\induced{f}$ ein Isomorphismus ist.
  Außerdem gelten
  \[
      \repmatrix{p}{\basis{B}}{\basis{B}'}
    = \begin{pmatrix}
        I_r & 0
      \end{pmatrix}
      \in
      \mnatrices{r}{n}{\complex}
    \quad
    \text{und}
    \quad
      \repmatrix{i}{\basis{C}'}{\basis{C}}
    = \begin{pmatrix}
        I_r \\
        0
      \end{pmatrix}
      \in
      \mnatrices{m}{r}{\complex}.
  \]
  Somit gilt ingesamt, dass
  \[
    \repmatrix{f}{\basis{B}}{\basis{C}}
    = \begin{pmatrix}
        I_r \\
        0
      \end{pmatrix}
      A'
      \begin{pmatrix}
        I_r & 0
      \end{pmatrix}
    = \begin{pmatrix}
        A' & 0  \\
        0  & 0
      \end{pmatrix}.
  \]
  
  Zusammen mit den beiden Basiswechselmatrizen $U_1 = \repmatrix{\id_{\complex^m}}{\basis{C}}{\basis{S}_m}$ und $U_2 = \repmatrix{\id_{\complex^n}}{\basis{S}_n}{\basis{B}}$ erhalten wir nun, dass
  \[
        A
    =   \repmatrix{f}{\basis{S}_n}{\basis{S}_m}
    =   \repmatrix{\id_{\complex^m}}{\basis{C}}{\basis{S}_m}
    \,  \repmatrix{f}{\basis{B}}{\basis{C}}
    \,  \repmatrix{\id_{\complex^n}}{\basis{S}_n}{\basis{B}}
    =   U_1
        \begin{pmatrix}
          A' & 0  \\
          0  & 0
        \end{pmatrix}
        U_2.
  \]
  
  Man bemerke, dass die Spalten von $U_2$ genau die Basis $\basis{B}$ sind, und die Spalten von $U_1^{-1}$ genau die Basis $\basis{C}$.
  Die Matrizen $U_1$ und $U_2$ sind also genau dann unitär, falls die beiden Basen $\basis{B}$ und $\basis{C}$ jeweils orthonormal sind.
  Da Basisergänzung auch für Orthonormalbasen funktoniert, lassen sich dabei $\basis{B}$ und $\basis{C}$ zusätzlich zu den bisherigen Bedingungen als orthonormal wählen.
\end{proof}


Es seien nun $A'$, $\tilde{U}_1$ und $\tilde{U}_2$ wie in Lemma~\ref{lemma: decomposition}.
Es sei $A' = S V$ die Polarzerlegung von $A$, wobei $S \in \matrices{n}{\complex}$ hermitesch und positiv semidefinit ist, und $V \in \unitary{n}$ unitär.
Da $S$ hermitesch ist, gibt es eine Orthonormalbasis von $\complex^r$ besteht aus Eigenvektoren von $S$, wobei alle Eigenwerte von $S$ reell sind.
Es gibt also $W \in \unitary{n}$ und $d_1, \dotsc, d_r \in \real$ mit
\[
    S
  = W D W^{-1}
  \quad
  \text{wobei}
  \quad
              D
  \coloneqq   \begin{pmatrix}
                d_1 &         &     \\
                    & \ddots  &     \\
                    &         & d_r
              \end{pmatrix}.
\]
Da $S$ positiv semidefinit ist, gilt dabei $d_1, \dotsc, d_r \geq 0$.
Da $A'$ invertierbar ist, gilt dies auch für $S$, weshalb sogar bereits $d_1, \dotsc, d_r > 0$ gilt.
Durch passende Nummerierung der Eigenwerte können wir ferner davon ausgehen, dass $d_1 \geq \dotsb \geq d_r > 0$ gilt.
Wir erhalten nun ingesamt, dass
\begin{align*}
      A
  &=  \tilde{U}_1
      \begin{pmatrix}
        A'  & 0 \\
        0   & 0
      \end{pmatrix}
      \tilde{U}_2
    = \tilde{U}_1
      \begin{pmatrix}
        SV  & 0 \\
        0   & 0
      \end{pmatrix}
      \tilde{U}_2
    = \tilde{U}_1
      \begin{pmatrix}
        W D W^{-1} V  & 0 \\
        0             & 0
      \end{pmatrix}
      \tilde{U}_2
  \\
  &=  \tilde{U}_1
      \begin{pmatrix}
        W & 0       \\
        0 & I_{m-r}
      \end{pmatrix}
      \begin{pmatrix}
        D & 0 \\
        0 & 0
      \end{pmatrix}
      \begin{pmatrix}
        W^{-1} V  & 0       \\
        0         & I_{n-r}
      \end{pmatrix}
      \tilde{U}_2
\end{align*}
Die Matrix $W^{-1} V$ ist unitär, da $W$ und $V$ unitär sind.
Da $W$ und $W^{-1} V$ unitär sind, gilt dies auch für die Matrizen
\[
  \begin{pmatrix}
    W & 0       \\
    0 & I_{m-r}
  \end{pmatrix}
  \quad\text{und}\quad
  \begin{pmatrix}
    W^{-1} V  & 0       \\
    0         & I_{n-r}
  \end{pmatrix}.
\]
Somit sind auch
\[
            U_1
  \coloneqq \tilde{U}_1
            \begin{pmatrix}
              W & 0       \\
              0 & I_{m-r}
            \end{pmatrix}
  \quad
  \text{und}
  \quad
            U_2'
  \coloneqq \tilde{U}_2
            \begin{pmatrix}
              W^{-1} V  & 0       \\
              0         & I_{n-r}
            \end{pmatrix}
\]
unitär.
Mit $U_2 \coloneqq (U_2')^{-1} = (U_2')^* \in \unitary{n}$ erhalten wir nun die gewünschte Zerlegung
\[
    A
  = \tilde{U}_1
    \begin{pmatrix}
      W & 0       \\
      0 & I_{m-r}
    \end{pmatrix}
    \begin{pmatrix}
      D & 0 \\
      0 & 0
    \end{pmatrix}
    \begin{pmatrix}
      W^{-1} V  & 0       \\
      0         & I_{n-r}
    \end{pmatrix}
    \tilde{U}_2
  = U_1
    \begin{pmatrix}
      D & 0 \\
      0 & 0
    \end{pmatrix}
    U_2'
  = U_1
    \begin{pmatrix}
      D & 0 \\
      0 & 0
    \end{pmatrix}
    U_2^*
\]
Damit ergibt sich nun auch, dass
\[
    A^*
  = U_2
    \begin{pmatrix}
      D & 0 \\
      0 & 0
    \end{pmatrix}
    U_1^*
  = U_2
    \begin{pmatrix}
      D & 0 \\
      0 & 0
    \end{pmatrix}
    U_1^{-1}.
\]
Somit gilt
\[
    A^* A
  = U_2
    \begin{pmatrix}
      D & 0 \\
      0 & 0
    \end{pmatrix}
    U_1^{-1}
    U_1
    \begin{pmatrix}
      D & 0 \\
      0 & 0
    \end{pmatrix}
    U_2^*
  = U_2
    \begin{pmatrix}
      D^2 & 0 \\
      0   & 0
    \end{pmatrix}
    U_2^*
  = U_2
    \begin{pmatrix}
      D^2 & 0 \\
      0   & 0
    \end{pmatrix}
    U_2^{-1}.
\]
Die Matrix $A^* A$ ist also ähnlich zu der Diagonalmatrix $\begin{psmallmatrix} D^2 & 0 \\ 0 & 0 \end{psmallmatrix} \in \matrices{n}{\complex}$.
Dabei gilt
\[
    D^2
  = \begin{pmatrix}
      d_1^2 &         &       \\
            & \ddots  &       \\
            &         & d_r^2
    \end{pmatrix}.
\]
Die Eigenwerte von $A^* A$ sind folglich $d_1^2, \dotsc, d_r^2$, sowie im Fall $n > r$ auch noch $0$.






