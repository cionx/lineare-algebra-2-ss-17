\section{}

Wir berechnen im Folgenden die Polarzerlegung $A = SU$, wobei $S \in \matrices{n}{\real}$ symmetrisch und positiv semidefinit ist, und $U \in \orthogonal{n}$ orthogonal.
Man könnte stattdessen auch die Zerlegung $A = U'S'$ berechnen, wir halten uns hier aber an die Konvention von Satz XIV.4 aus der Vorlesung.

Die Matrix $S$ ist eindeutig bestimmt durch $S = \sqrt{A \transpose{A}}$, d.h.\ $S$ ist die eindeutige symmetrische, positiv semidefinite Wurzel aus der symmetrischen, positiv semidefiniten Matrix $A \transpose{A}$.
Da $A$ invertierbar ist (denn $\det A = -6 \neq 0$) ist auch $S$ invertierbar, und $U$ somit eindeutig bestimmt als $U = S^{-1} A$.

Wir betrachten also zunächst die Matrix
\[
            B
  \coloneqq A \transpose{A}
  =         \begin{pmatrix*}[r]
              1 &   &    \\
                & 2 &  1 \\
                & 2 & -2
            \end{pmatrix*}
            \begin{pmatrix*}[r]
              1 &   &    \\
                & 2 &  2 \\
                & 1 & -2
            \end{pmatrix*}
  =         \begin{pmatrix}
              1 &   &   \\
                & 5 & 2 \\
                & 2 & 8
            \end{pmatrix}.
\]
Als reelle, symmetrische Matrix ist $B$ diagonalisierbar.
Das charakteristische Polynom von $B$ ist gegeben durch
\[
    p_B(t)
  = -(t-1)(t^2 - 13t + 36)
  = -(t-1)(t-4)(t-9).
\]
Die Eigenwerte von $B$ sind also $1$, $4$ und $9$;
entsprechende Eigenvektoren sind gegeben durch
\[
            \tilde{v}_1
  \coloneqq \vect{1 \\ 0 \\ 0},
  \quad
            \tilde{v}_2
  \coloneqq \vect{0 \\ 2 \\ -1},
  \quad
            \tilde{v}_3
  \coloneqq \vect{0 \\ 1 \\ 2}.
\]
Da es sich um Eigenvektoren zu paarweise verschiedenen Eigenwerten handelt, sind diese Vektoren linear unabhängig.
Da $B$ symmetrisch ist, sind sie sogar paarweise orthogonal zueinander.
Um uns im Folgenden ein wenig Rechenarbeit zu ersparen, normieren wir die obigen Vektoren noch; wir erhalten somit die folgende Orthonormalbasis von $\real^3$ aus Eigenvektoren von $B$:
\[
           v_1
  \coloneqq \vect{1 \\ 0 \\ 0},
  \quad
            v_2
  \coloneqq \frac{1}{\sqrt{5}} \vect{0 \\ 2 \\ -1},
  \quad
            v_3
  \coloneqq \frac{1}{\sqrt{5}} \vect{0 \\ 1 \\ 2}.
\]
Damit erhalten wir nun die Basiswechselmatrix
\begin{gather*}
    C
  \coloneqq \frac{1}{\sqrt{5}}
            \begin{pmatrix*}[r]
              \sqrt{5}  &     &   \\
                        &  2  & 1 \\
                        & -1  & 2
            \end{pmatrix*},
\shortintertext{für die}
    B
  = C
    \begin{pmatrix}
      1 &   &   \\
        & 4 &   \\
        &   & 9
    \end{pmatrix}
    C^{-1}
\end{gather*}
gilt.
Somit erhalten wir, dass
\[
    S
  = \sqrt{A \transpose{A}}
  = \sqrt{B}
  = C
    \begin{pmatrix}
      1 &   &   \\
        & 2 &   \\
        &   & 3
    \end{pmatrix}
    C^{-1}
\]
gilt.
Da die Spalten von $C$ eine Orthonormalbasis von $\real^3$ bilden, ist $C$ orthogonal.
Deshalb gilt $C^{-1} = \transpose{C}$, und somit auch 
\[
    S
  = C
    \begin{pmatrix}
      1 &   &   \\
        & 2 &   \\
        &   & 3
    \end{pmatrix}
    \transpose{C}
  = \frac{1}{5}
    \begin{pmatrix*}[r]
      \sqrt{5}  &     &   \\
                &  2  & 1 \\
                & -1  & 2
    \end{pmatrix*}
    \begin{pmatrix}
      1 &   &   \\
        & 2 &   \\
        &   & 3
    \end{pmatrix}
    \begin{pmatrix*}[r]
      \sqrt{5}  &     &     \\
                &  2  & -1  \\
                &  1  &  2
    \end{pmatrix*}
  = \frac{1}{5}
    \begin{pmatrix*}[r]
      5 &     &     \\
        & 11  &  2  \\
        &  2  & 14
    \end{pmatrix*}.
\]
Das Inverse $S^{-1}$ ergibt sich nun wahlweise durch den üblichen Gauß-Algorithmus, oder als
\begin{align*}
    S^{-1}
  &=  \left(
      C
      \begin{pmatrix}
        1 &   &   \\
          & 2 &   \\
          &   & 3
      \end{pmatrix}
      C^{-1}
      \right)^{-1}
   =  C
      \begin{pmatrix}
        1 &   &   \\
          & 2 &   \\
          &   & 3
      \end{pmatrix}^{-1}
      C^{-1}
   =  C
      \begin{pmatrix}
        1 &             &             \\
          & \frac{1}{2} &             \\
          &             & \frac{1}{3}
      \end{pmatrix}
      C^{-1}
  \\
  &=  \frac{1}{6}
      C
      \begin{pmatrix}
        6 &   &   \\
          & 3 &   \\
          &   & 2
      \end{pmatrix}
      C^{-1}
   =  \frac{1}{6}
      C
      \begin{pmatrix}
        6 &   &   \\
          & 3 &   \\
          &   & 2
      \end{pmatrix}
      \transpose{C}
  \\
  &=  \frac{1}{30}
      \begin{pmatrix*}[r]
        \sqrt{5}  &     &   \\
                  &  2  & 1 \\
                  & -1  & 2
      \end{pmatrix*}
      \begin{pmatrix}
        6 &   &   \\
          & 3 &   \\
          &   & 2
      \end{pmatrix}
      \begin{pmatrix*}[r]
        \sqrt{5}  &   &     \\
                  & 2 & -1  \\
                  & 1 &  2
    \end{pmatrix*}
  = \frac{1}{30}
    \begin{pmatrix*}[r]
      30  &     &     \\
          & 14  & -2  \\
          & -2  & 11
    \end{pmatrix*}.
\end{align*}
Somit erhalten wir nun, dass
\[
    U
  = S^{-1} A
  = \frac{1}{30}
    \begin{pmatrix*}[r]
      30  &     &     \\
          & 14  & -2  \\
          & -2  & 11
    \end{pmatrix*}
    \begin{pmatrix*}[r]
      1 &   &     \\
        & 2 &  1  \\
        & 2 & -2
    \end{pmatrix*}
  = \frac{1}{5}
    \begin{pmatrix}
      5 &   &     \\
        & 4 &  3  \\
        & 3 & -4
    \end{pmatrix}.
\]

\begin{remark}
  Anstelle der von uns genutzten Orthonormalbasis $(v_1, v_2, v_3)$ hätte man auch eine beliebige nicht-orthonormale Basis von $\real^3$ aus Eigenvektoren von $B$ nutzen können, etwa $(\tilde{v}_1, \tilde{v}_2, \tilde{v}_3)$.
  Für die entsprechende Basiswechselmatrix $\tilde{C}$ gilt dann allerdings die Gleichheit $\tilde{C}^{-1} = \transpose{\tilde{C}}$ nicht mehr, weshalb auch noch $\tilde{C}^{-1}$ berechnet werden muss.
\end{remark}
